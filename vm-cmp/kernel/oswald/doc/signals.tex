% ------------------------------------------------------------------------+
% Copyright (c) 2001 by Punch Telematix. All rights reserved.             |
%                                                                         |
% Redistribution and use in source and binary forms, with or without      |
% modification, are permitted provided that the following conditions      |
% are met:                                                                |
% 1. Redistributions of source code must retain the above copyright       |
%    notice, this list of conditions and the following disclaimer.        |
% 2. Redistributions in binary form must reproduce the above copyright    |
%    notice, this list of conditions and the following disclaimer in the  |
%    documentation and/or other materials provided with the distribution. |
% 3. Neither the name of Punch Telematix nor the names of other           |
%    contributors may be used to endorse or promote products derived      |
%    from this software without specific prior written permission.        |
%                                                                         |
% THIS SOFTWARE IS PROVIDED ``AS IS'' AND ANY EXPRESS OR IMPLIED          |
% WARRANTIES, INCLUDING, BUT NOT LIMITED TO, THE IMPLIED WARRANTIES OF    |
% MERCHANTABILITY AND FITNESS FOR A PARTICULAR PURPOSE ARE DISCLAIMED.    |
% IN NO EVENT SHALL PUNCH TELEMATIX OR OTHER CONTRIBUTORS BE LIABLE       |
% FOR ANY DIRECT, INDIRECT, INCIDENTAL, SPECIAL, EXEMPLARY, OR            |
% CONSEQUENTIAL DAMAGES (INCLUDING, BUT NOT LIMITED TO, PROCUREMENT OF    |
% SUBSTITUTE GOODS OR SERVICES; LOSS OF USE, DATA, OR PROFITS; OR         |
% BUSINESS INTERRUPTION) HOWEVER CAUSED AND ON ANY THEORY OF LIABILITY,   |
% WHETHER IN CONTRACT, STRICT LIABILITY, OR TORT (INCLUDING NEGLIGENCE    |
% OR OTHERWISE) ARISING IN ANY WAY OUT OF THE USE OF THIS SOFTWARE, EVEN  |
% IF ADVISED OF THE POSSIBILITY OF SUCH DAMAGE.                           |
% ------------------------------------------------------------------------+

%
% $Id: signals.tex,v 1.1.1.1 2004/07/12 14:07:44 cvs Exp $
%

\subsection{Signals}

\subsubsection{Operation}

Signals are a set of 31 flags, that can be used by threads to communicate certain
conditions to other threads. The most significant bit of the signals
structure is reserved for future purposes (e.g. to control atomic operations
on signals.).

By means of logical operators, threads can set or query individual bit flags
or sets of bit flags in the signal structure.

Threads can also wait until a certain bit flag configuration appears in the
signals structure. The use and combination of the logical operators at query
or setting time can provide powerful but complex (read \textit{dangerous}) semantics.

\subsubsection{Signals Structure Definition}

The structure definition of a set of signals is as follows:

\bcode
\begin{verbatim}
 1: typedef struct x_Signals * x_signals;
 2:
 3: typedef struct x_Signals {
 4:   x_Event Event;
 5:   volatile x_flags flags;
 6: } x_Signals;
\end{verbatim}
\ecode

The relevant fields in the signals structure are the following:

\begin{itemize}
\item \txt{x\_signals$\rightarrow$Event} This is the universal event structure that is a field
in all threadable components or elements. It controls the synchronized access
to the signals component and the signalling between threads to indicate changes
in the signals structure.
\item \txt{x\_signals$\rightarrow$flags} The word in which the lower 31
bits are used to store the bit states of the different signals.
\end{itemize}


\subsubsection{Signal Set or Get Options}

The signal get and set functions take a logical option indicator that
controls the behavior of these functions. The different options and
in which context they are applicable are indicated in table
\ref{table:xo_options}.


\footnotesize
\begin{longtable}{||l|p{2cm}|p{9cm}||}
\hline
\hfill \textbf{Option} \hfill\null & \textbf{Applies to} & \textbf{Meaning} \\ 
\hline
\endhead
\hline
\endfoot
\endlastfoot
\hline

% \begin{table}[!ht]
%   \begin{center}
% \begin{tabular}{||>{\footnotesize}l<{\normalsize}|>{\footnotesize}l<{\normalsize}|>{\footnotesize}l<{\normalsize}||} \hline
% \textbf{Option} & \textbf{Applies to} & \textbf{Meaning} \\ \hline

\txt{xo\_or} &

\begin{minipage}[t]{2cm}
\txt{x\_signals\_get} \\
\txt{x\_signals\_set} \\
\end{minipage} &

\begin{minipage}[t]{8cm}
For the \txt{x\_signals\_get} function, this option indicates that the
query is satisfied when \textbf{any} of the requested flags becomes set during the
timeout window. When this call returns, the flag or flags that satisfied the
condition, \textbf{remain set} in the signals structure.\\

For the \txt{x\_signals\_set} function, this option indicates that the
flags that are passed as an argument are logically ORed with the actual flags
in the signal structure; the result of this logical or become the new flags
of the signals structure.
\end{minipage} \\

 & & \\

\txt{xo\_or\_clear} &

\begin{minipage}[t]{2cm}
\txt{x\_signals\_get} \\
\end{minipage} &

\begin{minipage}[t]{8cm}
This option indicates that the
query is satisfied when \textbf{any} of the requested flags becomes set during the
timeout window. Before the call returns, the flag or flags that satisfied the
condition, will be reset (to 0) before returning. Since upon such a return,
the signal flags have changed again, all waiting threads will be woken up to
recheck if their conditions are satisfied.
\end{minipage} \\

 & & \\

\txt{xo\_and} &

\begin{minipage}[t]{2cm}
\txt{x\_signals\_get} \\
\txt{x\_signals\_set} \\
\end{minipage} &

\begin{minipage}[t]{8cm}
For the \txt{x\_signals\_get} function, this option indicates that the
query is only satisfied when \textbf{all} of the requested flags become set during the
timeout window. When this call returns, the flag or flags that satisfied the
condition, remain set in the signals structure.\\

For the \txt{x\_signals\_set} function, this option indicates that the
flags that are passed as an argument or logically anded with the actual flags
in the signal structure; the result of this logical and become the new flags
of the signals structure.
\end{minipage} \\

 & & \\

\txt{xo\_and\_clear} &

\begin{minipage}[t]{2cm}
\txt{x\_signals\_get} \\
\end{minipage} &

\begin{minipage}[t]{8cm}
This option indicates that the
query is satisfied when \textbf{all} of the requested flags becomes set during the
timeout window. Before the call returns, the flags that satisfied the
condition, will be reset (to 0) before returning. Since upon such a return,
the signal flags have changed again, all waiting threads will be woken up to
recheck if their conditions are satisfied.
\end{minipage} \\



\hline 
\multicolumn{3}{c}{} \\
\caption{The different signal set and get options}
\label{table:xo_options}
\end{longtable}
\normalsize


%     \hline
%     \end{tabular}
%     \caption{The different signal set and get options}
%     \label{table:xo_options}
%   \end{center}
% \end{table}

\subsubsection{Creating a Set of Signals}

A set of 31 signalling flags is created by means of the call:\\

\txt{x\_status x\_signals\_create(x\_signals signals);}

This call results in signal event being set up. All 31 signals bits are initialized to
0. This function always returns \txt{xs\_success}. 

\subsubsection{Deleting a Set of Signals}

\subsection{Querying for a Condition}

\txt{x\_status x\_signals\_get(x\_signals si, x\_flags co, x\_option op, x\_flags ac, x\_sleep to);}

The different return values that this call can produce are summarized in
table \ref{table:rs_signals_get}.


\footnotesize
\begin{longtable}{||l|p{9cm}||}
\hline
\hfill \textbf{Return Value} \hfill\null & \textbf{Meaning}  \\ 
\hline
\endhead
\hline
\endfoot
\endlastfoot
\hline

% \begin{table}[!ht]
%   \begin{center}
%     \begin{tabular}{||>{\footnotesize}l<{\normalsize}|>{\footnotesize}c<{\normalsize}||} \hline
%     \textbf{Return Value} & \textbf{Meaning} \\ \hline

\txt{xs\_success} &

\begin{minipage}[t]{9cm}
The call succeeded and the query flags are satisfied according to the
\txt{op} argument given.
\end{minipage} \\

\txt{xs\_no\_instance} &

\begin{minipage}[t]{9cm}
The call did not succeed within the given timeout and with the given option.
\end{minipage} \\

\txt{xs\_bad\_option} &

\begin{minipage}[t]{9cm}
The call failed since the passed \txt{op} option argument isn't one of
\txt{xo\_or}, \txt{xo\_or\_clear}, \txt{xo\_and} or
\txt{xo\_and\_clear}.
\end{minipage} \\

\txt{xs\_bad\_context} &

\begin{minipage}[t]{9cm}
A timeout argument 'to' other than \txt{x\_no\_wait} was given from within a timer
or interrupt handler context.
\end{minipage} \\

\txt{xs\_deleted} &

\begin{minipage}[t]{9cm}
The signals structure has been deleted by another thread during the call.
\end{minipage} \\

\txt{xs\_bad\_element} &

\begin{minipage}[t]{9cm}
The passed reference \txt{si} doesn't refer to a signal event.
\end{minipage} \\



\hline 
\multicolumn{2}{c}{} \\
\caption{The different signal set and get options}
\label{table:xo_options}
\end{longtable}
\normalsize

%     \hline
%     \end{tabular}
%     \caption{Return Status for \txt{x\_signals\_get}}
%     \label{table:rs_signals_get}
%   \end{center}
% \end{table}

\subsection{Setting a Condition}

Signal bits in the signals structure can be set by means of the following
call:

\txt{x\_status x\_signals\_set(x\_signals si, x\_flags co, x\_option op);}

The different set options and in which context they are applicable are indicated in table
\ref{table:xo_options}.














