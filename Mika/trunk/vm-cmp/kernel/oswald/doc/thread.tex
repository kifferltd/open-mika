%
% $Id: thread.tex,v 1.1.1.1 2004/07/12 14:07:44 cvs Exp $
%

\subsection{Threads}

\subsubsection{Thread Structure Definition}

The structure definition of a thread is as follows:

\bcode
\begin{verbatim}
 1: typedef struct x_Thread * x_thread;
 2: typedef void (*x_entry)(void * argument);
 3: typedef void (*x_action)(x_thread thread);
 4:
 5: typedef struct x_Thread {
 6:   x_cpu cpu;
 7:   w_ubyte * sp;
 8:   w_ubyte * b_stack;
 9:   w_ubyte * e_stack;
10:   x_entry entry;
11:   void * argument;
12:   w_ushort id;
13:   w_ubyte a_prio;
14:   volatile w_ubyte c_prio;
15:   w_ubyte c_quantums;
16:   w_ubyte a_quantums;
17:   volatile w_ubyte state;
18:   void * xref;
19:   x_flags flags;
20:   volatile x_thread next;
21:   volatile x_thread previous;
22:   x_sleep sticks;
23:   x_sleep wakeup;
24:   volatile x_thread snext;
25:   x_action action;
26:   volatile x_thread l_waiting;
27:   volatile x_monitor waiting_for;
28:   volatile x_thread l_competing;
29:   volatile x_event competing_for;
30:   volatile x_event l_owned;
31:   w_size m_count;
32: } x_Thread;
\end{verbatim}
\ecode

The relevant fields in the thread structure are the following:

\begin{itemize}
\item \txt{x\_thread$\rightarrow$cpu} This is the CPU specific
placeholder for things like saved stack pointer, program counter and other
arguments. Please refer to the cpu section for more information on specific
fields in this structure. TODO
\item \txt{x\_thread$\rightarrow$s\_sp} For some ports, this is the field
in which the stack pointer is saved at the moment of a thread switch.
\item \txt{x\_thread$\rightarrow$b\_stack} This is the lowest address of
the user supplied stack space. For the user, it is the first available byte of
the memory that was allocated and passed as stack space at thread creation
time. Note that on most processors and hosts, the stack grows downwards and
this is thus, from this viewpoint, the end of the stack.
\item \txt{x\_thread$\rightarrow$e\_stack} This is the opposite of the
previous field, the end of the stack, or from the point of view of
processors that have a downwards growing stack, the place where the pushes
of stack items begins. Note that this value is word aligned, so it could be
that it is not the end of the memory region that has been passed at thread
creation time.
\item \txt{x\_thread$\rightarrow$entry} This is the entry function for
the thread. At thread start, this is the function that will be called. The
thread will end when it returns from this function.
\item \txt{x\_thread$\rightarrow$argument} This is the argument, given at
thread creation time, that is passed to the function described in the
previous field. See also the type definition at line 2.
\item \txt{x\_thread$\rightarrow$id} The identification number of a
thread. This is a unique, 16 bits number in \oswald.
\item \txt{x\_thread$\rightarrow$a\_prio} The assigned priority of a
thread.
\item \txt{x\_thread$\rightarrow$c\_prio} The current priority of a
thread. Note that during some operations, namely mutex and monitor
operations, the threads priority can be boosted to fight priority inversion.
This is the threads priority at any moment in \oswald. When the threads
priority has been boosted, after the operation, this field will be reset to
the \txt{a\_prio} field value.
\item \txt{x\_thread$\rightarrow$c\_quantums} Threads execute for a
number of time slices, before they hand over the processor to a thread at
the same priority, that is ready to run. This field holds the number of
slices or quantums that the thread still has left.
\item \txt{x\_thread$\rightarrow$a\_quantums} This is the number of
quantums that the \txt{c\_quantums} field will be 'reloaded' with when
the thread has exhausted its
number of quantums. It is fixed now, but could become an argument that can
be modulated later.
\item \txt{x\_thread$\rightarrow$state} The state that the thread is in;
more information on thread states can be found below.
\item \txt{x\_thread$\rightarrow$xref} A cross reference pointer that can
be used to point back to user supplied structures. It is not used by any
Owald function.
\item \txt{x\_thread$\rightarrow$flags} The flags that this thread has.
The flags can carry \oswald information and user information. Flags are
described more in detail below.
\item \txt{x\_thread$\rightarrow$next} Linked list field. See also next
field.
\item \txt{x\_thread$\rightarrow$previous} Threads are kept in a doubly
linked circular list; each priority control block has two lists, one that
keeps the ready to run threads and one that keeps the pending threads; 
this and the previous field form this list.
\item \txt{x\_thread$\rightarrow$sticks} When a thread is pending,
waiting for an event to happen or just sleeping, this field holds the number
of ticks that it still has to go before being woken up.
\item \txt{x\_thread$\rightarrow$wakeup} At wakeup time, \oswald saves the
current time in this field; between the time of wakeup and the time
that the thread starts executing again, time could have elapsed allready. We
need to compensate this time in the event routines that take a timing
argument.
\item \txt{x\_thread$\rightarrow$snext} The pending threads are kept in a
singly linked list, that is woven through this field.
\item \txt{x\_thread$\rightarrow$action} When a thread is pending and
becomes alive again because the \txt{sticks} field became 0, this is the
function pointer that is called to execute whatever householding stuff that
needs done, determined by the pending state the thread was in.
\item \txt{x\_thread$\rightarrow$l\_waiting} When threads are waiting on
monitors, this field is used to form the singly linked list of threads that
are waiting on the same monitor. Note that a thread can only be waiting on a
single monitor at a time.
\item \txt{x\_thread$\rightarrow$waiting\_for} When the thread is in a
waiting list of a monitor, the monitor it is waiting for is given in this
field.
\item \txt{x\_thread$\rightarrow$l\_competing} A thread can be competing
for an event (mutex, monitor, queue, i.e. any of the events); the list of
threads that are competing for the event is competing formed by this field. Again note that a thread
can only be competing for a single event.
\item \txt{x\_thread$\rightarrow$competing\_for} The event that a thread
is competing for.
\item \txt{x\_thread$\rightarrow$l\_owned} Events like mutexes and
monitors are 'owned' by a thread, when they are 'locked'. Threads can own
multiple events at the same time. The linked list of events that are owned
by a thread, starts at this field, and is further formed by the
\txt{event$\rightarrow$l\_owned} field.
\item \txt{x\_thread$\rightarrow$m\_count} When a thread waits on a
monitor, the number of times it has locked or entered the monitor, is saved
in this field, so that it can be restored later, when the thread becomes
owner again of the monitor.
\end{itemize}

\subsubsection{Thread States}

The different states that a thread can be in are summarised in table
\ref{table:thread_states}.

A thread is only ready to be scheduled and to run, when its state is not 0.
Any other state in table \ref{table:thread_states} indicates that the thread
is waiting for a certain event to happen and is not ready to run.

Note that the thread states that are not 0 (ready) and that indicate that
the thread is waiting for a certain event to happen, correspond with the
numerical value of the event type. I.e. the event type indicator of a mutex
event has value 1, for a semaphore event the value is 4 and so forth.


\footnotesize
\begin{longtable}{||l|c|p{9cm}||}
\hline
\hfill \textbf{State Name} \hfill\null & \textbf{Value} & \textbf{Meaning}  \\ 
\hline
\endhead
\hline
\endfoot
\endlastfoot
\hline


% \begin{table}[!ht]
%   \begin{center}
%    \begin{tabular}{||>{\footnotesize}l<{\normalsize}|>{\footnotesize}c<{\normalsize}|>{\footnotesize}c<{\normalsize}||} \hline
%     \textbf{State Name} & \textbf{Value} & \textbf{Meaning} \\ \hline

\txt{xt\_ready} & 0 &
\begin{minipage}[t]{8.5cm}
This is the state when a thread is ready to run, or is
running. In \oswald, there is no special state for the currently running
thread. 
\end{minipage} \\

\txt{xt\_mutex} & 1 &
\begin{minipage}[t]{8.5cm}
The state of a thread that is waiting for a mutex to become
available for locking.
\end{minipage} \\

\txt{xt\_queue} & 2 &
\begin{minipage}[t]{8.5cm}
The state of a thread waiting for elements to be posted on an empty queue or
waiting for space to become available in a full queue.
\end{minipage} \\

\txt{xt\_mailbox} & 3 &
\begin{minipage}[t]{8.5cm}
The state of a thread waiting for data to become available in a mailbox or
for the mailbox to become empty so that a message can be posted.
\end{minipage} \\

\txt{xt\_semaphore} & 4 &
\begin{minipage}[t]{8.5cm}
The state of a thread waiting for a semaphores count to become greater than
0.
\end{minipage} \\

\txt{xt\_signals} & 5 &
\begin{minipage}[t]{8.5cm}
The state of a thread waiting on a set of signals to satisfy the get
request.
\end{minipage} \\

\txt{xt\_monitor} & 6 &
\begin{minipage}[t]{8.5cm}
The state of a thread that is trying to enter a monitor.
\end{minipage} \\

\txt{xt\_block} & 7 &
\begin{minipage}[t]{8.5cm}
The state of a thread that is waiting for a free block to become available
in a block pool.
\end{minipage} \\

\txt{xt\_map} & 8 &
\begin{minipage}[t]{8.5cm}
The state of a thread that is waiting for a bit to become 0 (reset) in an event
bitmap.
\end{minipage} \\

\txt{xt\_waiting} & 9 &
\begin{minipage}[t]{8.5cm}
The state of a thread that is waiting on a monitor.
\end{minipage} \\

\txt{xt\_suspended} & 10 &
\begin{minipage}[t]{8.5cm}
The state of a thread that is suspended.
\end{minipage} \\

\txt{xt\_sleeping} & 11 &
\begin{minipage}[t]{8.5cm}
The state of a thread that is sleeping after calling the
\txt{x\_thread\_sleep} function.
\end{minipage} \\

\txt{xt\_ended} & 13 &
\begin{minipage}[t]{8.5cm}
The state of a thread that returned from the start function.
\end{minipage} \\



\hline 
\multicolumn{3}{c}{} \\
\caption{Thread states}
\label{table:thread_states}
\end{longtable}
\normalsize


%     \hline
%     \end{tabular}
%     \caption{Thread states}
%     \label{table:thread_states}
%   \end{center}
% \end{table}

\subsubsection{Creating a Thread}

A thread is created by means of the call:

\txt{x\_status x\_thread\_create(x\_thread thread, x\_entry entry, void
*argument, x\_ubyte *stack, x\_size size, x\_size prio, x\_flags flags);}

The arguments to this call are:
\begin{enumerate}
\item \txt{thread}, the thread structure pointer. This structure will contain the internal state of a thread.

\item \txt{entry}, the pointer to a function that the thread will start with. The type definition \txt{x\_entry}
is \txt{typedef void (*x\_entry)(void * argument)}, so \txt{x\_entry}
is a function pointer of a function that returns void and takes a void
pointer as its single argument. The thread stops and is put an an
\txt{xt\_ended} state when this function returns.

\item \txt{argument}, the argument that will be passed to the entry
function. It's use is defined by the programmer. \oswald will not use this
argument for any reason and will just pass it along to the entry function.

\item \txt{stack}, points to a suitably sized block of memory that will
be used as stack space by the thread.

\item \txt{size}, indicates the number of bytes that the \txt{stack}
argument has. Note that the size allowed must be larger or equal than
\txt{MIN\_STACK\_SIZE}.

\item \txt{prio}, is the priority of the thread. Real time priorities start from 1
and go up to 63. Soft priorities (round robin threads) start from 64 and go
up to 128. The lower the value of this \txt{prio} argument, the higher
the priority of the thread.

\item \txt{flags}, this argument can have only 2 different values:
\begin{itemize}
\item \txt{TF\_SUSPENDED}, indicates that the thread should not start
immediately but is put in a suspended state. The thread must explicitely be
started with an \txt{x\_thread\_resume} call.

\item \txt{TF\_START} to indicate that the thread should start immediately,
i.e. not in a suspended state.
\end{itemize}

\end{enumerate}

The possible return values for this call are sumarized in table
\ref{table:x_thread_create}.

\footnotesize
\begin{longtable}{||l|p{9cm}||}
\hline
\hfill \textbf{Return Value} \hfill\null & \textbf{Meaning}  \\ 
\hline
\endhead
\hline
\endfoot
\endlastfoot
\hline


% \begin{table}[!ht]
%   \begin{center}
%     \begin{tabular}{||>{\footnotesize}l<{\normalsize}|>{\footnotesize}c<{\normalsize}||} \hline
%     \textbf{Return Value} & \textbf{Meaning} \\ \hline

\txt{xs\_success} &

\begin{minipage}[t]{9cm}
The thread was succesfully created.
\end{minipage} \\

\txt{xs\_bad\_argument} &

\begin{minipage}[t]{9cm}
Some argument to the thread create was bad, e.g. the stack size was less
than \txt{MIN\_STACK\_SIZE}, or the entry function was NULL or the flags argument
was not one of the allowed values.
\end{minipage} \\


\hline 
\multicolumn{2}{c}{} \\
\caption{Return Status for \txt{x\_thread\_create}}
\label{table:x_thread_create}
\end{longtable}
\normalsize


%     \hline
%     \end{tabular}
%     \caption{Return Status for \txt{x\_thread\_create}}
%     \label{table:x_thread_create}
%   \end{center}
% \end{table}

\subsubsection{Suspending a Thread}

There are two different calls available to suspend a thread:

\txt{x\_status x\_thread\_suspend(x\_thread thread);} \\
\txt{x\_status x\_thread\_suspend\_cb(x\_thread thread, x\_ecb cb, void *arg);}

Both functions will suspend a thread but the second call implements a
callback facility which makes it potentially much safer to suspend a thread.

The thread that is to be suspended is passed as \txt{thread} argument. A
thread can call this function to suspend itself.

A thread that is in a suspended state can be deleted with the
\txt{x\_thread\_delete} call or resumed with the
\txt{x\_thread\_resume} function.

It is potentially unsafe to suspend a thread when the caller doesn't know
exactly in what state a thread is. Suppose that thread A is just performing
a memory allocation with the \txt{x\_mem\_alloc} call. The memory package
that implements this call will lock a mutex while manipulating internal
memory structures. The owner of this mutex is the thread that is performing
the call, in this example thread A. If thread A was to be suspended when
owning this lock, no other thread in the entire system could perform a
memory allocation call or release call, potentially bringing the whole
system to a virtual standstill.

To give the programmer the possibility to remedy these kind of situations,
the callback system is available. The \txt{x\_ecb} declares the following
type of function.

\txt{typedef x\_boolean (*x\_ecb)(x\_event event, void *argument);}

It is a function pointer, which function takes as arguments an event pointer
and an argument pointer and returns a boolean.

When a thread is suspended with the callback variant, the callback function
is called for each event, a mutex or a monitor, that the thread being
suspended, is owner of. The \txt{argument} field of the
\txt{x\_thread\_suspend\_cb} function, is passed on as the second argument
to the callback function and is programmer defined.

In the callback function, any actions can be undertaken under control of the
programmer. She could decide to check wether the owned event is a mutex or
monitor and call the relevant release function for it. When the return value
of the callback, for any invocation, was not \txt{true}, the
thread will \textbf{not} be suspended and the return status of the
\txt{x\_thread\_suspend\_cb} call will be \txt{xs\_owner} to indicate
that the thread being suspended owned at least one event.

If all the invocations of the callback returned the \txt{false} value.
The thread will be suspended, regardless wether it still owned events or
not. If the thread still owned events, the return status will also be
\txt{xs\_owner}. When the thread didn't own any events anymore, it
returns \txt{xs\_success}.

For a full overview of the return values of either suspend call, refer to
table \ref{table:x_thread_suspend}.

\footnotesize
\begin{longtable}{||l|p{9cm}||}
\hline
\hfill \textbf{Return Value} \hfill\null & \textbf{Meaning}  \\ 
\hline
\endhead
\hline
\endfoot
\endlastfoot
\hline


% \begin{table}[!ht]
%   \begin{center}
%     \begin{tabular}{||>{\footnotesize}l<{\normalsize}|>{\footnotesize}c<{\normalsize}||} \hline
%     \textbf{Return Value} & \textbf{Meaning} \\ \hline

\txt{xs\_success} &

\begin{minipage}[t]{9cm}
The thread was succesfully suspended and doesn't own any events anymore.
\end{minipage} \\

\txt{xs\_no\_instance} &

\begin{minipage}[t]{9cm}
The thread being suspended was either in a \txt{xt\_ended} state or a
\txt{xt\_suspended} state allready.
\end{minipage} \\

\txt{xs\_owner} &

\begin{minipage}[t]{9cm}
This indicates that the thread still owns events. Wether the thread was
really suspended depends on the context. When the suspension was tried with
the non callback variant, the thread is suspended.

When the suspension was performed with the callback variant, it depends on
the context. If the callback returned at least a single \txt{false}
value. The thread will not be suspended. If the callback always returned
\txt{true} as value, the thread will be suspended, but still ows events.
I.e. the callback did not succeed in releasing the owned events.
\end{minipage} \\


\hline 
\multicolumn{2}{c}{} \\
\caption{Return Status for \txt{x\_thread\_suspend}}
\label{table:x_thread_suspend}
\end{longtable}
\normalsize

%     \hline
%     \end{tabular}
%     \caption{Return Status for \txt{x\_thread\_suspend}}
%     \label{table:x_thread_suspend}
%   \end{center}
% \end{table}

\subsubsection{Resuming a Thread}

A thread can be resumed with the call:

\txt{x\_status x\_thread\_resume(x\_thread thread);}

The \txt{thread} argument to the call is the thread that needs resuming.
The return values of this call are simple and summarised in table
\ref{table:x_thread_resume}.

\footnotesize
\begin{longtable}{||l|p{9cm}||}
\hline
\hfill \textbf{Return Value} \hfill\null & \textbf{Meaning}  \\ 
\hline
\endhead
\hline
\endfoot
\endlastfoot
\hline


% \begin{table}[h]
%   \begin{center}
%     \begin{tabular}{||>{\footnotesize}l<{\normalsize}|>{\footnotesize}c<{\normalsize}||} \hline
%     \textbf{Return Value} & \textbf{Meaning} \\ \hline

\txt{xs\_success} &

\begin{minipage}[t]{9cm}
The thread was succesfully resumed.
\end{minipage} \\

\txt{xs\_no\_instance} &

\begin{minipage}[t]{9cm}
The thread being resumed is not in a suspended state, i.e. there was no
preceeding \txt{x\_thread\_suspend} call for this thread.
\end{minipage} \\


\hline 
\multicolumn{2}{c}{} \\
\caption{Return Status for \txt{x\_thread\_resume}}
\label{table:x_thread_resume}
\end{longtable}
\normalsize

%     \hline
%     \end{tabular}
%     \caption{Return Status for \txt{x\_thread\_resume}}
%     \label{table:x_thread_resume}
%   \end{center}
% \end{table}

\subsubsection{Making a Thread Sleep \& Wake up}

A thread can go to sleep for a number a certain number of ticks with the
following call:

\txt{void x\_thread\_sleep(x\_sleep ticks);}

Note that a \txt{ticks} argument of 0 has no effect. A thread can be made
to sleep for an eternal amount of time by given the \txt{ticks} argument
the value of \txt{x\_eternal}. 

A thread that has been put to sleep, for an eternal time or a limited time,
can be woken up with the following call:

\txt{x\_status x\_thread\_wakeup(x\_thread thread);}

When the thread was not sleeping, the status returned is
\txt{xs\_no\_instance}, otherwise, \txt{xs\_success} is returned.

\subsubsection{Deleting a Thread}

A thread can only be deleted when its state is either \txt{xt\_ended}
because if returned from the entry function, or when it's state is
\txt{xt\_suspended}, because of one of the \txt{x\_thread\_suspend}
calls.

\subsubsection{Changing Thread Priority}

A threads priority can be changed with the call:

\txt{x\_status x\_thread\_priority\_set(x\_thread thread, x\_size newprio);}

This function will return \txt{xs\_success} when the thread did change
priority or \txt{xs\_bad\_argument} if the \txt{newprio} argument was
out of bounds.

\subsubsection{Identifying the Current Thread}

The \txt{x\_thread} pointer of the currently running thread can be
found with the following call:

\txt{x\_thread x\_thread\_current(void);}

The 16 bits unique ID of a thread is found in the
\txt{x\_thread$\rightarrow$id} field of the \txt{x\_thread} structure.

Note that this ID is unique and is recycled. This means that the thread ID
does not convey any information about the start order of threads; e.g. a
thread with ID 128 could have been created after the thread with ID 130.





