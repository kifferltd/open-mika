% ------------------------------------------------------------------------+
% Copyright (c) 2001 by Punch Telematix. All rights reserved.             |
%                                                                         |
% Redistribution and use in source and binary forms, with or without      |
% modification, are permitted provided that the following conditions      |
% are met:                                                                |
% 1. Redistributions of source code must retain the above copyright       |
%    notice, this list of conditions and the following disclaimer.        |
% 2. Redistributions in binary form must reproduce the above copyright    |
%    notice, this list of conditions and the following disclaimer in the  |
%    documentation and/or other materials provided with the distribution. |
% 3. Neither the name of Punch Telematix nor the names of other           |
%    contributors may be used to endorse or promote products derived      |
%    from this software without specific prior written permission.        |
%                                                                         |
% THIS SOFTWARE IS PROVIDED ``AS IS'' AND ANY EXPRESS OR IMPLIED          |
% WARRANTIES, INCLUDING, BUT NOT LIMITED TO, THE IMPLIED WARRANTIES OF    |
% MERCHANTABILITY AND FITNESS FOR A PARTICULAR PURPOSE ARE DISCLAIMED.    |
% IN NO EVENT SHALL PUNCH TELEMATIX OR OTHER CONTRIBUTORS BE LIABLE       |
% FOR ANY DIRECT, INDIRECT, INCIDENTAL, SPECIAL, EXEMPLARY, OR            |
% CONSEQUENTIAL DAMAGES (INCLUDING, BUT NOT LIMITED TO, PROCUREMENT OF    |
% SUBSTITUTE GOODS OR SERVICES; LOSS OF USE, DATA, OR PROFITS; OR         |
% BUSINESS INTERRUPTION) HOWEVER CAUSED AND ON ANY THEORY OF LIABILITY,   |
% WHETHER IN CONTRACT, STRICT LIABILITY, OR TORT (INCLUDING NEGLIGENCE    |
% OR OTHERWISE) ARISING IN ANY WAY OUT OF THE USE OF THIS SOFTWARE, EVEN  |
% IF ADVISED OF THE POSSIBILITY OF SUCH DAMAGE.                           |
% ------------------------------------------------------------------------+

%
% $Id: monitor.tex,v 1.1.1.1 2004/07/12 14:07:44 cvs Exp $
%

\subsection{Monitors}

\subsection{Operation}

Monitors are a means for threads to control shared variables and communicate
amongst threads about the status of shared variables. Usually, monitors are
used with the following template or pattern.

\bcode
\begin{verbatim}
   1 : <a 'monitor' is created by a single thread only>
   2 : ...
   3 : status = x_monitor_enter(monitor, x_eternal);
   4 : if (status == xs_success) {
   5 :   ...
   6 :   while ( <a certain condition X is not true> ) {
   7 :     x_monitor_wait(monitor, x_eternal);
   8 :   }
   9 :   ...
  10 :   <code to manipulate the condition X>
  11 :   status = x_monitor_notify_all(monitor);
  12 :   if (status == xs_success) {
  13 :     status = x_monitor_exit(monitor);
  14 :   }
  15 : }
  16 :
\end{verbatim}
\ecode

Note that the return values of the \oswald calls are not checked for errors.
Of course good programming requires this checking, which is only ommitted in
the above example to keep the code small. (If you know a better excuse, let
me know)

When several threads want to use a monitor to synchronize amongst
themselves, only one single monitor is in use in most simple cases.
Sometimes, several monitors are used for other synchronization, more complex patterns.
This pattern is the basic pattern for use of monitors.

The area where threads manipulate shared data in a safe way, starts from
line 3 and extends to line 13, in the above example. After line 13, the area
of common manipulation stops.

It is important to notice in this example that several threads are executing
this sample of code, virtually at the same time.

There is only a single line, in the above example, where the thread that is
executing this code, has not locked the whole range of lines 3 to 13; this
is place where the currently executing thread 'waits' for a certain
condition to become true, i.e. line 7. Waiting on a monitor means that the
currently executing thread temporarily unlocks the monitor, so that other
threads can relock the monitor and manipulate the variable X such that the 
while loop on line 6, will be exited. After exiting the while loop, the
monitor will again be owned by the thread, exactly as it was on line 5.

When a thread is waiting, other threads will be able, one by one, after
getting or locking the monitor, to manipulate the variable or condition X,
referred to on lines 6 and 10. As soon as a thread has changed this variable
X, so that the while loop on line 6, will be exited, the thread that has
been waiting, needs to be 'notified' of this change so that it can
reevaluate the while condition on line 6. 

This is done by means of the \txt{x\_monitor\_notify} statement at line
11. The notification does not unlock the monitor, it merely signals a single
other thread that it should reevaluate the while loop, as soon as the
current thread unlocks the monitor, at line 13.

When multiple condition variables are being used for synchronization in more
complex pieces of code, the \txt{x\_monitor\_notify\_all} must be used.
This call will wake up all threads waiting on the monitor to reevaluate the
while loop they are waiting in, to see if the condition has changed in their
favor so that they can exit the while loop.

It is \textbf{very important} to note that at line 6, always a
\textbf{while} clause should be used and not an \textbf{if} clause. The
thread waiting does not know, how many threads are going to be notified (all
or a single) and in the case of multiple synchronization variables X, he
does not know whether his own condition will become false, so that he can
exit the loop. So each time the thread is woken up from the wait, he needs
to reevaluate the condition, before deciding to proceed with the rest of the
code, i.e. it requires a \textbf{while} in stead of an \textbf{if} clause.

Another important aspect of monitors is that a thread can lock a monitor
several times. I.e., unlike mutexes, there is no \txt{x\_deadlock} status
when a monitor is entered several times. In other words, when a monitor is
entered Z times, it needs to be exited Z times by the same thread before it
really is released by the thread. Monitor locks (entering) and unlocks
(exiting) are thus nestable. The lock count is kept in the monitor
structure.

\subsection{Monitor Structure Definition}

The structure definition of a monitor is as follows:

\bcode
\begin{verbatim}
 1: typedef struct x_Monitor * x_monitor;
 2:
 3: typedef struct x_Monitor {
 4:   x_Event Event;
 5:   volatile x_thread owner;
 6:   volatile w_ushort count;
 7:   volatile w_ushort n_waiting;
 8:   volatile x_thread l_waiting;
 9: } x_Monitor;
\end{verbatim}
\ecode

The relevant fields in the monitor structure are the following:

\begin{itemize}
\item \txt{x\_monitor$\rightarrow$Event} This is the universal event structure that is a field
in all threadable components or elements. It controls the synchronized access
to the monitor component and the signalling between threads to indicate changes
in the monitor structure.
\item \txt{x\_monitor$\rightarrow$owner} The current owner thread of the monitor
or \txt{NULL} when there is no current owner of the monitor.
\item \txt{x\_monitor$\rightarrow$count} The number of times that the
current owner thread has locked the monitor. Monitor entries can be nested
and must be exited as many times as a thread has entered the monitor.
\item \txt{x\_monitor$\rightarrow$n\_waiting} The number of threads in
the list, indicated by the previous field, that are waiting on this monitor.
\item \txt{x\_monitor$\rightarrow$l\_waiting} The linked list of threads
that are waiting on the monitor. See below for an explanation of 'waiting on
a monitor'. This field is continued via the
\txt{x\_thread$\rightarrow$l\_waiting} field of a thread.
\end{itemize}

\subsubsection{Creating a Monitor}

A monitor is created by means of the following call:

\txt{x\_status x\_monitor\_create(x\_monitor monitor);}

This will initialize the monitor so that it can be used in the subsequent
calls as described further. Note that the creation of a monitor by a certain
thread, does not make the thread owner of the monitor. To acquire a monitor,
the thread should perform the \txt{x\_monitor\_enter} call as described
below.

\subsubsection{Deleting a Monitor}

A monitor can be deleted by means of the following call:

\txt{x\_status x\_monitor\_delete(x\_monitor monitor);}

The different return values that this call can produce are summarized
in table \ref{table:monitor_delete}.  


\footnotesize
\begin{longtable}{||l|p{9cm}||}
\hline
\hfill \textbf{Return Value} \hfill\null & \textbf{Meaning}  \\ 
\hline
\endhead
\hline
\endfoot
\endlastfoot
\hline

% \begin{table}[!ht]
%   \begin{center}
%     \begin{tabular}{||>{\footnotesize}l<{\normalsize}|>{\footnotesize}c<{\normalsize}||} \hline
%     \textbf{Return Value} & \textbf{Meaning} \\ \hline

\txt{xs\_success} &
\begin{minipage}[t]{9cm}
The monitor has been successfully deleted and no other threads were
attempting an operation on it.
\end{minipage} \\

\txt{xs\_not\_owner} &
\begin{minipage}[t]{9cm}
The current thread does not own the monitor. A thread must own the monitor
or the monitor owner must be \txt{NULL} before it can be successfully
deleted.
\end{minipage} \\

\txt{xs\_waiting} &
\begin{minipage}[t]{9cm}
Some other threads were attempting an operation on the monitor. These
threads were successfully informed about the deletion of the monitor. In any
case, this could indicate bad programming.
\end{minipage} \\

\txt{xs\_incomplete} &

\begin{minipage}[t]{9cm}
Some threads were attempting an operation on the monitor but
haven't acknowledged yet that they are aborting this operation. Proceed
with caution in further deleting the monitor, like e.g. releasing the
memory of the monitor.
\end{minipage} \\

\txt{xs\_deleted} &

\begin{minipage}[t]{9cm}
Some other thread has been deleting this element already.
\end{minipage} \\

\txt{xs\_bad\_element} &

\begin{minipage}[t]{9cm}
The passed \txt{monitor} structure is not pointing to a valid monitor
structure.
\end{minipage} \\


\hline 
\multicolumn{2}{c}{} \\
\caption{Return Status for \txt{x\_monitor\_delete}}
\label{table:monitor_delete}
\end{longtable}
\normalsize

%     \hline
%     \end{tabular}
%     \caption{Return Status for \txt{x\_monitor\_delete}}
%     \label{table:monitor_delete}
%   \end{center}
% \end{table}

\subsubsection{Entering a Monitor}

\txt{x\_status x\_monitor\_enter(x\_monitor monitor, x\_sleep to);}

As already noted above, a thread can lock or enter a monitor several times.
When a thread already owns a monitor, the lock count is just incremented by
1.

The different return values that this call can produce are summarized
in table \ref{table:monitor_enter}.  


\footnotesize
\begin{longtable}{||l|p{9cm}||}
\hline
\hfill \textbf{Return Value} \hfill\null & \textbf{Meaning} \\ 
\hline
\endhead
\hline
\endfoot
\endlastfoot
\hline

% \begin{table}[!ht]
%   \begin{center}
%     \begin{tabular}{||>{\footnotesize}l<{\normalsize}|>{\footnotesize}c<{\normalsize}||} \hline
%     \textbf{Return Value} & \textbf{Meaning} \\ \hline

\txt{xs\_success} &
\begin{minipage}[t]{9cm}
The call succeeded and the current thread owns the monitor. If the current
thread owned the monitor already, the lock count has been incremented,
otherwise it is now 1.
\end{minipage} \\

\txt{xs\_no\_instance} &
\begin{minipage}[t]{9cm}
The monitor could not be locked by the current thread in the timeout window
given by the \txt{to} argument.
\end{minipage} \\

\txt{xs\_bad\_context} &
\begin{minipage}[t]{9cm}
A timeout value \txt{to} other than \txt{x\_no\_wait} has been given
in the context of a timer handler or interrupt handler.
\end{minipage} \\

\txt{xs\_deleted} &

\begin{minipage}[t]{9cm}
Another thread has deleted the monitor while this current thread was
trying to enter it.
\end{minipage} \\

\txt{xs\_bad\_element} &

\begin{minipage}[t]{9cm}
The passed \txt{monitor} structure is not pointing to a valid monitor
structure.
\end{minipage} \\


\hline 
\multicolumn{2}{c}{} \\
\caption{Return Status for \txt{x\_monitor\_enter}}
\label{table:monitor_enter}
\end{longtable}
\normalsize

%     \hline
%     \end{tabular}
%     \caption{Return Status for \txt{x\_monitor\_enter}}
%     \label{table:monitor_enter}
%   \end{center}
% \end{table}

\subsubsection{Waiting on a Monitor}

A thread can wait on a monitor by means of the following call:

\txt{x\_status x\_monitor\_wait(x\_monitor monitor, x\_sleep to);}

The different return values that this call can produce are summarized
in table \ref{table:monitor_wait}.  


\footnotesize
\begin{longtable}{||l|p{9cm}||}
\hline
\hfill \textbf{Return Value} \hfill\null & \textbf{Meaning} \\ 
\hline
\endhead
\hline
\endfoot
\endlastfoot
\hline

% \begin{table}[!ht]
%   \begin{center}
%     \begin{tabular}{||>{\footnotesize}l<{\normalsize}|>{\footnotesize}c<{\normalsize}||} \hline
%     \textbf{Return Value} & \textbf{Meaning} \\ \hline

\txt{xs\_success} &
\begin{minipage}[t]{9cm}
The call succeeded and the current thread has regained the lock on the
monitor. It should reevaluate the while condition.
\end{minipage} \\

\txt{xs\_not\_owner} &
\begin{minipage}[t]{9cm}
The current thread did not own the monitor. A thread should own the monitor,
by issuing a \txt{x\_monitor\_enter} before attempting a wait on a
monitor.
\end{minipage} \\

\txt{xs\_bad\_context} &
\begin{minipage}[t]{9cm}
A timeout value \txt{to} other than \txt{x\_no\_wait} has been given
in the context of a timer handler or interrupt handler. Could indicate bad
programming, to say it softly...
\end{minipage} \\

\txt{xs\_deleted} &

\begin{minipage}[t]{9cm}
Another thread has deleted the monitor while this current thread was
trying to enter it.
\end{minipage} \\

\txt{xs\_bad\_element} &

\begin{minipage}[t]{9cm}
The passed \txt{monitor} structure is not pointing to a valid monitor
structure.
\end{minipage} \\


\hline 
\multicolumn{2}{c}{} \\
\caption{Return Status for \txt{x\_monitor\_wait}}
\label{table:monitor_wait}
\end{longtable}
\normalsize


%     \hline
%     \end{tabular}
%     \caption{Return Status for \txt{x\_monitor\_wait}}
%     \label{table:monitor_wait}
%   \end{center}
% \end{table}

\subsubsection{Notifying Threads Waiting on a Monitor}

\begin{itemize}
\item Notifying a single thread, that is waiting on a monitor. The first in
line waiting. This done by means of the following call:

\txt{x\_status x\_monitor\_notify(x\_monitor monitor);}

The different return values that this call can produce are summarized
in table \ref{table:monitor_notify}.  

\item Notifying all threads that are waiting on a monitor.

\txt{x\_status x\_monitor\_notify\_all(x\_monitor monitor);}

The different return values that this call can produce are summarized
in table \ref{table:monitor_notify_all}.
\end{itemize}


\footnotesize
\begin{longtable}{||l|p{9cm}||}
\hline
\hfill \textbf{Return Value} \hfill\null & \textbf{Meaning} \\ 
\hline
\endhead
\hline
\endfoot
\endlastfoot
\hline

% \begin{table}[!ht]
%   \begin{center}
%     \begin{tabular}{||>{\footnotesize}l<{\normalsize}|>{\footnotesize}c<{\normalsize}||} \hline
%     \textbf{Return Value} & \textbf{Meaning} \\ \hline

\txt{xs\_success} &
\begin{minipage}[t]{9cm}
The call succeeded and the first thread that was waiting has been woken up.
The current thread still owns the lock though.
\end{minipage} \\

\txt{xs\_not\_owner} &
\begin{minipage}[t]{9cm}
The current thread did not own the monitor. A thread should own the monitor,
by issuing a \txt{x\_monitor\_enter} before attempting a notify on a
monitor.
\end{minipage} \\

\txt{xs\_deleted} &

\begin{minipage}[t]{9cm}
Another thread has deleted the monitor while this current thread was
trying to notify it.
\end{minipage} \\

\txt{xs\_bad\_element} &

\begin{minipage}[t]{9cm}
The passed \txt{monitor} structure is not pointing to a valid monitor
structure.
\end{minipage} \\


\hline 
\multicolumn{2}{c}{} \\
\caption{Return Status for \txt{x\_monitor\_notify}}
\label{table:monitor_notify}
\end{longtable}
\normalsize


%     \hline
%     \end{tabular}
%     \caption{Return Status for \txt{x\_monitor\_notify}}
%     \label{table:monitor_notify}
%   \end{center}
% \end{table}


\footnotesize
\begin{longtable}{||l|p{9cm}||}
\hline
\hfill \textbf{Return Value} \hfill\null & \textbf{Meaning} \\ 
\hline
\endhead
\hline
\endfoot
\endlastfoot
\hline

% \begin{table}[!ht]
%   \begin{center}
%     \begin{tabular}{||>{\footnotesize}l<{\normalsize}|>{\footnotesize}c<{\normalsize}||} \hline
%     \textbf{Return Value} & \textbf{Meaning} \\ \hline

\txt{xs\_success} &
\begin{minipage}[t]{9cm}
The call succeeded and all threads that were waiting are woken up.
The current thread still owns the lock though.
\end{minipage} \\

\txt{xs\_not\_owner} &
\begin{minipage}[t]{9cm}
The current thread did not own the monitor. A thread should own the monitor,
by issuing a \txt{x\_monitor\_enter} before attempting a notify all on a
monitor.
\end{minipage} \\

\txt{xs\_deleted} &

\begin{minipage}[t]{9cm}
Another thread has deleted the monitor while this current thread was
trying to notify it.
\end{minipage} \\

\txt{xs\_bad\_element} &

\begin{minipage}[t]{9cm}
The passed \txt{monitor} structure is not pointing to a valid monitor
structure.
\end{minipage} \\


\hline 
\multicolumn{2}{c}{} \\
\caption{Return Status for \txt{x\_monitor\_notify\_all}}
\label{table:monitor_notify_all}
\end{longtable}
\normalsize

%    \hline
%     \end{tabular}
%     \caption{Return Status for \txt{x\_monitor\_notify\_all}}
%     \label{table:monitor_notify_all}
%   \end{center}
% \end{table}

\subsubsection{Exiting a Monitor}

\txt{x\_status x\_monitor\_exit(x\_monitor monitor);}

The different return values that this call can produce are summarized
in table \ref{table:monitor_exit}.  


\footnotesize
\begin{longtable}{||l|p{9cm}||}
\hline
\hfill \textbf{Return Value} \hfill\null & \textbf{Meaning} \\ 
\hline
\endhead
\hline
\endfoot
\endlastfoot
\hline

% \begin{table}[!ht]
%   \begin{center}
%     \begin{tabular}{||>{\footnotesize}l<{\normalsize}|>{\footnotesize}c<{\normalsize}||} \hline
%     \textbf{Return Value} & \textbf{Meaning} \\ \hline

\txt{xs\_success} &
\begin{minipage}[t]{9cm}
The call succeeded and if the lock count of the monitor has reached 0, the
monitor has been released by the current thread. If the count did not reach
0, the current thread still owns the monitor.
\end{minipage} \\

\txt{xs\_deleted} &

\begin{minipage}[t]{9cm}
Another thread has deleted the monitor while this current thread was
trying to exit a monitor.
\end{minipage} \\

\txt{xs\_bad\_element} &

\begin{minipage}[t]{9cm}
The passed \txt{monitor} structure is not pointing to a valid monitor
structure.
\end{minipage} \\


\hline 
\multicolumn{2}{c}{} \\
\caption{Return Status for \txt{x\_monitor\_exit}}
\label{table:monitor_exit}
\end{longtable}
\normalsize

%     \hline
%     \end{tabular}
%     \caption{Return Status for \txt{x\_monitor\_exit}}
%     \label{table:monitor_exit}
%   \end{center}
% \end{table}

\subsubsection{Removing a Waiting Thread}

Sometimes, it is required that a certain thread removes another thread from
a list of waiting threads. This can be accomplished wit the following call:

\txt{x\_status x\_monitor\_stop\_waiting(x\_monitor monitor, x\_thread thread);}

The passed \txt{thread} is removed from the waiting list and as a result,
the status \txt{xs\_success} is returned. If \txt{thread} was not
found in the waiting list of a monitor, the status \txt{xs\_no\_instance}
is returned.


