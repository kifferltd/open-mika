%
% $Id: mutex.tex,v 1.1.1.1 2004/07/12 14:07:44 cvs Exp $
%

\subsection{Mutexes}

\subsubsection{Operation}

\subsubsection{Mutex Structure Definition}

The structure definition of a mutex is as follows:

\bcode
\begin{verbatim}
 1: typedef struct x_Mutex * x_mutex;
 2:
 3: typedef struct x_Mutex {
 4:   x_Event Event;
 5:   volatile x_thread owner;
 6: } x_Mutex;
\end{verbatim}
\ecode

The relevant fields in the mutex structure are the following:

\begin{itemize}
\item \txt{x\_mutex$\rightarrow$Event} This is the universal event structure that is a field
in all threadable components or elements. It controls the synchronized access
to the component and the signalling between threads.
\item \txt{x\_mutex$\rightarrow$owner} This field indicates which
thread has the mutex lock; when this field is \txt{NULL}, the
mutex is not owned by any thread.
\end{itemize}

\subsubsection{Creating a Mutex}

A mutex is created by means of the following call:

\txt{x\_status x\_mutex\_create(x\_mutex mutex);}

This call will initialize a mutex that is referred to by the \txt{mutex}
argument. Memory for the mutex must be allocated by the caller. Creating the
mutex does not result in locking the mutex. A mutex is always created in an
unlocked mode. A subsequent call to \txt{x\_mutex\_lock} is required for
locking a mutex.

\subsubsection{Deleting a Mutex}

A mutex is deleted by means of the following call:

\txt{x\_status x\_mutex\_delete(x\_mutex mutex);}

This call will delete a mutex that is referred to by the \txt{mutex} argument.
The mutex can be in any state, either locked or unlocked; when the mutex is
in the locked state, it is required that the thread that makes the call is
owner of the mutex, otherwise the \txt{xs\_not\_owner} status is
returned.

\subsubsection{Acquiring a Mutex}

A mutex is 'acquired' or locked by means of the following call:

\txt{x\_status x\_mutex\_lock(x\_mutex mutex, x\_sleep timeout);}

The kernel will try to lock the mutex referred to by the \txt{mutex} argument,
for the calling thread. This attempt will be done during the \txt{timeout}
window. If the call succeeds within the \txt{timeout} window, the
\txt{xs\_success} status is returned. 

If the mutex is deleted within the \txt{timeout} window, the returned status is
\txt{xs\_deleted}. If the call did not succeed within the given time, the
returned status is \txt{xs\_no\_instance}.

If the calling thread has a higher priority than the thread that currently
owns the mutex, the currently owning thread is temporarily boosted to the
calling thread priority, to defeat priority inversion. When the owning
thread releases the mutex, it will be reset to his assigned priority.

The waiting list on the mutex is priority ordered such that the highest
waiting priority thread will become the next owner.

\subsubsection{Releasing a Mutex}

A mutex is released by a calling thread by means of the following call:

\txt{x\_status x\_mutex\_unlock(x\_mutex mutex);}

When the calling thread does
not currently own the mutex referred to by the \txt{mutex} argument, the
\txt{xs\_not\_owner} status is returned, in any other case, the call must
succeed (note that a mutex can not be deleted by a thread that is not the
owner) and will return the \txt{xs\_success} status.

When the calling threads priority has been boosted to defeat the priority
inversion problem, the thread priority will be reset to the priority it
was assigned at creation time, when the calling thread doesn't own any other
mutexes. Should the calling thread own any other mutexes, its priority will
be set to the highest waiting thread priority on any mutexes it owns.

