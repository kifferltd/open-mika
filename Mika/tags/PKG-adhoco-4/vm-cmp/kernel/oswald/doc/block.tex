% ------------------------------------------------------------------------+
% Copyright (c) 2001 by Punch Telematix. All rights reserved.             |
%                                                                         |
% Redistribution and use in source and binary forms, with or without      |
% modification, are permitted provided that the following conditions      |
% are met:                                                                |
% 1. Redistributions of source code must retain the above copyright       |
%    notice, this list of conditions and the following disclaimer.        |
% 2. Redistributions in binary form must reproduce the above copyright    |
%    notice, this list of conditions and the following disclaimer in the  |
%    documentation and/or other materials provided with the distribution. |
% 3. Neither the name of Punch Telematix nor the names of other           |
%    contributors may be used to endorse or promote products derived      |
%    from this software without specific prior written permission.        |
%                                                                         |
% THIS SOFTWARE IS PROVIDED ``AS IS'' AND ANY EXPRESS OR IMPLIED          |
% WARRANTIES, INCLUDING, BUT NOT LIMITED TO, THE IMPLIED WARRANTIES OF    |
% MERCHANTABILITY AND FITNESS FOR A PARTICULAR PURPOSE ARE DISCLAIMED.    |
% IN NO EVENT SHALL PUNCH TELEMATIX OR OTHER CONTRIBUTORS BE LIABLE       |
% FOR ANY DIRECT, INDIRECT, INCIDENTAL, SPECIAL, EXEMPLARY, OR            |
% CONSEQUENTIAL DAMAGES (INCLUDING, BUT NOT LIMITED TO, PROCUREMENT OF    |
% SUBSTITUTE GOODS OR SERVICES; LOSS OF USE, DATA, OR PROFITS; OR         |
% BUSINESS INTERRUPTION) HOWEVER CAUSED AND ON ANY THEORY OF LIABILITY,   |
% WHETHER IN CONTRACT, STRICT LIABILITY, OR TORT (INCLUDING NEGLIGENCE    |
% OR OTHERWISE) ARISING IN ANY WAY OUT OF THE USE OF THIS SOFTWARE, EVEN  |
% IF ADVISED OF THE POSSIBILITY OF SUCH DAMAGE.                           |
% ------------------------------------------------------------------------+

%$\Rightarrow$
%
% $Id: block.tex,v 1.1.1.1 2004/07/12 14:07:44 cvs Exp $
%

\subsection{Block Pools}

\subsubsection{Operation}

Block pools provide a convenient and fast way to allocate memory of fixed
size blocks in \oswald. A block allocation, when there still are blocks
available in the block pool, is done in constant time. Therefore block pools
should be used primarily in interrupt handlers and in these contexts where
it is known beforehand that only a limited or fixed amount of memory will be needed. 

For situations where this is not the case, the usual \txt{x\_alloc\_mem}
and \txt{x\_free\_mem} routines that allocate and free different sized
memory blocks from the heap, should be used.

Note that the sizes of the blocks, are always rounded to a word boundary,
i.e. the size will be rounded up to a number of bytes that is divisible by 4.
Some processors are not able or not optimized to handle pointers that do not
begin on a word boundary.

Also note that when a block is allocated, it contains a hidden pointer to
the block pool it belongs to. Therefore, when e.g. a block size of 16 bytes
or 4 words is required, a block will be of size 16 + 4 = 20 bytes. This
should be taken into consideration when allocating memory for a block pool.

There is a convenience function that should preferably be used when
calculating the size of memory for the blocks in a pool:

\txt{w\_size x\_block\_calc(w\_size block\_size, w\_size num\_blocks);}

This routine will return the number of \textbf{bytes} required to allocate
for a block pool with \txt{num\_blocks} blocks that require a usable (not
including the hidden pointer) space of \txt{block\_size} bytes.

\subsubsection{Block Structure Definition}

The structure definition of a block is as follows:

\bcode
\begin{verbatim}
 1: typedef struct x_Boll * x_boll;
 2: typedef struct x_Block * x_block;
 3:
 4: typedef struct x_Boll {
 5:   union {
 6:     x_block block;
 7:     x_boll next;
 8:   } header;
 9:   w_ubyte bytes[1];
10: } x_Boll;
11:
12: typedef struct x_Block {
13:   x_Event Event;
14:   w_size boll_size;
15:   w_size bolls_max;
16:   w_size space_size;
17:   volatile w_size bolls_left;
18:   volatile x_boll bolls;
19: } x_Block;
\end{verbatim}
\ecode

The \txt{x\_Boll} structure is used internally by the block pool code to
organize a linked list of free blocks. It accomodates the pointer back to
the block pool, when a block is in use (allocated) by means of the
\txt{x\_boll$\rightarrow$header.block} field and when a block is on the
free list of blocks in a block pool, the field
\txt{x\_boll$\rightarrow$header.next} points to the next free block in
line, or \txt{NULL} when no more free blocks are in the list. The
\txt{x\_boll$\rightarrow$bytes} represents the usable memory space in
each block. It is not really an array of 1 bytes, but extends beyond the end
of the structure to represent the useable space.

The relevant fields in the block structure are the following:

\begin{itemize}
\item \txt{x\_block$\rightarrow$Event} This is the universal event structure that is a field
in all threadable components or elements. It controls the synchronized access
to the block component and the signalling between threads to indicate changes
in the block structure.
\item \txt{x\_block$\rightarrow$boll\_size} The size of a single block in
bytes; this value is rounded up to a value that is divisible by 4 (word
size) so it maybe not the same as the value passed at the creation time of a
block pool.
\item \txt{x\_block$\rightarrow$bolls\_max} The number of blocks that the
block pool had after it has been created and before any blocks are allocated
from it.
\item \txt{x\_block$\rightarrow$space\_size} The number of bytes that are
available in the memory area that is passed at creation time and from which
the blocks are carved.
\item \txt{x\_block$\rightarrow$bolls\_left} The number of blocks that
are still available in the pool for allocating.
\item \txt{x\_block$\rightarrow$bolls} The linked list of free blocks in
the block pool.
\end{itemize}

\subsubsection{Creating a Block Pool}

A block pool is created by means of the following call:

\txt{x\_status x\_block\_create(x\_block block, w\_size bs, void * space, w\_size ss);}

Create a block pool, with each block having a size of \txt{bs} bytes and
where the memory for these blocks is indicated by \txt{space}. This
memory area has space for \txt{ss} bytes.

This call will generate the number blocks that can be carved out the memory
space and will assign them to the block pool.

The different return values that this call can produce are summarized
in table \ref{table:block_create}.  

\footnotesize
\begin{longtable}{||l|p{9cm}||}
\hline
\hfill \textbf{Return Value} \hfill\null & \textbf{Meaning}  \\ 
\hline
\endhead
\hline
\endfoot
\endlastfoot
\hline


% \begin{table}[!ht]
%   \begin{center}
%     \begin{tabular}{||>{\footnotesize}l<{\normalsize}|>{\footnotesize}c<{\normalsize}||} \hline
%     \textbf{Return Value} & \textbf{Meaning} \\ \hline

\txt{xs\_success} &
\begin{minipage}[t]{9cm}
The call succeeded and the block pool has been set up properly. Note that
the number of blocks that the pool has (allocated or released) is indicated
by the structure field \txt{block$\rightarrow$bolls\_max}.
\end{minipage} \\

\txt{xs\_no\_instance} &

\begin{minipage}[t]{9cm}
The size of the memory area passed and the block size given to create are
impossible. I.e., not a single block can be carved out of this memory area.
Note that there is a hidden cost of 4 bytes per block and that the size of a
block is rounded up to 4, first. 
\end{minipage} \\

\hline 
\multicolumn{2}{c}{} \\
\caption{Return Status for \txt{x\_block\_create}}
\label{table:block_create}
\end{longtable}
\normalsize

%     \hline
%     \end{tabular}
%     \caption{Return Status for \txt{x\_block\_create}}
%     \label{table:block_create}
%   \end{center}
% \end{table}

\subsubsection{Deleting a Block Pool}

A block pool can be deleted by means of the following call:

\txt{x\_status x\_block\_delete(x\_block block);}

Please note that deleting a block while there are still blocks in use by
other threads or this thread, leads to undefined behavior. Before
attempting the call, it is wise to check the fields
\txt{block$\rightarrow$bolls\_max} and
\txt{block$\rightarrow$bolls\_left}, The former indicates the number of
blocks that the pool has and the latter indicates the number of blocks not
allocated. When both are equal, there are no blocks in use. Beware that
while comparing both, another thread could have allocated a block. So make
sure that the thread doing the comparison, is running a the highest priority
and the pool is not being used by an interrupt handler. Deleting a pool of
an interrupt handler is a bad idea anyway.

The different return values that this call can produce are summarized
in table \ref{table:block_delete}.  

\footnotesize
\begin{longtable}{||l|p{9cm}||}
\hline
\hfill \textbf{Return Value} \hfill\null & \textbf{Meaning} \\ 
\hline
\endhead
\hline
\endfoot
\endlastfoot
\hline


% \begin{table}[!ht]
%   \begin{center}
%     \begin{tabular}{||>{\footnotesize}l<{\normalsize}|>{\footnotesize}c<{\normalsize}||} \hline
%     \textbf{Return Value} & \textbf{Meaning} \\ \hline

\txt{xs\_waiting} &
\begin{minipage}[t]{9cm}
Some other threads were attempting an allocate operation on the block pool.
\end{minipage} \\

\txt{xs\_incomplete} &

\begin{minipage}[t]{9cm}
Some threads were attempting an allocate operation on the block pool but
haven't acknowledged yet that they are aborting this operation. Proceed
with caution in further deleting the block pool, like e.g. releasing the
memory of the pool.
\end{minipage} \\

\txt{xs\_deleted} &

\begin{minipage}[t]{9cm}
Some other thread has been deleting this element already.
\end{minipage} \\

\txt{xs\_bad\_element} &

\begin{minipage}[t]{9cm}
The passed \txt{block} structure is not pointing to a valid block pool
structure.
\end{minipage} \\

\hline 
\multicolumn{2}{c}{} \\
\caption{Return Status for \txt{x\_block\_delete}}
\label{table:block_delete}
\end{longtable}
\normalsize

%     \hline
%     \end{tabular}
%     \caption{Return Status for \txt{x\_block\_delete}}
%     \label{table:block_delete}
%   \end{center}
% \end{table}

\subsubsection{Allocating a Block from the Pool}

A block can be allocated from the pool with the following call:

\txt{x\_status x\_block\_allocate(x\_block block, void ** bytes, x\_sleep to);}

The different return values that this call can produce are summarized
in table \ref{table:block_allocate}.  


\footnotesize
\begin{longtable}{||l|p{9cm}||}
\hline
\hfill \textbf{Return Value} \hfill\null & \textbf{Meaning} \\ 
\hline
\endhead
\hline
\endfoot
\endlastfoot
\hline


% \begin{table}[!ht]
%   \begin{center}
%     \begin{tabular}{||>{\footnotesize}l<{\normalsize}|>{\footnotesize}c<{\normalsize}||} \hline
%     \textbf{Return Value} & \textbf{Meaning} \\ \hline

\txt{xs\_success} &
\begin{minipage}[t]{9cm}
The call succeeded and the pointer indicated by \txt{*bytes} is pointing to the
allocated block.
\end{minipage} \\

\txt{xs\_no\_instance} &

\begin{minipage}[t]{9cm}
There was no block available for the calling thread within the timeout
specified. The pointer indicated by \txt{*bytes} is not changed during
the call.
\end{minipage} \\

\txt{xs\_deleted} &

\begin{minipage}[t]{9cm}
Another thread has deleted the block pool while this current thread was
trying to allocate a block.
\end{minipage} \\

\txt{xs\_bad\_element} &

\begin{minipage}[t]{9cm}
The passed \txt{block} structure is not pointing to a valid block pool
structure.
\end{minipage} \\

\hline 
\multicolumn{2}{c}{} \\
\caption{Return Status for \txt{x\_block\_allocate}}
\label{table:block_allocate}
\end{longtable}
\normalsize


%     \hline
%     \end{tabular}
%     \caption{Return Status for \txt{x\_block\_allocate}}
%     \label{table:block_allocate}
%   \end{center}
% \end{table}

\subsubsection{Releasing a Block from the Pool}

\txt{x\_status x\_block\_release(x\_block block, void * bytes);}

The different return values that this call can produce are summarized
in table \ref{table:block_release}.  

\footnotesize
\begin{longtable}{||l|p{9cm}||}
\hline
\hfill \textbf{Option} \hfill\null & \textbf{Meaning} \\ 
\hline
\endhead
\hline
\endfoot
\endlastfoot
\hline


% \begin{table}[!ht]
%   \begin{center}
%     \begin{tabular}{||>{\footnotesize}l<{\normalsize}|>{\footnotesize}c<{\normalsize}||} \hline
%     \textbf{Return Value} & \textbf{Meaning} \\ \hline

\txt{xs\_success} &
\begin{minipage}[t]{9cm}
The call succeeded and the block has been given back to the pool.
\end{minipage} \\

\txt{xs\_deleted} &

\begin{minipage}[t]{9cm}
Another thread has deleted the block pool while this current thread was
trying to release a block.
\end{minipage} \\

\txt{xs\_bad\_element} &

\begin{minipage}[t]{9cm}
The passed \txt{block} structure is not pointing to a valid block pool
structure.
\end{minipage} \\

\hline 
\multicolumn{2}{c}{} \\
\caption{Return Status for \txt{x\_block\_release}}
\label{table:block_release}
\end{longtable}
\normalsize

%     \hline
%     \end{tabular}
%     \caption{Return Status for \txt{x\_block\_release}}
%     \label{table:block_release}
%   \end{center}
% \end{table}




