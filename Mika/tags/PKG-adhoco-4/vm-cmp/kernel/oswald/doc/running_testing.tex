% ------------------------------------------------------------------------+
% Copyright (c) 2001 by Punch Telematix. All rights reserved.             |
%                                                                         |
% Redistribution and use in source and binary forms, with or without      |
% modification, are permitted provided that the following conditions      |
% are met:                                                                |
% 1. Redistributions of source code must retain the above copyright       |
%    notice, this list of conditions and the following disclaimer.        |
% 2. Redistributions in binary form must reproduce the above copyright    |
%    notice, this list of conditions and the following disclaimer in the  |
%    documentation and/or other materials provided with the distribution. |
% 3. Neither the name of Punch Telematix nor the names of other           |
%    contributors may be used to endorse or promote products derived      |
%    from this software without specific prior written permission.        |
%                                                                         |
% THIS SOFTWARE IS PROVIDED ``AS IS'' AND ANY EXPRESS OR IMPLIED          |
% WARRANTIES, INCLUDING, BUT NOT LIMITED TO, THE IMPLIED WARRANTIES OF    |
% MERCHANTABILITY AND FITNESS FOR A PARTICULAR PURPOSE ARE DISCLAIMED.    |
% IN NO EVENT SHALL PUNCH TELEMATIX OR OTHER CONTRIBUTORS BE LIABLE       |
% FOR ANY DIRECT, INDIRECT, INCIDENTAL, SPECIAL, EXEMPLARY, OR            |
% CONSEQUENTIAL DAMAGES (INCLUDING, BUT NOT LIMITED TO, PROCUREMENT OF    |
% SUBSTITUTE GOODS OR SERVICES; LOSS OF USE, DATA, OR PROFITS; OR         |
% BUSINESS INTERRUPTION) HOWEVER CAUSED AND ON ANY THEORY OF LIABILITY,   |
% WHETHER IN CONTRACT, STRICT LIABILITY, OR TORT (INCLUDING NEGLIGENCE    |
% OR OTHERWISE) ARISING IN ANY WAY OUT OF THE USE OF THIS SOFTWARE, EVEN  |
% IF ADVISED OF THE POSSIBILITY OF SUCH DAMAGE.                           |
% ------------------------------------------------------------------------+

%
% $Id: running_testing.tex,v 1.1.1.1 2004/07/12 14:07:44 cvs Exp $
%

\chapter{Running \& Testing Oswald}

\section{Introduction to the Oswald Tests}

Oswald comes with a whole test suite that can be compiled and run, to test
the internal consistency. 

At this moment\footnote{October 2001}, the number of lines of code of the
Oswald functionality has passed the 11000 lines mark, while the number of
lines of test code is slightly larger than that.

Since we try to adhere to the Extreme Programming principle as much as
possible, this is a good sign of health. The code base to test Oswald is
about the same size, even a bit larger, than the functionality code base.

The code that forms the test base is to be found in the
\textsf{kernel/oswald/tests} directory, if you consider that Oswald comes
with the Wonka Virtual Machine.
The best documentation for the tests is ultimately found in the source code
in that directory.

The main control of the tests start in the file \textsf{main\_test.c} that
can be reconfigured to run all the tests or only some individual tests.

Each of the components of Oswald, like e.g. have their own file. So the
tests for the normal Oswald monitors, resides in the file
\textsf{monitor\_test.c}.

The components of Oswald that are events (monitors, queues, semaphores,
mutexes, ...) are tested by means of threads that run together and form a so
called finite state machine; the finite state machine puts the component
under test in a certain state and checks wether a function call on the
components puts or leaves the component in a correct state.

Note that all tests\footnote{Except for the module test code.} are run in
loops, i.e. they never end. At the end of a finite state machine run,
allocated memory is returned, the component is deleted and the test is
re-initialized to start again.

Most of the parameters and arguments used in tests are assigned by means of
the \textsf{x\_random} call that returns a random number of a thread by
thread basis; i.e. each thread has its own random number generator to reduce
the amount of locking required to get a random number\footnote{Note that
this random number generator is for testing purposes only. Don't try to
build any encryption or security based components with it.}.

\section{Non FSM Tests}

Some tests, like the module test and especially the memory allocation test
routines, don't run by means of a finite state machine. The memory test for
instance, just spawns a lot of threads that try to allocate, free, discard
and reallocate memory in a loop. When all memory is exhausted, it is
released and the process continues.

During the memory tests, the allocated blocks are filled with a pattern and
a certain tag is assigned to the blocks. Before releasing the blocks, they
are checked by the thread that allocated them to see if the pattern is still
intact.

Also, when allocating cleared memory, the memory test thread checks if all
bytes have been properly initialized to zero. For reallocation tests, the
thread checks wether the available bytes have been copied correctly, when
the reallocation resulted in a new block of memory being used.

For more information about these and other tests, we refer to the source
code itself, according to the principle \textit{"Use the source, Luke..."}.

\section{Compiling \& Running the Tests}

In the full Wonka tree, at the topmost level, you need to specify, by means
of an environment variable, the directory where you are. In this example,
I'll use as top directory \textsf{/home/buytaert/cvs/open-wonka}. You have
to modify this for your own environment. So before we start compiling, you
have to, assuming the bash shell and the '\$' sign is the prompt of the
shell, give in:

\begin{verbatim}
$ export WONKA_TOP=/home/buytaert/cvs/open-wonka
$ jam oswaldtests
\end{verbatim}

We further assume that you are running on an Intel X86 based linux box.
This should result in a program called \textsf{oswaldtests} that can be
found in:

\begin{verbatim}
./build-x86-linux/kernel/oswald/tests/
\end{verbatim}

You can run the program from the \textsf{\$WONKA\_TOP} directory with the
command:

\begin{verbatim}
./build-x86-linux/kernel/oswald/tests/oswaldtests
\end{verbatim}

Running this program should give you the messages passing by of all the
tests that are being run. You should see that the test output is mixed,
which is normal since the tests are running in muliple threads.

If you want to run the tests without any debuggin or runtime checking code
being put in the normal functionality of Oswald, you can issue the following
build commands:

\begin{verbatim}
$ export WONKA_TOP=/home/buytaert/cvs/open-wonka
$ jam -sDEBUG=false oswaldtests
\end{verbatim}

The result of which will be the same executable file as described above. The
Oswald tests use their own logging routine, so the printing of messages
should be the same as with the debug version of the Oswald tests.
