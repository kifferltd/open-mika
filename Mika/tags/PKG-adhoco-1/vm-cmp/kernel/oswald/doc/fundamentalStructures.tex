\chapter{Fundamental Structures and Programming Interface}

\section{Fundamental Structures}

% ------------------------------------------------------------------------+
% Copyright (c) 2001 by Punch Telematix. All rights reserved.             |
%                                                                         |
% Redistribution and use in source and binary forms, with or without      |
% modification, are permitted provided that the following conditions      |
% are met:                                                                |
% 1. Redistributions of source code must retain the above copyright       |
%    notice, this list of conditions and the following disclaimer.        |
% 2. Redistributions in binary form must reproduce the above copyright    |
%    notice, this list of conditions and the following disclaimer in the  |
%    documentation and/or other materials provided with the distribution. |
% 3. Neither the name of Punch Telematix nor the names of other           |
%    contributors may be used to endorse or promote products derived      |
%    from this software without specific prior written permission.        |
%                                                                         |
% THIS SOFTWARE IS PROVIDED ``AS IS'' AND ANY EXPRESS OR IMPLIED          |
% WARRANTIES, INCLUDING, BUT NOT LIMITED TO, THE IMPLIED WARRANTIES OF    |
% MERCHANTABILITY AND FITNESS FOR A PARTICULAR PURPOSE ARE DISCLAIMED.    |
% IN NO EVENT SHALL PUNCH TELEMATIX OR OTHER CONTRIBUTORS BE LIABLE       |
% FOR ANY DIRECT, INDIRECT, INCIDENTAL, SPECIAL, EXEMPLARY, OR            |
% CONSEQUENTIAL DAMAGES (INCLUDING, BUT NOT LIMITED TO, PROCUREMENT OF    |
% SUBSTITUTE GOODS OR SERVICES; LOSS OF USE, DATA, OR PROFITS; OR         |
% BUSINESS INTERRUPTION) HOWEVER CAUSED AND ON ANY THEORY OF LIABILITY,   |
% WHETHER IN CONTRACT, STRICT LIABILITY, OR TORT (INCLUDING NEGLIGENCE    |
% OR OTHERWISE) ARISING IN ANY WAY OUT OF THE USE OF THIS SOFTWARE, EVEN  |
% IF ADVISED OF THE POSSIBILITY OF SUCH DAMAGE.                           |
% ------------------------------------------------------------------------+

%
% $Id: event.tex,v 1.1.1.1 2004/07/12 14:07:44 cvs Exp $
%

\subsection{The Event Structure}

The event structure is the fundamental component in any element of \oswald
that is used to pass messages between threads (mutexes, queues, signals,
...). It is always the first field in any event component of \oswald and is
included in such a structure, not referred to.

Its definition is the following:

\bcode
\begin{verbatim}
 1: typedef struct x_Event * x_event;
 2:
 3: typedef struct x_Event {
 4:   volatile w_ushort flags_type;
 5:   volatile w_ushort n_competing;
 6:   volatile x_thread l_competing;
 7:   volatile x_event l_owned;
 8: } x_Event;
\end{verbatim}
\ecode

The relevant fields in the event structure are the following:

\begin{itemize}
\item \txt{x\_event$\rightarrow$type\_flags} The different flags to record the
state of the event, merged with the type of the event to save space. The type of an event is an
element of the \txt{x\_type} enumeration. This enumeration can be found
back in the \fl{event.h} header file.
\item \txt{x\_event$\rightarrow$n\_competing} The number of threads that
are competing on a change in this event.
\item \txt{x\_event$\rightarrow$l\_competing}
The list of threads that are competing for a change in this event. This list
is continued through the \txt{x\_thread$\rightarrow$l\_competing} field.
The event that a thread in this list is competing for, is recorded in the
\txt{x\_thread$\rightarrow$competing\_for} field.
\item \txt{x\_event$\rightarrow$l\_owned} Certain events can be owned by
a thread (mutexes or monitors). A thread that owns events is linked through
this field.
\end{itemize}

There is a special function that can be used to 'wait until an event has
been deleted'.

\txt{x\_status x\_event\_join(void * event, x\_window timeout);}

This function will make a thread wait untill the event (mutex, queue,
mailbox, ...) has been deleted. When the event has been deleted and no
threads are attempting any operation on it, this call returns
\txt{xs\_success}, if some threads were still trying to abort any
operation on the event, the call will return \txt{xs\_incomplete} to
indicate that some threads did not complete the call for the operation yet.

The calling thread will attempt to join the event for as long a number of
ticks as indicated by the \txt{timeout} argument.

This functionality could be used in a thread that allocates memory for the
element and creates it as a service for other threads. After allocation and
creation, the service thread can execute a join call on the event to wait until
all threads have stopped using the element.


%
% $Id: pcb.tex,v 1.1.1.1 2004/07/12 14:07:44 cvs Exp $
%

\subsection{The Priority Control Block Structure}


\subsection{The Thread Structure}

\subsection{The Timer Structure}

\section{Programming Interfaces}

\subsection{Conventions}

The programming interfaces for \oswald are designed to be simple, intuitive,
consistent and small in number. Each function call in \oswald starts with the
\txt{'x\_'}. The next word is the element to which the call refers, again
followed by an underscore and then a description of what the function does
with the element. This description has been carefully chosen so that it can
be used consistently for other elements, where applicable; E.g.

\txt{x\_status x\_mailbox\_send(x\_mailbox mailbox, word * data, x\_sleep timeout);}

\txt{x\_status x\_queue\_send(x\_queue, word * data, x\_sleep timeout);}

The first function describes how to send a message to a mailbox, while the
second is trying to send a message to a queue.

\subsection{Structure Fields}

For each element that can be used in \oswald, e.g. mutexes, semaphores, ... the
fields of the structure will be explained shortly. A programmer is allowed
to look at these fields, but \textbf{should not} be tempted to change any of
these fields.

Another important note is that \oswald will update these fields internally in
a controlled way; therefore reading of these fields by a thread can give an
indication of its value but it is never guaranteed to another thread is
changing or has changed the field so that it's value is not constent
anymore.

In other words, a programmer can decide to read the fields and take
according action, but must always assume that the fields value could
have changed allready and must therefore accommodate for such changes.

For example, before locking a mutex, a programmer can decide to peek at the
field \txt{x\_mutex$\rightarrow$owner} and see that it is NULL. He can
not assume that the mutex is unlocked, since the \txt{owner} field of the
mutex could have been changed allready. If it is not NULL, she may neither
assume that the mutex is locked by the thread indicated by the
\txt{x\_mutex$\rightarrow$owner} field. The only thing that can be
deduced is that the thread indicated by this field \textbf{has} locked the mutex in
the past and \textbf{could} very well be the current owner.

Note that the fields in the structure that can be changed by different
threads, have the qualifier \txt{volatile} to prevent the compiler of
optimizing away possibly changing values, as constant values. In other
words, fields that don't have the \txt{volatile} qualifier, will not be
changed by different threads and can be treated as constants during the
lifetime of the event.

\subsection{Return Values}

The API is consistent also in the return values. Most functions in \oswald return
an \txt{x\_status} enumeration. When the call succeeds, this is always the
same value of \txt{xs\_success}, whatever the element the call was tried on.

This implies that all data that is to be changed by a called function, is to
be passed as a pointer.

The generic meaning of the return values and the situation where they can occur are summarized 
in table \ref{table:xstatus}.

\footnotesize
\begin{longtable}{|c|c|c|}
\hline
\textbf{Status} & \textbf{Returned by} & \textbf{Meaning} \\ \hline
\endhead
\hline
\endfoot
\endlastfoot
\txt{xs\_success} &

All functions. &

\begin{minipage}[t]{7.5cm}
When an operation succeeds, this is the returned status.
\end{minipage} \\
 & & \\

\txt{xs\_no\_instance} &

\begin{minipage}[t]{3.5cm}
All functions that try to set or try to get something, or functions that try
to achieve an operation being done, mostly in with the constraint of a
timing window during which the operation can be tried.
\end{minipage} &

\begin{minipage}[t]{7.5cm}
When the specified operation did not succeed within the given time frame,
this is the status that is returned, or when the specified operation did not
perform the operation. For more specific information, we refer to the status
return values of each operation.
\end{minipage} \\

 & & \\

\txt{xs\_bad\_context} &

\begin{minipage}[t]{3.5cm}
All functions that take a time out value during which the operation is
tried.
\end{minipage} &

\begin{minipage}[t]{7.5cm}
When the operation was tried from an interrupt or timer (not
yet implemented) and the given timeout value was not \txt{x\_no\_wait}, this is
the returned status. Interrupt handlers or timer handler functions should
not become suspended, therefore the only timing argument allowed is
\txt{x\_no\_wait}.
\end{minipage} \\

 & & \\

\txt{xs\_not\_owner} &

\begin{minipage}[t]{3.5cm}
\txt{x\_mutex\_unlock} 
\txt{x\_mutex\_delete} 
\txt{x\_monitor\_exit} 
\txt{x\_monitor\_delete} 
\txt{x\_monitor\_notify} 
\txt{x\_monitor\_wait} 
\end{minipage} &

\begin{minipage}[t]{7.5cm}
When the thread tries to unlock or delete a mutex or monitor that it doesn't own, this is the
returned status.
\end{minipage} \\

 & & \\

\txt{xs\_deadlock} &

\txt{x\_mutex\_lock} &

\begin{minipage}[t]{7.5cm}
When the thread tries to lock a mutex that it owns already, this is the
returned status. Monitors in \oswald can be locked multiple times by a
thread. No deadlock can appear on monitors.
\end{minipage} \\

 & & \\

\txt{xs\_bad\_option} &

\begin{minipage}[t]{3.5cm}
\txt{x\_signals\_put} \\
\txt{x\_signals\_get} 
\end{minipage} &

\begin{minipage}[t]{7.5cm}
When the option given to any signals get or set operation does not apply to
the given situation, this is the returned status. More information can be
found in the section on signals.\\
\end{minipage} \\

 & & \\

\txt{xs\_deleted} &

\begin{minipage}[t]{3.5cm}
All operations on events (mutexes, monitors, signals, maps, block,
semaphores, queues).
\end{minipage} &

\begin{minipage}[t]{7.5cm}
When the event structure was deleted when some threads were trying an
operation on it, this is the status that is returned to the waiting threads.
Note that this nearly always means bad programming! The programmer should
treat this as an error and should fix it before releasing the software.
\end{minipage} \\

 & & \\

\txt{xs\_waiting} &

\begin{minipage}[t]{3.5cm}
When the structure was deleted when some threads were still waiting, this is the
status that is returned to the waiting threads.
\end{minipage} &

\begin{minipage}[t]{7.5cm}
When the structure was deleted and some threads where waiting on it, the
caller that deletes the structure will get this status returned.
Note that this nearly always means bad programming! The programmer should
treat this as an error and should fix it before releasing the software.
\end{minipage} \\

\hline 
\multicolumn{3}{c}{} \\
\caption{The generic meaning of the different status return values} 
\end{longtable}
\normalsize

%\begin{figure}[!ht]
%  \begin{center}
%    \includegraphics[height=0.6\textheight]{init_stack.eps}
%    \caption{Pathological GC Situations.\label{fig:GC}}
% \end{center}
%\end{figure}

\subsection{Timeout Values}

Several \oswald functions take a timeout parameter. This is a value that
specifies a timing window during which the operating system will try to
succeed in the requested operation. 

The type of the timeout value is \txt{x\_sleep} or \txt{x\_window},
both types are synonyms. The timeout type is
unsigned. There are however 2 very special timeout values,
\txt{x\_eternal} indicates that \oswald should try for an indefinite
amount of time to try the operation, while the \txt{x\_no\_wait} value
indicates that the operation system should just try the operation and return
immediately, whether it was successful or not.

Note that for timer handler functions and interrupt handlers,
\txt{x\_no\_wait} is the only valid timeout option.

For consistency, this timeout value is always passed as the last argument to
a function call, where applicable.

%
% $Id: queue.tex,v 1.1.1.1 2004/07/12 14:07:44 cvs Exp $
%

\subsection{Queues}

\subsubsection{Operation}

\subsubsection{Queue Structure Definition}

The structure definition of a queue is as follows:

\bcode
\begin{verbatim}
 1: typedef struct x_Queue * x_queue;
 2:
 3: typedef struct x_Queue {
 4:   x_Event Event;
 5:   w_word * messages;
 6:   volatile w_word * write;
 7:   volatile w_word * read;
 8:   volatile w_size available;
 9:   w_size capacity;
10:   w_word * limit;
11: } x_Queue;
\end{verbatim}
\ecode

The relevant fields in the queue structure are the following:

\begin{itemize}
\item \txt{x\_queue$\rightarrow$Event} This is the universal event structure that is a field
in all threadable components or elements. It controls the synchronized access
to the queue component and the signalling between threads to indicate changes
in the queue structure.
\item \txt{x\_queue$\rightarrow$messages} The memory area passed at
creation time that will contain the 32 bit messages of the queue. This field
is not changed during the lifetime of the queue.
\item \txt{x\_queue$\rightarrow$write} The current pointer in the message
area, where the next message to send can be written.
\item \txt{x\_queue$\rightarrow$read} The pointer to the message that is
to be read at the next call to \txt{x\_queue\_receive}.
\item \txt{x\_queue$\rightarrow$available} The number of messages that
are waiting in the queue to be read.
\item \txt{x\_queue$\rightarrow$limit} A convenience pointer that points
1 word beyond the memory area given by the \txt{messages} field. It is
used in the wrap around checks for \txt{read} and \txt{write}.
\end{itemize}

\subsubsection{Creating a Queue}

\txt{x\_status x\_queue\_create(x\_queue queue, void * space, w\_size capacity);}

A queue is created and initialized; \txt{queue} is a reference to a queue structure,
\txt{space} is a pointer to suitable sized buffer, i.e. \txt{space} must point to a storage space of
\txt{capacity} 32 bit words.

\subsubsection{Deleting a Queue}

The following call will delete a queue:

\txt{x\_status x\_queue\_delete(x\_queue queue);}

This call will try to delete a queue that is referred to by the \txt{queue} argument. 
Any possible remaining messages are discarded and any waiting threads are notified from
this fact by means of the \txt{xs\_deleted} status.

The memory of the queue structure itself is not released in any way. The memory that was passed
as the \txt{space} argument at queue creation time is not released. It is up to the application to release
these resources.

\subsubsection{Sending Data over a Queue}

Data is send to a queue by means of the following call:

\txt{x\_status x\_queue\_send(x\_queue queue, void * data, x\_sleep timeout);}

This call will try to send \txt{data} over a queue that is referred to by the
\txt{queue} argument. If there was space available in the queue, during the
\txt{timeout} window, the value of \txt{data}, \textbf{not the data
pointed to it,} will be put into the queue. If no space was available within the
\txt{timeout} window, the status \txt{xs\_no\_instance} is returned. 

Data is copied into the queue in a 'first in - first out' fashion.

If within the \txt{timeout} window, no space became available in the queue, the returned status will
be \txt{xs\_no\_instance}.

If a \txt{timeout} value is given from within an interrupt handler or timer handler that was not
\txt{x\_no\_wait}, the status \txt{xs\_bad\_context} is returned.

If this call resulted in the thread waiting for
space in the queue to come available, and the queue was deleted during the
\txt{timeout} value, the returned status will be \txt{xs\_deleted}.

The different return values that this call can produce are summarized in
table \ref{table:rs_queue_send}.


\begin{longtable}{||l|p{7cm}||}
\hline
\hfill \textbf{Return Value} \hfill\null & \textbf{Meaning} \\ 
\hline
\endhead
\hline
\endfoot
\endlastfoot
\hline




% \begin{table}[!ht]
%   \begin{center}
%     \begin{tabular}{||>{\footnotesize}l<{\normalsize}|>{\footnotesize}c<{\normalsize}||} \hline
%     \textbf{Return Value} & \textbf{Meaning} \\ \hline

\txt{xs\_success} & The call succeeded and the message has been delivered into the queue. \\

\txt{xs\_no\_instance} & There became no message slot available in the queue during the timeout window. \\

\txt{xs\_bad\_context} & A \txt{timeout} argument other than \txt{x\_no\_wait} was given from within a timer or interrupt handler context. \\

\txt{xs\_deleted} & The queue structure has been deleted by another thread during the call. \\

\txt{xs\_bad\_element} & The passed reference \txt{queue} doesn't refer to a valid queue structure. \\

%     \hline
%     \end{tabular}
%     \caption{Return Status for \txt{x\_queue\_send}}
%     \label{table:rs_queue_send}
%   \end{center}
% \end{table}

\hline 
\multicolumn{2}{c}{} \\
\caption{Return Status for \txt{x\_queue\_send}}
\label{table:rs_queue_send}
\end{longtable}
\normalsize

\subsubsection{Receiving Data from a Queue}

Data is received from a queue by means of the following call:

\txt{x\_status x\_queue\_receive(x\_queue queue, void * * data, x\_sleep timeout);}

This call will try to receive data from a queue that is referred to by the
\txt{queue} argument. If there was any data available in the queue, during the
\txt{timeout} window, this data will
be copied into the variable pointed to by \txt{data}. If no data was available within the
\txt{timeout} window, the status \txt{xs\_no\_instance} is returned. 

Data is returned from the queue in a 'first in - first out' fashion.

If a \txt{timeout} value is given from within an interrupt handler or timer handler that was not
\txt{x\_no\_wait}, the status \txt{xs\_bad\_context} is returned.

If this call resulted in the thread waiting for
data in the queue to come available, and the queue was deleted during the
\txt{timeout} value, the returned status will be \txt{xs\_deleted}.

The different return values that this call can produce are summarized in
table \ref{table:rs_queue_receive}.

\begin{longtable}{||l|p{7cm}||}
\hline
\hfill \textbf{Return Value} \hfill\null & \textbf{Meaning} \\ 
\hline 
\endhead
\hline
\endfoot
\endlastfoot
\hline


% \begin{table}[!ht]
%   \begin{center}
%     \begin{tabular}{||>{\footnotesize}l<{\normalsize}|>{\footnotesize}c<{\normalsize}||} \hline
%     \textbf{Return Value} & \textbf{Meaning} \\ \hline

\txt{xs\_success} & The call succeeded and data contains the message retrieved from the queue. \\

\txt{xs\_no\_instance} & There became no message available in the queue during the timeout window. \\

\txt{xs\_bad\_context} & A \txt{timeout} argument other than \txt{x\_no\_wait} was given from within a timer or interrupt handler context. \\

\txt{xs\_deleted} & The queue structure has been deleted by another thread during the call.  \\

\txt{xs\_bad\_element} & The passed reference \txt{queue} doesn't refer to a queue structure. \\


\hline 
\multicolumn{2}{c}{} \\
\caption{Return Status for \txt{x\_queue\_receive}}
\label{table:rs_queue_receive}
\end{longtable}
\normalsize




%     \hline
%     \end{tabular}
%     \caption{Return Status for \txt{x\_queue\_receive}}
%     \label{table:rs_queue_receive}
%   \end{center}
% \end{table}










%
% $Id: mutex.tex,v 1.1.1.1 2004/07/12 14:07:44 cvs Exp $
%

\subsection{Mutexes}

\subsubsection{Operation}

\subsubsection{Mutex Structure Definition}

The structure definition of a mutex is as follows:

\bcode
\begin{verbatim}
 1: typedef struct x_Mutex * x_mutex;
 2:
 3: typedef struct x_Mutex {
 4:   x_Event Event;
 5:   volatile x_thread owner;
 6: } x_Mutex;
\end{verbatim}
\ecode

The relevant fields in the mutex structure are the following:

\begin{itemize}
\item \txt{x\_mutex$\rightarrow$Event} This is the universal event structure that is a field
in all threadable components or elements. It controls the synchronized access
to the component and the signalling between threads.
\item \txt{x\_mutex$\rightarrow$owner} This field indicates which
thread has the mutex lock; when this field is \txt{NULL}, the
mutex is not owned by any thread.
\end{itemize}

\subsubsection{Creating a Mutex}

A mutex is created by means of the following call:

\txt{x\_status x\_mutex\_create(x\_mutex mutex);}

This call will initialize a mutex that is referred to by the \txt{mutex}
argument. Memory for the mutex must be allocated by the caller. Creating the
mutex does not result in locking the mutex. A mutex is always created in an
unlocked mode. A subsequent call to \txt{x\_mutex\_lock} is required for
locking a mutex.

\subsubsection{Deleting a Mutex}

A mutex is deleted by means of the following call:

\txt{x\_status x\_mutex\_delete(x\_mutex mutex);}

This call will delete a mutex that is referred to by the \txt{mutex} argument.
The mutex can be in any state, either locked or unlocked; when the mutex is
in the locked state, it is required that the thread that makes the call is
owner of the mutex, otherwise the \txt{xs\_not\_owner} status is
returned.

\subsubsection{Acquiring a Mutex}

A mutex is 'acquired' or locked by means of the following call:

\txt{x\_status x\_mutex\_lock(x\_mutex mutex, x\_sleep timeout);}

The kernel will try to lock the mutex referred to by the \txt{mutex} argument,
for the calling thread. This attempt will be done during the \txt{timeout}
window. If the call succeeds within the \txt{timeout} window, the
\txt{xs\_success} status is returned. 

If the mutex is deleted within the \txt{timeout} window, the returned status is
\txt{xs\_deleted}. If the call did not succeed within the given time, the
returned status is \txt{xs\_no\_instance}.

If the calling thread has a higher priority than the thread that currently
owns the mutex, the currently owning thread is temporarily boosted to the
calling thread priority, to defeat priority inversion. When the owning
thread releases the mutex, it will be reset to his assigned priority.

The waiting list on the mutex is priority ordered such that the highest
waiting priority thread will become the next owner.

\subsubsection{Releasing a Mutex}

A mutex is released by a calling thread by means of the following call:

\txt{x\_status x\_mutex\_unlock(x\_mutex mutex);}

When the calling thread does
not currently own the mutex referred to by the \txt{mutex} argument, the
\txt{xs\_not\_owner} status is returned, in any other case, the call must
succeed (note that a mutex can not be deleted by a thread that is not the
owner) and will return the \txt{xs\_success} status.

When the calling threads priority has been boosted to defeat the priority
inversion problem, the thread priority will be reset to the priority it
was assigned at creation time, when the calling thread doesn't own any other
mutexes. Should the calling thread own any other mutexes, its priority will
be set to the highest waiting thread priority on any mutexes it owns.



% ------------------------------------------------------------------------+
% Copyright (c) 2001 by Punch Telematix. All rights reserved.             |
%                                                                         |
% Redistribution and use in source and binary forms, with or without      |
% modification, are permitted provided that the following conditions      |
% are met:                                                                |
% 1. Redistributions of source code must retain the above copyright       |
%    notice, this list of conditions and the following disclaimer.        |
% 2. Redistributions in binary form must reproduce the above copyright    |
%    notice, this list of conditions and the following disclaimer in the  |
%    documentation and/or other materials provided with the distribution. |
% 3. Neither the name of Punch Telematix nor the names of other           |
%    contributors may be used to endorse or promote products derived      |
%    from this software without specific prior written permission.        |
%                                                                         |
% THIS SOFTWARE IS PROVIDED ``AS IS'' AND ANY EXPRESS OR IMPLIED          |
% WARRANTIES, INCLUDING, BUT NOT LIMITED TO, THE IMPLIED WARRANTIES OF    |
% MERCHANTABILITY AND FITNESS FOR A PARTICULAR PURPOSE ARE DISCLAIMED.    |
% IN NO EVENT SHALL PUNCH TELEMATIX OR OTHER CONTRIBUTORS BE LIABLE       |
% FOR ANY DIRECT, INDIRECT, INCIDENTAL, SPECIAL, EXEMPLARY, OR            |
% CONSEQUENTIAL DAMAGES (INCLUDING, BUT NOT LIMITED TO, PROCUREMENT OF    |
% SUBSTITUTE GOODS OR SERVICES; LOSS OF USE, DATA, OR PROFITS; OR         |
% BUSINESS INTERRUPTION) HOWEVER CAUSED AND ON ANY THEORY OF LIABILITY,   |
% WHETHER IN CONTRACT, STRICT LIABILITY, OR TORT (INCLUDING NEGLIGENCE    |
% OR OTHERWISE) ARISING IN ANY WAY OUT OF THE USE OF THIS SOFTWARE, EVEN  |
% IF ADVISED OF THE POSSIBILITY OF SUCH DAMAGE.                           |
% ------------------------------------------------------------------------+

%
% $Id: mailbox.tex,v 1.1.1.1 2004/07/12 14:07:44 cvs Exp $
%

\subsection{Mailboxes}

\subsubsection{Operation}

Mailboxes are elements that allow inter thread communication of a single 32
bit variable, e.g. a pointer. Mailboxes work the same as queues with a
message area size of 1 word.

So a mailbox can have two states:

\begin{enumerate}
\item The mailbox' slot be empty, in which case a new message can be posted
to the mailbox. A thread trying to get a message from the mailbox will be
blocked until a message comes available or until the timeout window expires.
A thread trying to send a message to the mailbox will not block, and the
mailbox will come into the next state.
\item The mailbox' slot can contain a 32 bit message, in which case a thread
that wants to send something to the mailbox will be blocked, until the slot
becomes available or until the timeout window expires. A thread trying to
read a message from the mailbox will not block, the available message will
be delivered immediately and the slot will become again empty so that the
mailbox will again be in the first state.
\end{enumerate}

\subsubsection{Mailbox Structure Definition}

The structure definition of a mailbox is as follows:

\bcode
\begin{verbatim}
 1: typedef struct x_Mailbox * x_mailbox;
 2:
 3: typedef struct x_Mailbox {
 4:   x_Event Event;
 5:   volatile void * message;
 6: } x_Mailbox;
\end{verbatim}
\ecode

The relevant fields in the mailbox structure are the following:

\begin{itemize}
\item \txt{x\_mailbox$\rightarrow$Event} This is the universal event structure that is a field
in all threadable components or elements. It controls the synchronized access
to the mailbox component and the signalling between threads to indicate changes
in the mailbox structure.
\item \txt{x\_mailbox$\rightarrow$message} The pointer to the 32 bit
message that is contained in the mailbox. When there is no message in the
mailbox, this field contains \txt{NULL}.
\end{itemize}


\subsubsection{Creating a Mailbox}

A mailbox is created by means of the following call:

\txt{x\_status x\_mailbox\_create(x\_mailbox mailbox);}

The mailbox referred to by the \txt{mailbox} argument will be initialized. The
memory for the mailbox must be allocated and provided by the caller.

\subsubsection{Deleting a Mailbox}

A mailbox can be deleted with the following call:

\txt{x\_status x\_mailbox\_delete(x\_mailbox mailbox);}

The different return values that this call can produce are summarized in
table \ref{table:rs_mailbox_delete}.

\footnotesize
\begin{longtable}{||l|p{9cm}||}
\hline
\hfill \textbf{Return Value} \hfill\null & \textbf{Meaning}  \hfill \\ 
\hline \endhead
\hline
\endfoot
\endlastfoot
\hline


% \begin{table}[!ht]
%   \begin{center}
%     \begin{tabular}{||>{\footnotesize}l<{\normalsize}|>{\footnotesize}c<{\normalsize}||} \hline
%     \textbf{Return Value} & \textbf{Meaning} \\ \hline

\txt{xs\_success} & The call succeeded and the mailbox was deleted. There were no threads that
were trying to do a mailbox operation. \\

\txt{xs\_waiting} & The mailbox structure has been deleted while there where threads waiting on
the mailbox for sending or receiving a message. All these threads have been notified of the delete and have acknowledged this fact. This means that when this call returns, no threads are actively waiting on the mailbox. \\

\txt{xs\_incomplete} & the mailbox structure has been deleted, but there were some threads waiting on it to send or receive a message and at least one of these threads did NOT acknowledge the deletion. This means that at least one thread is falsely assuming that the mailbox is still active. The caller of this function should better keep the memory of the mailbox around... \\

\txt{xs\_deleted} & The mailbox structure has been deleted by another thread during the call. In
other words, somebody beat us. This is the return value when another thread has deleted the mailbox when this tried was trying an operation on it. \\

\txt{xs\_bad\_element} & The passed reference \txt{mailbox} doesn't refer to a mailbox event. Probable cause is that the mailbox has been deleted or the passed event structure is not properly set up or is not functional anymore. \\

 \hline 
\multicolumn{2}{c}{} \\
\caption{Return Status for \txt{x\_mailbox\_delete}}
\label{table:rs_mailbox_delete}
\end{longtable}
\normalsize

%    \hline
%     \end{tabular}
%     \caption{Return Status for \txt{x\_mailbox\_delete}}
%     \label{table:rs_mailbox_delete}
%   \end{center}
% \end{table}

\subsubsection{Sending a Message to a Mailbox}

Data is delivered to a mailbox with the following call:

\txt{x\_status x\_mailbox\_send(x\_mailbox mailbox, void * message, x\_sleep timeout);}

This call will try to send data to the single slot in a mailbox that is referred to by the
\txt{mailbox} argument. If the slot became empty, during the \txt{timeout} window, the word pointed to
by \txt{message} will be put into the slot of the mailbox. If the slot did not become available within the
\txt{timeout} window, the status \txt{xs\_no\_instance} is returned. 

If within the \txt{timeout} window, the slot did not become available, the returned status will
be \txt{xs\_no\_instance}.

If a \txt{timeout} value is given from within an interrupt handler or timer handler that was not
\txt{x\_no\_wait}, the status \txt{xs\_bad\_context} is returned.

If this call resulted in the thread waiting for
the slot in the mailbox to come available, and the mailbox was deleted during the
\txt{timeout} value, the returned status will be \txt{xs\_deleted}.

The different return values that this call can produce are summarized in
table \ref{table:rs_mailbox_send}.

\footnotesize
\begin{longtable}{||l|p{9cm}||}
\hline
\hfill \textbf{Return Value} \hfill\null & \textbf{Meaning} \hfill \\ 
\hline
\endhead
\hline
\endfoot
\endlastfoot
\hline



% \begin{table}[!ht]
%   \begin{center}
%     \begin{tabular}{||>{\footnotesize}l<{\normalsize}|>{\footnotesize}c<{\normalsize}||} \hline
%     \textbf{Return Value} & \textbf{Meaning} \\ \hline

\txt{xs\_success} & The call succeeded and message was delivered into the mailbox slot. \\

\txt{xs\_no\_instance} & The message slot of the mailbox did not become empty during the timeout window. \\

\txt{xs\_bad\_context} & A \txt{timeout} argument other than \txt{x\_no\_wait} was given from within a timer or interrupt handler context. \\

\txt{xs\_deleted} & The mailbox structure has been deleted by another thread during the call. \\

\txt{xs\_bad\_element} & The passed reference \txt{mailbox} doesn't refer to a mailbox event or the mailbox structure became invalid, e.g. because it was deleted. \\

\hline 
\multicolumn{2}{c}{} \\
\caption{Return Status for \txt{x\_mailbox\_send}}
\label{table:rs_mailbox_send}
\end{longtable}
\normalsize




%     \hline
%     \end{tabular}
%     \caption{Return Status for \txt{x\_mailbox\_send}}
%     \label{table:rs_mailbox_send}
%   \end{center}
% \end{table}

\subsubsection{Receiving a Message from a Mailbox}

A message is received from a mailbox by means of the following call:

\txt{x\_status x\_mailbox\_receive(x\_mailbox mailbox, void * message, x\_sleep timeout);}

This call will try to receive a message from a mailbox that is referred to by the
\txt{mailbox} argument. If there was no message available in the mailbox, during the
\txt{timeout} window, this message will
be copied into the variable pointed to by \txt{message}. If no message was available within the
\txt{timeout} window, the status \txt{xs\_no\_instance} is returned. 

If a \txt{timeout} value is given from within an interrupt handler or timer handler that was not
\txt{x\_no\_wait}, the status \txt{xs\_bad\_context} is returned.

If this call resulted in the thread waiting for
a message to become available, and the mailbox was deleted during the
\txt{timeout} value, the returned status will be \txt{xs\_deleted}.

The different return values that this call can produce are summarized in
table \ref{table:rs_mailbox_receive}.


\footnotesize
\begin{longtable}{||l|p{9cm}||}
\hline
\hfill \textbf{Return Value} \hfill\null & \textbf{Meaning} \hfill \\ 
\endhead
\hline
\endfoot
\endlastfoot
\hline



% \begin{table}[!ht]
%   \begin{center}
%     \begin{tabular}{||>{\footnotesize}l<{\normalsize}|>{\footnotesize}c<{\normalsize}||} \hline
%     \textbf{Return Value} & \textbf{Meaning} \\ \hline

\txt{xs\_success} & The call succeeded and message points to the 32 bit message that was available in the \txt{mailbox} argument given. \\

\txt{xs\_no\_instance} & There became no message available in the \txt{mailbox} during the
\txt{timeout} window. \\

\txt{xs\_bad\_context} & A \txt{timeout} argument other than \txt{x\_no\_wait} was given from within a timer or interrupt handler context. \\

\txt{xs\_deleted} & The \txt{mailbox} structure has been deleted by another thread during the call. \\

\txt{xs\_bad\_element} & The passed reference \txt{mailbox} does not refer to a mailbox event. \\

\hline 
\multicolumn{2}{c}{} \\
\caption{Return Status for \txt{x\_mailbox\_receive}}
\label{table:rs_mailbox_receive}
\end{longtable}
\normalsize



%     \hline
%     \end{tabular}
%     \caption{Return Status for \txt{x\_mailbox\_receive}}
%     \label{table:rs_mailbox_receive}
%   \end{center}
% \end{table}













% ------------------------------------------------------------------------+
% Copyright (c) 2001 by Punch Telematix. All rights reserved.             |
%                                                                         |
% Redistribution and use in source and binary forms, with or without      |
% modification, are permitted provided that the following conditions      |
% are met:                                                                |
% 1. Redistributions of source code must retain the above copyright       |
%    notice, this list of conditions and the following disclaimer.        |
% 2. Redistributions in binary form must reproduce the above copyright    |
%    notice, this list of conditions and the following disclaimer in the  |
%    documentation and/or other materials provided with the distribution. |
% 3. Neither the name of Punch Telematix nor the names of other           |
%    contributors may be used to endorse or promote products derived      |
%    from this software without specific prior written permission.        |
%                                                                         |
% THIS SOFTWARE IS PROVIDED ``AS IS'' AND ANY EXPRESS OR IMPLIED          |
% WARRANTIES, INCLUDING, BUT NOT LIMITED TO, THE IMPLIED WARRANTIES OF    |
% MERCHANTABILITY AND FITNESS FOR A PARTICULAR PURPOSE ARE DISCLAIMED.    |
% IN NO EVENT SHALL PUNCH TELEMATIX OR OTHER CONTRIBUTORS BE LIABLE       |
% FOR ANY DIRECT, INDIRECT, INCIDENTAL, SPECIAL, EXEMPLARY, OR            |
% CONSEQUENTIAL DAMAGES (INCLUDING, BUT NOT LIMITED TO, PROCUREMENT OF    |
% SUBSTITUTE GOODS OR SERVICES; LOSS OF USE, DATA, OR PROFITS; OR         |
% BUSINESS INTERRUPTION) HOWEVER CAUSED AND ON ANY THEORY OF LIABILITY,   |
% WHETHER IN CONTRACT, STRICT LIABILITY, OR TORT (INCLUDING NEGLIGENCE    |
% OR OTHERWISE) ARISING IN ANY WAY OUT OF THE USE OF THIS SOFTWARE, EVEN  |
% IF ADVISED OF THE POSSIBILITY OF SUCH DAMAGE.                           |
% ------------------------------------------------------------------------+

%
% $Id: sem.tex,v 1.1.1.1 2004/07/12 14:07:44 cvs Exp $
%

\subsection{Semaphores}

\subsubsection{Operation}

\subsubsection{Semaphore Structure Definition}

The structure definition of a semaphore is as follows:

\bcode
\begin{verbatim}
 1: typedef struct x_Sem * x_sem;
 2:
 3: typedef struct x_Sem {
 4:   x_Event Event;
 5:   volatile w_size current;
 6: } x_Sem;
\end{verbatim}
\ecode

The relevant fields in the semaphore structure are the following:

\begin{itemize}
\item \txt{x\_sem$\rightarrow$Event} This is the universal event structure that is a field
in all threadable components or elements. It controls the synchronized access
to the semaphore component and the signalling between threads to indicate changes
in the semaphore structure.
\item \txt{x\_sem$\rightarrow$current} The current count of the
semaphore. Threads asking to get the semaphore, and find this field set to
0, will block as long as this value does not become greater than 0.
\end{itemize}


\subsubsection{Creating a Semaphore}

\txt{x\_status x\_sem\_create(x\_sem sem, w\_size initial);}

A semaphore is created and initialized; \txt{sem} is a reference to a semaphore
structure and will be initialized to a count indicated by \txt{initial}.

\subsubsection{Deleting a Semaphore}

The following call will deleted a semaphore:

\txt{x\_status x\_sem\_delete(x\_sem sem);}

This call will try to delete a semaphore that is referred to by the \txt{sem} argument. 
Any waiting threads are notified from this fact by means of the \txt{xs\_deleted} status.

The memory of the semaphore structure itself is not released in any way.
It is up to the application to release this resource.

\subsubsection{Incrementing a Semaphore}

The count of a semaphore is incremented by means of the following call:

\txt{x\_status x\_sem\_put(x\_sem sem);}

This call will increment the count of the semaphore that is referred to by the
\txt{sem} argument. If the call succeeds, \txt{xs\_success} is returned. Note
that a semaphore can start out at 0 and that subsequent push commands will
increase the semaphore. I.e., there is no mechanism to check that a
semaphore is pushed beyond a certain preset number. Control for this is up
to the programmer using the semaphore.

When any threads are waiting on the semaphore count to become higher than
zero, the thread with the highest priority will be woken up to acquire the
semaphore.

The different return values that this call can produce are summarized in
table \ref{table:rs_sem_put}.

\footnotesize
\begin{longtable}{||l|p{9cm}||}
\hline
\hfill \textbf{Return Value} \hfill\null & \textbf{Meaning} \hfill \\ 
\endhead
\hline
\endfoot
\endlastfoot
\hline

% \begin{table}[!ht]
%   \begin{center}
%     \begin{tabular}{||>{\footnotesize}l<{\normalsize}|>{\footnotesize}c<{\normalsize}||} \hline
%     \textbf{Return Value} & \textbf{Meaning} \\ \hline

\txt{xs\_success} & The call succeeded and the semaphore count has been incremented. \\

\txt{xs\_deleted} & The semaphore structure has been deleted by another thread during the call. \\

\txt{xs\_bad\_element} & The passed reference \txt{sem} does not refer to a semaphore structure. \\



\hline 
\multicolumn{2}{c}{} \\
\caption{Return Status for \txt{x\_sem\_put}}
\label{table:rs_sem_put}
\end{longtable}
\normalsize

%     \hline
%     \end{tabular}
%     \caption{Return Status for \txt{x\_sem\_put}}
%     \label{table:rs_sem_put}
%   \end{center}
% \end{table}

\subsubsection{Decrementing a Semaphore}

The count of a semaphore is decremented, or its resource is acquired, by means of the following call:

\txt{x\_status x\_sem\_get(x\_sem sem, x\_sleep timeout);}

This call will try to decrement the count of the semaphore that is referred to by the
\txt{sem} argument. If the current count is zero, the thread will wait for the
specified amount of \txt{timeout} ticks. If the count became greater than zero, during the
\txt{timeout} window, this call will return \txt{xs\_success}. If the count didn't become
greater than zero within the \txt{timeout} window, the status \txt{xs\_no\_instance} is returned. 

If a \txt{timeout} value is given from within an interrupt handler or timer handler that was not
\txt{x\_no\_wait}, the status \txt{xs\_bad\_context} is returned.

If this call resulted in the thread waiting for the count to become more
than zero, and the semaphore was deleted during the \txt{timeout} value, the
returned status will be \txt{xs\_deleted}.

The different return values that this call can produce are summarized in
table \ref{table:rs_sem_get}.

\begin{table}[!htbp]
  \begin{center}
    \begin{tabular}{||>{\footnotesize}l<{\normalsize}|>{\footnotesize}c<{\normalsize}||} \hline
    \textbf{Return Value} & \textbf{Meaning} \\ \hline

\txt{xs\_success} &

\begin{minipage}[t]{9cm}
The call succeeded and the semaphore count has been decremented.
\end{minipage} \\

\txt{xs\_no\_instance} &

\begin{minipage}[t]{9cm}
The call did not succeed within the given timeout window.
\end{minipage} \\

\txt{xs\_bad\_context} &

\begin{minipage}[t]{9cm}
A timeout option other than \txt{x\_no\_wait} has been given from within a timer
or interrupt handler context.
\end{minipage} \\

\txt{xs\_deleted} &

\begin{minipage}[t]{9cm}
The semaphore structure has been deleted by another thread during the call.
\end{minipage} \\

\txt{xs\_bad\_element} &

\begin{minipage}[t]{9cm}
The passed reference \txt{sem} doesn't refer to a semaphore structure.
\end{minipage} \\

    \hline
    \end{tabular}
    \caption{Return Status for \txt{x\_sem\_get}}
    \label{table:rs_sem_get}
  \end{center}
\end{table}
\normalsize


% ------------------------------------------------------------------------+
% Copyright (c) 2001 by Punch Telematix. All rights reserved.             |
%                                                                         |
% Redistribution and use in source and binary forms, with or without      |
% modification, are permitted provided that the following conditions      |
% are met:                                                                |
% 1. Redistributions of source code must retain the above copyright       |
%    notice, this list of conditions and the following disclaimer.        |
% 2. Redistributions in binary form must reproduce the above copyright    |
%    notice, this list of conditions and the following disclaimer in the  |
%    documentation and/or other materials provided with the distribution. |
% 3. Neither the name of Punch Telematix nor the names of other           |
%    contributors may be used to endorse or promote products derived      |
%    from this software without specific prior written permission.        |
%                                                                         |
% THIS SOFTWARE IS PROVIDED ``AS IS'' AND ANY EXPRESS OR IMPLIED          |
% WARRANTIES, INCLUDING, BUT NOT LIMITED TO, THE IMPLIED WARRANTIES OF    |
% MERCHANTABILITY AND FITNESS FOR A PARTICULAR PURPOSE ARE DISCLAIMED.    |
% IN NO EVENT SHALL PUNCH TELEMATIX OR OTHER CONTRIBUTORS BE LIABLE       |
% FOR ANY DIRECT, INDIRECT, INCIDENTAL, SPECIAL, EXEMPLARY, OR            |
% CONSEQUENTIAL DAMAGES (INCLUDING, BUT NOT LIMITED TO, PROCUREMENT OF    |
% SUBSTITUTE GOODS OR SERVICES; LOSS OF USE, DATA, OR PROFITS; OR         |
% BUSINESS INTERRUPTION) HOWEVER CAUSED AND ON ANY THEORY OF LIABILITY,   |
% WHETHER IN CONTRACT, STRICT LIABILITY, OR TORT (INCLUDING NEGLIGENCE    |
% OR OTHERWISE) ARISING IN ANY WAY OUT OF THE USE OF THIS SOFTWARE, EVEN  |
% IF ADVISED OF THE POSSIBILITY OF SUCH DAMAGE.                           |
% ------------------------------------------------------------------------+

%
% $Id: signals.tex,v 1.1.1.1 2004/07/12 14:07:44 cvs Exp $
%

\subsection{Signals}

\subsubsection{Operation}

Signals are a set of 31 flags, that can be used by threads to communicate certain
conditions to other threads. The most significant bit of the signals
structure is reserved for future purposes (e.g. to control atomic operations
on signals.).

By means of logical operators, threads can set or query individual bit flags
or sets of bit flags in the signal structure.

Threads can also wait until a certain bit flag configuration appears in the
signals structure. The use and combination of the logical operators at query
or setting time can provide powerful but complex (read \textit{dangerous}) semantics.

\subsubsection{Signals Structure Definition}

The structure definition of a set of signals is as follows:

\bcode
\begin{verbatim}
 1: typedef struct x_Signals * x_signals;
 2:
 3: typedef struct x_Signals {
 4:   x_Event Event;
 5:   volatile x_flags flags;
 6: } x_Signals;
\end{verbatim}
\ecode

The relevant fields in the signals structure are the following:

\begin{itemize}
\item \txt{x\_signals$\rightarrow$Event} This is the universal event structure that is a field
in all threadable components or elements. It controls the synchronized access
to the signals component and the signalling between threads to indicate changes
in the signals structure.
\item \txt{x\_signals$\rightarrow$flags} The word in which the lower 31
bits are used to store the bit states of the different signals.
\end{itemize}


\subsubsection{Signal Set or Get Options}

The signal get and set functions take a logical option indicator that
controls the behavior of these functions. The different options and
in which context they are applicable are indicated in table
\ref{table:xo_options}.


\footnotesize
\begin{longtable}{||l|p{2cm}|p{9cm}||}
\hline
\hfill \textbf{Option} \hfill\null & \textbf{Applies to} & \textbf{Meaning} \\ 
\hline
\endhead
\hline
\endfoot
\endlastfoot
\hline

% \begin{table}[!ht]
%   \begin{center}
% \begin{tabular}{||>{\footnotesize}l<{\normalsize}|>{\footnotesize}l<{\normalsize}|>{\footnotesize}l<{\normalsize}||} \hline
% \textbf{Option} & \textbf{Applies to} & \textbf{Meaning} \\ \hline

\txt{xo\_or} &

\begin{minipage}[t]{2cm}
\txt{x\_signals\_get} \\
\txt{x\_signals\_set} \\
\end{minipage} &

\begin{minipage}[t]{8cm}
For the \txt{x\_signals\_get} function, this option indicates that the
query is satisfied when \textbf{any} of the requested flags becomes set during the
timeout window. When this call returns, the flag or flags that satisfied the
condition, \textbf{remain set} in the signals structure.\\

For the \txt{x\_signals\_set} function, this option indicates that the
flags that are passed as an argument are logically ORed with the actual flags
in the signal structure; the result of this logical or become the new flags
of the signals structure.
\end{minipage} \\

 & & \\

\txt{xo\_or\_clear} &

\begin{minipage}[t]{2cm}
\txt{x\_signals\_get} \\
\end{minipage} &

\begin{minipage}[t]{8cm}
This option indicates that the
query is satisfied when \textbf{any} of the requested flags becomes set during the
timeout window. Before the call returns, the flag or flags that satisfied the
condition, will be reset (to 0) before returning. Since upon such a return,
the signal flags have changed again, all waiting threads will be woken up to
recheck if their conditions are satisfied.
\end{minipage} \\

 & & \\

\txt{xo\_and} &

\begin{minipage}[t]{2cm}
\txt{x\_signals\_get} \\
\txt{x\_signals\_set} \\
\end{minipage} &

\begin{minipage}[t]{8cm}
For the \txt{x\_signals\_get} function, this option indicates that the
query is only satisfied when \textbf{all} of the requested flags become set during the
timeout window. When this call returns, the flag or flags that satisfied the
condition, remain set in the signals structure.\\

For the \txt{x\_signals\_set} function, this option indicates that the
flags that are passed as an argument or logically anded with the actual flags
in the signal structure; the result of this logical and become the new flags
of the signals structure.
\end{minipage} \\

 & & \\

\txt{xo\_and\_clear} &

\begin{minipage}[t]{2cm}
\txt{x\_signals\_get} \\
\end{minipage} &

\begin{minipage}[t]{8cm}
This option indicates that the
query is satisfied when \textbf{all} of the requested flags becomes set during the
timeout window. Before the call returns, the flags that satisfied the
condition, will be reset (to 0) before returning. Since upon such a return,
the signal flags have changed again, all waiting threads will be woken up to
recheck if their conditions are satisfied.
\end{minipage} \\



\hline 
\multicolumn{3}{c}{} \\
\caption{The different signal set and get options}
\label{table:xo_options}
\end{longtable}
\normalsize


%     \hline
%     \end{tabular}
%     \caption{The different signal set and get options}
%     \label{table:xo_options}
%   \end{center}
% \end{table}

\subsubsection{Creating a Set of Signals}

A set of 31 signalling flags is created by means of the call:\\

\txt{x\_status x\_signals\_create(x\_signals signals);}

This call results in signal event being set up. All 31 signals bits are initialized to
0. This function always returns \txt{xs\_success}. 

\subsubsection{Deleting a Set of Signals}

\subsection{Querying for a Condition}

\txt{x\_status x\_signals\_get(x\_signals si, x\_flags co, x\_option op, x\_flags ac, x\_sleep to);}

The different return values that this call can produce are summarized in
table \ref{table:rs_signals_get}.


\footnotesize
\begin{longtable}{||l|p{9cm}||}
\hline
\hfill \textbf{Return Value} \hfill\null & \textbf{Meaning}  \\ 
\hline
\endhead
\hline
\endfoot
\endlastfoot
\hline

% \begin{table}[!ht]
%   \begin{center}
%     \begin{tabular}{||>{\footnotesize}l<{\normalsize}|>{\footnotesize}c<{\normalsize}||} \hline
%     \textbf{Return Value} & \textbf{Meaning} \\ \hline

\txt{xs\_success} &

\begin{minipage}[t]{9cm}
The call succeeded and the query flags are satisfied according to the
\txt{op} argument given.
\end{minipage} \\

\txt{xs\_no\_instance} &

\begin{minipage}[t]{9cm}
The call did not succeed within the given timeout and with the given option.
\end{minipage} \\

\txt{xs\_bad\_option} &

\begin{minipage}[t]{9cm}
The call failed since the passed \txt{op} option argument isn't one of
\txt{xo\_or}, \txt{xo\_or\_clear}, \txt{xo\_and} or
\txt{xo\_and\_clear}.
\end{minipage} \\

\txt{xs\_bad\_context} &

\begin{minipage}[t]{9cm}
A timeout argument 'to' other than \txt{x\_no\_wait} was given from within a timer
or interrupt handler context.
\end{minipage} \\

\txt{xs\_deleted} &

\begin{minipage}[t]{9cm}
The signals structure has been deleted by another thread during the call.
\end{minipage} \\

\txt{xs\_bad\_element} &

\begin{minipage}[t]{9cm}
The passed reference \txt{si} doesn't refer to a signal event.
\end{minipage} \\



\hline 
\multicolumn{2}{c}{} \\
\caption{The different signal set and get options}
\label{table:xo_options}
\end{longtable}
\normalsize

%     \hline
%     \end{tabular}
%     \caption{Return Status for \txt{x\_signals\_get}}
%     \label{table:rs_signals_get}
%   \end{center}
% \end{table}

\subsection{Setting a Condition}

Signal bits in the signals structure can be set by means of the following
call:

\txt{x\_status x\_signals\_set(x\_signals si, x\_flags co, x\_option op);}

The different set options and in which context they are applicable are indicated in table
\ref{table:xo_options}.
















%$\Rightarrow$
%
% $Id: block.tex,v 1.1.1.1 2004/07/12 14:07:44 cvs Exp $
%

\subsection{Block Pools}

\subsubsection{Operation}

Block pools provide a convenient and fast way to allocate memory of fixed
size blocks in \oswald. A block allocation, when there still are blocks
available in the block pool, is done in constant time. Therefore block pools
should be used primarily in interrupt handlers and in these contexts where
it is known beforehand that only a limited or fixed amount of memory will be needed. 

For situations where this is not the case, the usual \txt{x\_alloc\_mem}
and \txt{x\_free\_mem} routines that allocate and free different sized
memory blocks from the heap, should be used.

Note that the sizes of the blocks, are always rounded to a word boundary,
i.e. the size will be rounded up to a number of bytes that is divisible by 4.
Some processors are not able or not optimized to handle pointers that do not
begin on a word boundary.

Also note that when a block is allocated, it contains a hidden pointer to
the block pool it belongs to. Therefore, when e.g. a block size of 16 bytes
or 4 words is required, a block will be of size 16 + 4 = 20 bytes. This
should be taken into consideration when allocating memory for a block pool.

There is a convenience function that should preferably be used when
calculating the size of memory for the blocks in a pool:

\txt{w\_size x\_block\_calc(w\_size block\_size, w\_size num\_blocks);}

This routine will return the number of \textbf{bytes} required to allocate
for a block pool with \txt{num\_blocks} blocks that require a usable (not
including the hidden pointer) space of \txt{block\_size} bytes.

\subsubsection{Block Structure Definition}

The structure definition of a block is as follows:

\bcode
\begin{verbatim}
 1: typedef struct x_Boll * x_boll;
 2: typedef struct x_Block * x_block;
 3:
 4: typedef struct x_Boll {
 5:   union {
 6:     x_block block;
 7:     x_boll next;
 8:   } header;
 9:   w_ubyte bytes[1];
10: } x_Boll;
11:
12: typedef struct x_Block {
13:   x_Event Event;
14:   w_size boll_size;
15:   w_size bolls_max;
16:   w_size space_size;
17:   volatile w_size bolls_left;
18:   volatile x_boll bolls;
19: } x_Block;
\end{verbatim}
\ecode

The \txt{x\_Boll} structure is used internally by the block pool code to
organize a linked list of free blocks. It accomodates the pointer back to
the block pool, when a block is in use (allocated) by means of the
\txt{x\_boll$\rightarrow$header.block} field and when a block is on the
free list of blocks in a block pool, the field
\txt{x\_boll$\rightarrow$header.next} points to the next free block in
line, or \txt{NULL} when no more free blocks are in the list. The
\txt{x\_boll$\rightarrow$bytes} represents the usable memory space in
each block. It is not really an array of 1 bytes, but extends beyond the end
of the structure to represent the useable space.

The relevant fields in the block structure are the following:

\begin{itemize}
\item \txt{x\_block$\rightarrow$Event} This is the universal event structure that is a field
in all threadable components or elements. It controls the synchronized access
to the block component and the signalling between threads to indicate changes
in the block structure.
\item \txt{x\_block$\rightarrow$boll\_size} The size of a single block in
bytes; this value is rounded up to a value that is divisible by 4 (word
size) so it maybe not the same as the value passed at the creation time of a
block pool.
\item \txt{x\_block$\rightarrow$bolls\_max} The number of blocks that the
block pool had after it has been created and before any blocks are allocated
from it.
\item \txt{x\_block$\rightarrow$space\_size} The number of bytes that are
available in the memory area that is passed at creation time and from which
the blocks are carved.
\item \txt{x\_block$\rightarrow$bolls\_left} The number of blocks that
are still available in the pool for allocating.
\item \txt{x\_block$\rightarrow$bolls} The linked list of free blocks in
the block pool.
\end{itemize}

\subsubsection{Creating a Block Pool}

A block pool is created by means of the following call:

\txt{x\_status x\_block\_create(x\_block block, w\_size bs, void * space, w\_size ss);}

Create a block pool, with each block having a size of \txt{bs} bytes and
where the memory for these blocks is indicated by \txt{space}. This
memory area has space for \txt{ss} bytes.

This call will generate the number blocks that can be carved out the memory
space and will assign them to the block pool.

The different return values that this call can produce are summarized
in table \ref{table:block_create}.  

\footnotesize
\begin{longtable}{||l|p{9cm}||}
\hline
\hfill \textbf{Return Value} \hfill\null & \textbf{Meaning}  \\ 
\hline
\endhead
\hline
\endfoot
\endlastfoot
\hline


% \begin{table}[!ht]
%   \begin{center}
%     \begin{tabular}{||>{\footnotesize}l<{\normalsize}|>{\footnotesize}c<{\normalsize}||} \hline
%     \textbf{Return Value} & \textbf{Meaning} \\ \hline

\txt{xs\_success} &
\begin{minipage}[t]{9cm}
The call succeeded and the block pool has been set up properly. Note that
the number of blocks that the pool has (allocated or released) is indicated
by the structure field \txt{block$\rightarrow$bolls\_max}.
\end{minipage} \\

\txt{xs\_no\_instance} &

\begin{minipage}[t]{9cm}
The size of the memory area passed and the block size given to create are
impossible. I.e., not a single block can be carved out of this memory area.
Note that there is a hidden cost of 4 bytes per block and that the size of a
block is rounded up to 4, first. 
\end{minipage} \\

\hline 
\multicolumn{2}{c}{} \\
\caption{Return Status for \txt{x\_block\_create}}
\label{table:block_create}
\end{longtable}
\normalsize

%     \hline
%     \end{tabular}
%     \caption{Return Status for \txt{x\_block\_create}}
%     \label{table:block_create}
%   \end{center}
% \end{table}

\subsubsection{Deleting a Block Pool}

A block pool can be deleted by means of the following call:

\txt{x\_status x\_block\_delete(x\_block block);}

Please note that deleting a block while there are still blocks in use by
other threads or this thread, leads to undefined behavior. Before
attempting the call, it is wise to check the fields
\txt{block$\rightarrow$bolls\_max} and
\txt{block$\rightarrow$bolls\_left}, The former indicates the number of
blocks that the pool has and the latter indicates the number of blocks not
allocated. When both are equal, there are no blocks in use. Beware that
while comparing both, another thread could have allocated a block. So make
sure that the thread doing the comparison, is running a the highest priority
and the pool is not being used by an interrupt handler. Deleting a pool of
an interrupt handler is a bad idea anyway.

The different return values that this call can produce are summarized
in table \ref{table:block_delete}.  

\footnotesize
\begin{longtable}{||l|p{9cm}||}
\hline
\hfill \textbf{Return Value} \hfill\null & \textbf{Meaning} \\ 
\hline
\endhead
\hline
\endfoot
\endlastfoot
\hline


% \begin{table}[!ht]
%   \begin{center}
%     \begin{tabular}{||>{\footnotesize}l<{\normalsize}|>{\footnotesize}c<{\normalsize}||} \hline
%     \textbf{Return Value} & \textbf{Meaning} \\ \hline

\txt{xs\_waiting} &
\begin{minipage}[t]{9cm}
Some other threads were attempting an allocate operation on the block pool.
\end{minipage} \\

\txt{xs\_incomplete} &

\begin{minipage}[t]{9cm}
Some threads were attempting an allocate operation on the block pool but
haven't acknowledged yet that they are aborting this operation. Proceed
with caution in further deleting the block pool, like e.g. releasing the
memory of the pool.
\end{minipage} \\

\txt{xs\_deleted} &

\begin{minipage}[t]{9cm}
Some other thread has been deleting this element already.
\end{minipage} \\

\txt{xs\_bad\_element} &

\begin{minipage}[t]{9cm}
The passed \txt{block} structure is not pointing to a valid block pool
structure.
\end{minipage} \\

\hline 
\multicolumn{2}{c}{} \\
\caption{Return Status for \txt{x\_block\_delete}}
\label{table:block_delete}
\end{longtable}
\normalsize

%     \hline
%     \end{tabular}
%     \caption{Return Status for \txt{x\_block\_delete}}
%     \label{table:block_delete}
%   \end{center}
% \end{table}

\subsubsection{Allocating a Block from the Pool}

A block can be allocated from the pool with the following call:

\txt{x\_status x\_block\_allocate(x\_block block, void ** bytes, x\_sleep to);}

The different return values that this call can produce are summarized
in table \ref{table:block_allocate}.  


\footnotesize
\begin{longtable}{||l|p{9cm}||}
\hline
\hfill \textbf{Return Value} \hfill\null & \textbf{Meaning} \\ 
\hline
\endhead
\hline
\endfoot
\endlastfoot
\hline


% \begin{table}[!ht]
%   \begin{center}
%     \begin{tabular}{||>{\footnotesize}l<{\normalsize}|>{\footnotesize}c<{\normalsize}||} \hline
%     \textbf{Return Value} & \textbf{Meaning} \\ \hline

\txt{xs\_success} &
\begin{minipage}[t]{9cm}
The call succeeded and the pointer indicated by \txt{*bytes} is pointing to the
allocated block.
\end{minipage} \\

\txt{xs\_no\_instance} &

\begin{minipage}[t]{9cm}
There was no block available for the calling thread within the timeout
specified. The pointer indicated by \txt{*bytes} is not changed during
the call.
\end{minipage} \\

\txt{xs\_deleted} &

\begin{minipage}[t]{9cm}
Another thread has deleted the block pool while this current thread was
trying to allocate a block.
\end{minipage} \\

\txt{xs\_bad\_element} &

\begin{minipage}[t]{9cm}
The passed \txt{block} structure is not pointing to a valid block pool
structure.
\end{minipage} \\

\hline 
\multicolumn{2}{c}{} \\
\caption{Return Status for \txt{x\_block\_allocate}}
\label{table:block_allocate}
\end{longtable}
\normalsize


%     \hline
%     \end{tabular}
%     \caption{Return Status for \txt{x\_block\_allocate}}
%     \label{table:block_allocate}
%   \end{center}
% \end{table}

\subsubsection{Releasing a Block from the Pool}

\txt{x\_status x\_block\_release(x\_block block, void * bytes);}

The different return values that this call can produce are summarized
in table \ref{table:block_release}.  

\footnotesize
\begin{longtable}{||l|p{9cm}||}
\hline
\hfill \textbf{Option} \hfill\null & \textbf{Meaning} \\ 
\hline
\endhead
\hline
\endfoot
\endlastfoot
\hline


% \begin{table}[!ht]
%   \begin{center}
%     \begin{tabular}{||>{\footnotesize}l<{\normalsize}|>{\footnotesize}c<{\normalsize}||} \hline
%     \textbf{Return Value} & \textbf{Meaning} \\ \hline

\txt{xs\_success} &
\begin{minipage}[t]{9cm}
The call succeeded and the block has been given back to the pool.
\end{minipage} \\

\txt{xs\_deleted} &

\begin{minipage}[t]{9cm}
Another thread has deleted the block pool while this current thread was
trying to release a block.
\end{minipage} \\

\txt{xs\_bad\_element} &

\begin{minipage}[t]{9cm}
The passed \txt{block} structure is not pointing to a valid block pool
structure.
\end{minipage} \\

\hline 
\multicolumn{2}{c}{} \\
\caption{Return Status for \txt{x\_block\_release}}
\label{table:block_release}
\end{longtable}
\normalsize

%     \hline
%     \end{tabular}
%     \caption{Return Status for \txt{x\_block\_release}}
%     \label{table:block_release}
%   \end{center}
% \end{table}






% ------------------------------------------------------------------------+
% Copyright (c) 2001 by Punch Telematix. All rights reserved.             |
%                                                                         |
% Redistribution and use in source and binary forms, with or without      |
% modification, are permitted provided that the following conditions      |
% are met:                                                                |
% 1. Redistributions of source code must retain the above copyright       |
%    notice, this list of conditions and the following disclaimer.        |
% 2. Redistributions in binary form must reproduce the above copyright    |
%    notice, this list of conditions and the following disclaimer in the  |
%    documentation and/or other materials provided with the distribution. |
% 3. Neither the name of Punch Telematix nor the names of other           |
%    contributors may be used to endorse or promote products derived      |
%    from this software without specific prior written permission.        |
%                                                                         |
% THIS SOFTWARE IS PROVIDED ``AS IS'' AND ANY EXPRESS OR IMPLIED          |
% WARRANTIES, INCLUDING, BUT NOT LIMITED TO, THE IMPLIED WARRANTIES OF    |
% MERCHANTABILITY AND FITNESS FOR A PARTICULAR PURPOSE ARE DISCLAIMED.    |
% IN NO EVENT SHALL PUNCH TELEMATIX OR OTHER CONTRIBUTORS BE LIABLE       |
% FOR ANY DIRECT, INDIRECT, INCIDENTAL, SPECIAL, EXEMPLARY, OR            |
% CONSEQUENTIAL DAMAGES (INCLUDING, BUT NOT LIMITED TO, PROCUREMENT OF    |
% SUBSTITUTE GOODS OR SERVICES; LOSS OF USE, DATA, OR PROFITS; OR         |
% BUSINESS INTERRUPTION) HOWEVER CAUSED AND ON ANY THEORY OF LIABILITY,   |
% WHETHER IN CONTRACT, STRICT LIABILITY, OR TORT (INCLUDING NEGLIGENCE    |
% OR OTHERWISE) ARISING IN ANY WAY OUT OF THE USE OF THIS SOFTWARE, EVEN  |
% IF ADVISED OF THE POSSIBILITY OF SUCH DAMAGE.                           |
% ------------------------------------------------------------------------+

%
% $Id: monitor.tex,v 1.1.1.1 2004/07/12 14:07:44 cvs Exp $
%

\subsection{Monitors}

\subsection{Operation}

Monitors are a means for threads to control shared variables and communicate
amongst threads about the status of shared variables. Usually, monitors are
used with the following template or pattern.

\bcode
\begin{verbatim}
   1 : <a 'monitor' is created by a single thread only>
   2 : ...
   3 : status = x_monitor_enter(monitor, x_eternal);
   4 : if (status == xs_success) {
   5 :   ...
   6 :   while ( <a certain condition X is not true> ) {
   7 :     x_monitor_wait(monitor, x_eternal);
   8 :   }
   9 :   ...
  10 :   <code to manipulate the condition X>
  11 :   status = x_monitor_notify_all(monitor);
  12 :   if (status == xs_success) {
  13 :     status = x_monitor_exit(monitor);
  14 :   }
  15 : }
  16 :
\end{verbatim}
\ecode

Note that the return values of the \oswald calls are not checked for errors.
Of course good programming requires this checking, which is only ommitted in
the above example to keep the code small. (If you know a better excuse, let
me know)

When several threads want to use a monitor to synchronize amongst
themselves, only one single monitor is in use in most simple cases.
Sometimes, several monitors are used for other synchronization, more complex patterns.
This pattern is the basic pattern for use of monitors.

The area where threads manipulate shared data in a safe way, starts from
line 3 and extends to line 13, in the above example. After line 13, the area
of common manipulation stops.

It is important to notice in this example that several threads are executing
this sample of code, virtually at the same time.

There is only a single line, in the above example, where the thread that is
executing this code, has not locked the whole range of lines 3 to 13; this
is place where the currently executing thread 'waits' for a certain
condition to become true, i.e. line 7. Waiting on a monitor means that the
currently executing thread temporarily unlocks the monitor, so that other
threads can relock the monitor and manipulate the variable X such that the 
while loop on line 6, will be exited. After exiting the while loop, the
monitor will again be owned by the thread, exactly as it was on line 5.

When a thread is waiting, other threads will be able, one by one, after
getting or locking the monitor, to manipulate the variable or condition X,
referred to on lines 6 and 10. As soon as a thread has changed this variable
X, so that the while loop on line 6, will be exited, the thread that has
been waiting, needs to be 'notified' of this change so that it can
reevaluate the while condition on line 6. 

This is done by means of the \txt{x\_monitor\_notify} statement at line
11. The notification does not unlock the monitor, it merely signals a single
other thread that it should reevaluate the while loop, as soon as the
current thread unlocks the monitor, at line 13.

When multiple condition variables are being used for synchronization in more
complex pieces of code, the \txt{x\_monitor\_notify\_all} must be used.
This call will wake up all threads waiting on the monitor to reevaluate the
while loop they are waiting in, to see if the condition has changed in their
favor so that they can exit the while loop.

It is \textbf{very important} to note that at line 6, always a
\textbf{while} clause should be used and not an \textbf{if} clause. The
thread waiting does not know, how many threads are going to be notified (all
or a single) and in the case of multiple synchronization variables X, he
does not know whether his own condition will become false, so that he can
exit the loop. So each time the thread is woken up from the wait, he needs
to reevaluate the condition, before deciding to proceed with the rest of the
code, i.e. it requires a \textbf{while} in stead of an \textbf{if} clause.

Another important aspect of monitors is that a thread can lock a monitor
several times. I.e., unlike mutexes, there is no \txt{x\_deadlock} status
when a monitor is entered several times. In other words, when a monitor is
entered Z times, it needs to be exited Z times by the same thread before it
really is released by the thread. Monitor locks (entering) and unlocks
(exiting) are thus nestable. The lock count is kept in the monitor
structure.

\subsection{Monitor Structure Definition}

The structure definition of a monitor is as follows:

\bcode
\begin{verbatim}
 1: typedef struct x_Monitor * x_monitor;
 2:
 3: typedef struct x_Monitor {
 4:   x_Event Event;
 5:   volatile x_thread owner;
 6:   volatile w_ushort count;
 7:   volatile w_ushort n_waiting;
 8:   volatile x_thread l_waiting;
 9: } x_Monitor;
\end{verbatim}
\ecode

The relevant fields in the monitor structure are the following:

\begin{itemize}
\item \txt{x\_monitor$\rightarrow$Event} This is the universal event structure that is a field
in all threadable components or elements. It controls the synchronized access
to the monitor component and the signalling between threads to indicate changes
in the monitor structure.
\item \txt{x\_monitor$\rightarrow$owner} The current owner thread of the monitor
or \txt{NULL} when there is no current owner of the monitor.
\item \txt{x\_monitor$\rightarrow$count} The number of times that the
current owner thread has locked the monitor. Monitor entries can be nested
and must be exited as many times as a thread has entered the monitor.
\item \txt{x\_monitor$\rightarrow$n\_waiting} The number of threads in
the list, indicated by the previous field, that are waiting on this monitor.
\item \txt{x\_monitor$\rightarrow$l\_waiting} The linked list of threads
that are waiting on the monitor. See below for an explanation of 'waiting on
a monitor'. This field is continued via the
\txt{x\_thread$\rightarrow$l\_waiting} field of a thread.
\end{itemize}

\subsubsection{Creating a Monitor}

A monitor is created by means of the following call:

\txt{x\_status x\_monitor\_create(x\_monitor monitor);}

This will initialize the monitor so that it can be used in the subsequent
calls as described further. Note that the creation of a monitor by a certain
thread, does not make the thread owner of the monitor. To acquire a monitor,
the thread should perform the \txt{x\_monitor\_enter} call as described
below.

\subsubsection{Deleting a Monitor}

A monitor can be deleted by means of the following call:

\txt{x\_status x\_monitor\_delete(x\_monitor monitor);}

The different return values that this call can produce are summarized
in table \ref{table:monitor_delete}.  


\footnotesize
\begin{longtable}{||l|p{9cm}||}
\hline
\hfill \textbf{Return Value} \hfill\null & \textbf{Meaning}  \\ 
\hline
\endhead
\hline
\endfoot
\endlastfoot
\hline

% \begin{table}[!ht]
%   \begin{center}
%     \begin{tabular}{||>{\footnotesize}l<{\normalsize}|>{\footnotesize}c<{\normalsize}||} \hline
%     \textbf{Return Value} & \textbf{Meaning} \\ \hline

\txt{xs\_success} &
\begin{minipage}[t]{9cm}
The monitor has been successfully deleted and no other threads were
attempting an operation on it.
\end{minipage} \\

\txt{xs\_not\_owner} &
\begin{minipage}[t]{9cm}
The current thread does not own the monitor. A thread must own the monitor
or the monitor owner must be \txt{NULL} before it can be successfully
deleted.
\end{minipage} \\

\txt{xs\_waiting} &
\begin{minipage}[t]{9cm}
Some other threads were attempting an operation on the monitor. These
threads were successfully informed about the deletion of the monitor. In any
case, this could indicate bad programming.
\end{minipage} \\

\txt{xs\_incomplete} &

\begin{minipage}[t]{9cm}
Some threads were attempting an operation on the monitor but
haven't acknowledged yet that they are aborting this operation. Proceed
with caution in further deleting the monitor, like e.g. releasing the
memory of the monitor.
\end{minipage} \\

\txt{xs\_deleted} &

\begin{minipage}[t]{9cm}
Some other thread has been deleting this element already.
\end{minipage} \\

\txt{xs\_bad\_element} &

\begin{minipage}[t]{9cm}
The passed \txt{monitor} structure is not pointing to a valid monitor
structure.
\end{minipage} \\


\hline 
\multicolumn{2}{c}{} \\
\caption{Return Status for \txt{x\_monitor\_delete}}
\label{table:monitor_delete}
\end{longtable}
\normalsize

%     \hline
%     \end{tabular}
%     \caption{Return Status for \txt{x\_monitor\_delete}}
%     \label{table:monitor_delete}
%   \end{center}
% \end{table}

\subsubsection{Entering a Monitor}

\txt{x\_status x\_monitor\_enter(x\_monitor monitor, x\_sleep to);}

As already noted above, a thread can lock or enter a monitor several times.
When a thread already owns a monitor, the lock count is just incremented by
1.

The different return values that this call can produce are summarized
in table \ref{table:monitor_enter}.  


\footnotesize
\begin{longtable}{||l|p{9cm}||}
\hline
\hfill \textbf{Return Value} \hfill\null & \textbf{Meaning} \\ 
\hline
\endhead
\hline
\endfoot
\endlastfoot
\hline

% \begin{table}[!ht]
%   \begin{center}
%     \begin{tabular}{||>{\footnotesize}l<{\normalsize}|>{\footnotesize}c<{\normalsize}||} \hline
%     \textbf{Return Value} & \textbf{Meaning} \\ \hline

\txt{xs\_success} &
\begin{minipage}[t]{9cm}
The call succeeded and the current thread owns the monitor. If the current
thread owned the monitor already, the lock count has been incremented,
otherwise it is now 1.
\end{minipage} \\

\txt{xs\_no\_instance} &
\begin{minipage}[t]{9cm}
The monitor could not be locked by the current thread in the timeout window
given by the \txt{to} argument.
\end{minipage} \\

\txt{xs\_bad\_context} &
\begin{minipage}[t]{9cm}
A timeout value \txt{to} other than \txt{x\_no\_wait} has been given
in the context of a timer handler or interrupt handler.
\end{minipage} \\

\txt{xs\_deleted} &

\begin{minipage}[t]{9cm}
Another thread has deleted the monitor while this current thread was
trying to enter it.
\end{minipage} \\

\txt{xs\_bad\_element} &

\begin{minipage}[t]{9cm}
The passed \txt{monitor} structure is not pointing to a valid monitor
structure.
\end{minipage} \\


\hline 
\multicolumn{2}{c}{} \\
\caption{Return Status for \txt{x\_monitor\_enter}}
\label{table:monitor_enter}
\end{longtable}
\normalsize

%     \hline
%     \end{tabular}
%     \caption{Return Status for \txt{x\_monitor\_enter}}
%     \label{table:monitor_enter}
%   \end{center}
% \end{table}

\subsubsection{Waiting on a Monitor}

A thread can wait on a monitor by means of the following call:

\txt{x\_status x\_monitor\_wait(x\_monitor monitor, x\_sleep to);}

The different return values that this call can produce are summarized
in table \ref{table:monitor_wait}.  


\footnotesize
\begin{longtable}{||l|p{9cm}||}
\hline
\hfill \textbf{Return Value} \hfill\null & \textbf{Meaning} \\ 
\hline
\endhead
\hline
\endfoot
\endlastfoot
\hline

% \begin{table}[!ht]
%   \begin{center}
%     \begin{tabular}{||>{\footnotesize}l<{\normalsize}|>{\footnotesize}c<{\normalsize}||} \hline
%     \textbf{Return Value} & \textbf{Meaning} \\ \hline

\txt{xs\_success} &
\begin{minipage}[t]{9cm}
The call succeeded and the current thread has regained the lock on the
monitor. It should reevaluate the while condition.
\end{minipage} \\

\txt{xs\_not\_owner} &
\begin{minipage}[t]{9cm}
The current thread did not own the monitor. A thread should own the monitor,
by issuing a \txt{x\_monitor\_enter} before attempting a wait on a
monitor.
\end{minipage} \\

\txt{xs\_bad\_context} &
\begin{minipage}[t]{9cm}
A timeout value \txt{to} other than \txt{x\_no\_wait} has been given
in the context of a timer handler or interrupt handler. Could indicate bad
programming, to say it softly...
\end{minipage} \\

\txt{xs\_deleted} &

\begin{minipage}[t]{9cm}
Another thread has deleted the monitor while this current thread was
trying to enter it.
\end{minipage} \\

\txt{xs\_bad\_element} &

\begin{minipage}[t]{9cm}
The passed \txt{monitor} structure is not pointing to a valid monitor
structure.
\end{minipage} \\


\hline 
\multicolumn{2}{c}{} \\
\caption{Return Status for \txt{x\_monitor\_wait}}
\label{table:monitor_wait}
\end{longtable}
\normalsize


%     \hline
%     \end{tabular}
%     \caption{Return Status for \txt{x\_monitor\_wait}}
%     \label{table:monitor_wait}
%   \end{center}
% \end{table}

\subsubsection{Notifying Threads Waiting on a Monitor}

\begin{itemize}
\item Notifying a single thread, that is waiting on a monitor. The first in
line waiting. This done by means of the following call:

\txt{x\_status x\_monitor\_notify(x\_monitor monitor);}

The different return values that this call can produce are summarized
in table \ref{table:monitor_notify}.  

\item Notifying all threads that are waiting on a monitor.

\txt{x\_status x\_monitor\_notify\_all(x\_monitor monitor);}

The different return values that this call can produce are summarized
in table \ref{table:monitor_notify_all}.
\end{itemize}


\footnotesize
\begin{longtable}{||l|p{9cm}||}
\hline
\hfill \textbf{Return Value} \hfill\null & \textbf{Meaning} \\ 
\hline
\endhead
\hline
\endfoot
\endlastfoot
\hline

% \begin{table}[!ht]
%   \begin{center}
%     \begin{tabular}{||>{\footnotesize}l<{\normalsize}|>{\footnotesize}c<{\normalsize}||} \hline
%     \textbf{Return Value} & \textbf{Meaning} \\ \hline

\txt{xs\_success} &
\begin{minipage}[t]{9cm}
The call succeeded and the first thread that was waiting has been woken up.
The current thread still owns the lock though.
\end{minipage} \\

\txt{xs\_not\_owner} &
\begin{minipage}[t]{9cm}
The current thread did not own the monitor. A thread should own the monitor,
by issuing a \txt{x\_monitor\_enter} before attempting a notify on a
monitor.
\end{minipage} \\

\txt{xs\_deleted} &

\begin{minipage}[t]{9cm}
Another thread has deleted the monitor while this current thread was
trying to notify it.
\end{minipage} \\

\txt{xs\_bad\_element} &

\begin{minipage}[t]{9cm}
The passed \txt{monitor} structure is not pointing to a valid monitor
structure.
\end{minipage} \\


\hline 
\multicolumn{2}{c}{} \\
\caption{Return Status for \txt{x\_monitor\_notify}}
\label{table:monitor_notify}
\end{longtable}
\normalsize


%     \hline
%     \end{tabular}
%     \caption{Return Status for \txt{x\_monitor\_notify}}
%     \label{table:monitor_notify}
%   \end{center}
% \end{table}


\footnotesize
\begin{longtable}{||l|p{9cm}||}
\hline
\hfill \textbf{Return Value} \hfill\null & \textbf{Meaning} \\ 
\hline
\endhead
\hline
\endfoot
\endlastfoot
\hline

% \begin{table}[!ht]
%   \begin{center}
%     \begin{tabular}{||>{\footnotesize}l<{\normalsize}|>{\footnotesize}c<{\normalsize}||} \hline
%     \textbf{Return Value} & \textbf{Meaning} \\ \hline

\txt{xs\_success} &
\begin{minipage}[t]{9cm}
The call succeeded and all threads that were waiting are woken up.
The current thread still owns the lock though.
\end{minipage} \\

\txt{xs\_not\_owner} &
\begin{minipage}[t]{9cm}
The current thread did not own the monitor. A thread should own the monitor,
by issuing a \txt{x\_monitor\_enter} before attempting a notify all on a
monitor.
\end{minipage} \\

\txt{xs\_deleted} &

\begin{minipage}[t]{9cm}
Another thread has deleted the monitor while this current thread was
trying to notify it.
\end{minipage} \\

\txt{xs\_bad\_element} &

\begin{minipage}[t]{9cm}
The passed \txt{monitor} structure is not pointing to a valid monitor
structure.
\end{minipage} \\


\hline 
\multicolumn{2}{c}{} \\
\caption{Return Status for \txt{x\_monitor\_notify\_all}}
\label{table:monitor_notify_all}
\end{longtable}
\normalsize

%    \hline
%     \end{tabular}
%     \caption{Return Status for \txt{x\_monitor\_notify\_all}}
%     \label{table:monitor_notify_all}
%   \end{center}
% \end{table}

\subsubsection{Exiting a Monitor}

\txt{x\_status x\_monitor\_exit(x\_monitor monitor);}

The different return values that this call can produce are summarized
in table \ref{table:monitor_exit}.  


\footnotesize
\begin{longtable}{||l|p{9cm}||}
\hline
\hfill \textbf{Return Value} \hfill\null & \textbf{Meaning} \\ 
\hline
\endhead
\hline
\endfoot
\endlastfoot
\hline

% \begin{table}[!ht]
%   \begin{center}
%     \begin{tabular}{||>{\footnotesize}l<{\normalsize}|>{\footnotesize}c<{\normalsize}||} \hline
%     \textbf{Return Value} & \textbf{Meaning} \\ \hline

\txt{xs\_success} &
\begin{minipage}[t]{9cm}
The call succeeded and if the lock count of the monitor has reached 0, the
monitor has been released by the current thread. If the count did not reach
0, the current thread still owns the monitor.
\end{minipage} \\

\txt{xs\_deleted} &

\begin{minipage}[t]{9cm}
Another thread has deleted the monitor while this current thread was
trying to exit a monitor.
\end{minipage} \\

\txt{xs\_bad\_element} &

\begin{minipage}[t]{9cm}
The passed \txt{monitor} structure is not pointing to a valid monitor
structure.
\end{minipage} \\


\hline 
\multicolumn{2}{c}{} \\
\caption{Return Status for \txt{x\_monitor\_exit}}
\label{table:monitor_exit}
\end{longtable}
\normalsize

%     \hline
%     \end{tabular}
%     \caption{Return Status for \txt{x\_monitor\_exit}}
%     \label{table:monitor_exit}
%   \end{center}
% \end{table}

\subsubsection{Removing a Waiting Thread}

Sometimes, it is required that a certain thread removes another thread from
a list of waiting threads. This can be accomplished wit the following call:

\txt{x\_status x\_monitor\_stop\_waiting(x\_monitor monitor, x\_thread thread);}

The passed \txt{thread} is removed from the waiting list and as a result,
the status \txt{xs\_success} is returned. If \txt{thread} was not
found in the waiting list of a monitor, the status \txt{xs\_no\_instance}
is returned.




% ------------------------------------------------------------------------+
% Copyright (c) 2001 by Punch Telematix. All rights reserved.             |
%                                                                         |
% Redistribution and use in source and binary forms, with or without      |
% modification, are permitted provided that the following conditions      |
% are met:                                                                |
% 1. Redistributions of source code must retain the above copyright       |
%    notice, this list of conditions and the following disclaimer.        |
% 2. Redistributions in binary form must reproduce the above copyright    |
%    notice, this list of conditions and the following disclaimer in the  |
%    documentation and/or other materials provided with the distribution. |
% 3. Neither the name of Punch Telematix nor the names of other           |
%    contributors may be used to endorse or promote products derived      |
%    from this software without specific prior written permission.        |
%                                                                         |
% THIS SOFTWARE IS PROVIDED ``AS IS'' AND ANY EXPRESS OR IMPLIED          |
% WARRANTIES, INCLUDING, BUT NOT LIMITED TO, THE IMPLIED WARRANTIES OF    |
% MERCHANTABILITY AND FITNESS FOR A PARTICULAR PURPOSE ARE DISCLAIMED.    |
% IN NO EVENT SHALL PUNCH TELEMATIX OR OTHER CONTRIBUTORS BE LIABLE       |
% FOR ANY DIRECT, INDIRECT, INCIDENTAL, SPECIAL, EXEMPLARY, OR            |
% CONSEQUENTIAL DAMAGES (INCLUDING, BUT NOT LIMITED TO, PROCUREMENT OF    |
% SUBSTITUTE GOODS OR SERVICES; LOSS OF USE, DATA, OR PROFITS; OR         |
% BUSINESS INTERRUPTION) HOWEVER CAUSED AND ON ANY THEORY OF LIABILITY,   |
% WHETHER IN CONTRACT, STRICT LIABILITY, OR TORT (INCLUDING NEGLIGENCE    |
% OR OTHERWISE) ARISING IN ANY WAY OUT OF THE USE OF THIS SOFTWARE, EVEN  |
% IF ADVISED OF THE POSSIBILITY OF SUCH DAMAGE.                           |
% ------------------------------------------------------------------------+

%
% $Id: map.tex,v 1.1.1.1 2004/07/12 14:07:44 cvs Exp $
%

\subsection{Bitmaps}

\subsubsection{Operation}

Bitmaps can be used to store a binary state. A bitmap can contain, at a
certain position, called an 'index' a state symbolized by a 0 (reset state)
or a 1 (set state). Bitmaps can be used for whatever functionality where a
programmer wants to store a 0 or 1 state in a very space conserving way.

In \oswald, bitmaps are always a multiple of a 32 bit rounded number, since
the bits are stored on a word basis. This means that the smallest bitmap that
can be created is 32 bits wide or 1 word wide.

The positions in a bit map start out at 0 and go up to the number of
positions as given in the respective create calls, minus 1. To say it
simple, when a map is created with 64 positions, the indexes start at 0 and
go up to and including position 63.

In \oswald, there are 2 flavors of bitmaps, that also have slightly
different semantics (different meaning for the application interface
functions).

\begin{itemize}
\item Non event bitmaps; these bitmaps structures don't allow synchronized
access. Two threads can be changing, setting or resetting, the same bit at
the same time, and the outcome of operations that are happening virtually at
the same time is undefined. These bitmaps can be used to assign unique ids
or can be used by a single thread at the time.
\item Event or synchronized bitmaps; these bitmaps do allow several threads
to operate on them at the same time and \oswald will make sure that the
access is controlled.
\end{itemize}

We refer to the different functions outlined below to make the difference in
meaning clear. The non synchronized bitmaps are structures of the type
\txt{x\_Umap}, for 'unsynchronized map', while the event like bitmaps are
structures of type \txt{x\_Map}. References to these structure are
indicated by the \txt{x\_umap} and \txt{x\_map} types respectively.
The structure of both types of maps is given below in the structure
definition.

The \txt{x\_map} event bitmaps are implemented by means of the
\txt{x\_umap} calls, so that no code is duplicated in \oswald. The
\txt{x\_Map} type is also implemented by including the \txt{x\_Umap} as
shown below in the structure definition.

\subsubsection{Umap and Map Structure Definition}

The structure definition of an umap is as follows:

\bcode
\begin{verbatim}
 1: typedef struct x_Umap * x_umap;
 2:
 3: typedef struct x_Umap {
 5:   w_size entries;
 6:   w_word mask;
 7:   volatile w_size cache_0;
 8:   w_word * table;
 9: } x_Umap;
\end{verbatim}
\ecode

The relevant fields in the umap structure are the following:

\begin{itemize}
\item \txt{x\_umap$\rightarrow$entries} The number of bits that this map
contains. Note that this field, which is passed as an argument at the map
creation time, is always rounded up to a multiple of 32.
\item \txt{x\_umap$\rightarrow$mask} This is the mask we should use to
clear out the space required for an index in a 32 bit word. This is usefull
in the case where a 32 bit word is used for both flags and an id. The lower
bits of the flags word can be used to store the index than. The mask can be
used to clear this area in the flags word.
\item \txt{x\_umap$\rightarrow$cache\_0} When looking for the first bit in
the map that is 0, this is the index in the table array of words where the
first 0 bit can be found.
\item \txt{x\_umap$\rightarrow$table} The memory area passed at creation
time where the bits of the map will be stored. It must be able to accommodate
the number of words indicated by \txt{x\_umap$\rightarrow$entries} /
\txt{sizeof(w\_word)};
\end{itemize}

\subsubsection{Map Structure Definition}

The structure definition of a map is as follows:

\bcode
\begin{verbatim}
 1: typedef struct x_Map * x_map;
 2:
 3: typedef struct x_Map {
 4:   x_Event Event;
 5:   x_Umap umap;
 6: } x_Map;
\end{verbatim}
\ecode

The relevant fields in the map structure are the following:

\begin{itemize}
\item \txt{x\_map$\rightarrow$Event} This is the universal event structure that is a field
in all threadable components or elements. It controls the synchronized access
to the map component and the signalling between threads to indicate changes
in the map structure.
\item \txt{x\_map$\rightarrow$Umap} The Umap structure that will be used
to manipulate the bit positions, as defined above.
\end{itemize}

For more information on the two types of bitmaps and the difference between
the two types, please refer to the next section on operation of a bitmap.

\subsubsection{Non Event Maps}

The following operations are defined for non synchronized bitmaps.

\begin{itemize}
\item Creating a map, note that there is no delete function for this kind of
maps.
\item Setting the first available 0 bit to 1. I.e. requesting for a bit to
be set from 0 to 1 at \textbf{any} index that has a 0 bit available.
\item Setting a bit from 0 to 1 at a specific index in the map.
\item Resetting a given 1 bit to 0 again.
\item Probing the status of a bit at a certain position.
\end{itemize}

As already indicated, when several threads have simultaneous access to a
map, the outcome of any of these calls is undefined.

\subsubsection{Creating a Non Event Map}

\txt{x\_size x\_umap\_create(x\_umap umap, w\_size entries, w\_word * table);}

This creates the map, with \txt{entries} number of indexes. The
\txt{entries} number is first rounded up to a size that is a multiple of
32, since maps are handled on a word basis. Therefore, the space pointed to
by the \txt{table} argument, must be able to accommodate the correct,
rounded up number of words.

This function returns the rounded up number of entries that are available in
the bitmap. This value returned is the number of entries that will be used
for internal calculations. E.g., when the call is performed with
\txt{entries} equal to 60, the table will accommodate 64 entries and the
\txt{table} argument must point to a suitably sized space. This
information can also be found in the \txt{x\_umap$\rightarrow$entries}
field of the \txt{x\_Umap} structure.

The possible index numbers start at 0 and go up to and including
\txt{entries} - 1.

\subsubsection{Setting A Non Specific Bit in a Non Event Map}

For requesting that the first available 0 bit will be set, the following
call can be used:

\txt{x\_status x\_umap\_any(x\_umap umap, w\_size * index);}

The different return values that this call can produce are summarized
in table \ref{table:umap_any}.  



\footnotesize
\begin{longtable}{||l|p{9cm}||}
\hline
\hfill \textbf{Return Value} \hfill\null & \textbf{Meaning}  \\ 
\hline
\endhead
\hline
\endfoot
\endlastfoot
\hline

% \begin{table}[!ht]
%   \begin{center}
%     \begin{tabular}{||>{\footnotesize}l<{\normalsize}|>{\footnotesize}c<{\normalsize}||} \hline
%     \textbf{Return Value} & \textbf{Meaning} \\ \hline

\txt{xs\_success} &
\begin{minipage}[t]{9cm}
The call succeeded and the pointer indicated by \txt{index} is set to the
bit index that has been set to 1.
\end{minipage} \\

\txt{xs\_no\_instance} &

\begin{minipage}[t]{9cm}
There was no single 0 bit available in the map. The value pointed to by
\txt{index} has not been changed.
\end{minipage} \\


\hline 
\multicolumn{2}{c}{} \\
\caption{Return Status for \txt{x\_umap\_any}}
\label{table:umap_any}
\end{longtable}
\normalsize



%     \hline
%     \end{tabular}
%     \caption{Return Status for \txt{x\_umap\_any}}
%     \label{table:umap_any}
%   \end{center}
% \end{table}

\subsubsection{Setting a Specific Bit in a Non Event Map}

The function call to set a bit to 1 at a specific index in the table is
given by:

\txt{x\_boolean x\_umap\_set(x\_umap umap, w\_size index);}

This function returns a boolean \txt{true} or \txt{false} value; it
returns \txt{true} when the bit at \txt{index} was changed from a 0 to
a 1 position. It returns \txt{false}, when the bit position at
\txt{index} already was set to 1, i.e. did not change.

Note that when the index requested to be set, exceeds the number of entries
in the map, the return value will always be \txt{false} and the bitmap
remains unchanged.

\subsubsection{Resetting a Bit in a Non Event Map}

A specific bit can be reset to 0, by means of the following function call:

\txt{x\_status x\_umap\_reset(x\_umap umap, w\_size index);}

The different return values that this call can produce are summarized
in table \ref{table:umap_reset}.  



\footnotesize
\begin{longtable}{||l|p{9cm}||}
\hline
\hfill \textbf{Return Value} \hfill\null & \textbf{Meaning}  \\ 
\hline
\endhead
\hline
\endfoot
\endlastfoot
\hline



% \begin{table}[!ht]
%   \begin{center}
%     \begin{tabular}{||>{\footnotesize}l<{\normalsize}|>{\footnotesize}c<{\normalsize}||} \hline
%     \textbf{Return Value} & \textbf{Meaning} \\ \hline

\txt{xs\_success} &
\begin{minipage}[t]{9cm}
The call succeeded and the bit at \txt{index} has been successfully reset
from 1 to 0.
\end{minipage} \\

\txt{xs\_no\_instance} &

\begin{minipage}[t]{9cm}
The \txt{index} was beyond the number of entries of the bitmap or the bit
at position \txt{index} was already 0, so no change happened.
\end{minipage} \\


\hline 
\multicolumn{2}{c}{} \\
\caption{Return Status for \txt{x\_umap\_reset}}
\label{table:umap_reset}
\end{longtable}
\normalsize



%     \hline
%     \end{tabular}
%     \caption{Return Status for \txt{x\_umap\_reset}}
%     \label{table:umap_reset}
%   \end{center}
% \end{table}

\subsubsection{Probing a Bit in a Non Event Map}

The value of a bit in a non event map can be determined by means of the
following function call:

\txt{x\_boolean x\_umap\_probe(x\_umap umap, w\_size index);}

The function returns the boolean value \txt{true} when the bit at
position \txt{index} is set to 1. It will return the value \txt{value}
when the bit at position \txt{index} is reset to 0 or when the value of
\txt{index} is beyond the number available positions in the bit map.

\subsubsection{Event Bitmaps}

Event bitmaps can be used to communicate between more than 1 thread. The
event mechanisms of \oswald will prevent simultaneous modification of the map
in an uncontrolled or undefined way.

Threads can wait until a certain bit is reset from 1 to 0, or until a 0
bit becomes available. 

Note that the structure type of an event type map is \txt{x\_Map} and a
pointer to this structure is of the type \txt{x\_map}.

\subsubsection{Creating an Event Map} 

An event bitmap can be created by means of the following call.

\txt{x\_status x\_map\_create(x\_map map, w\_size entries, w\_word * table);}

The arguments passed have the same semantic meaning as with the
\txt{x\_umap\_create} call described above, except the return argument is
different. 

The value of the \txt{entries} is rounded up to the nearest value that is
divisible by 32. The result of this rounding up is not returned as a result,
but can be found back in the field \txt{map$\rightarrow$Umap.entries},
i.e. in the \txt{entries} field of the non synchronized bitmap, that is
one of the fields of the \txt{x\_Map} structure.

\subsubsection{Deleting an Event Map} 

\subsubsection{Setting A Non Specific Bit in an Event Map}

Requesting that the first available 0 bit be set to 1 in an event map, is
established by means of the following function call:

\txt{x\_status x\_map\_any(x\_map map, w\_size * index, x\_window to);}

The \txt{index} points to a an index variable and will be suitably
updated upon successful return of the call. The timeout argument
\txt{to} indicates the window during which the thread will wait for a 0
bit to become available. It can take the normal range of timeout, starting
from \txt{x\_no\_wait} up to the value \txt{x\_eternal} to wait
forever for a bit to become available. 

The different return values that this call can produce are summarized
in table \ref{table:map_any}.  




\footnotesize
\begin{longtable}{||l|p{9cm}||}
\hline
\hfill \textbf{Return Value} \hfill\null & \textbf{Meaning}  \\ 
\hline
\endhead
\hline
\endfoot
\endlastfoot
\hline



% \begin{table}[!ht]
%   \begin{center}
%     \begin{tabular}{||>{\footnotesize}l<{\normalsize}|>{\footnotesize}c<{\normalsize}||} \hline
%     \textbf{Return Value} & \textbf{Meaning} \\ \hline

\txt{xs\_success} &
\begin{minipage}[t]{9cm}
The call succeeded and the bit at the index pointed to by the \txt{index}
argument has been successfully set from 0 to 1.
\end{minipage} \\

\txt{xs\_no\_instance} &

\begin{minipage}[t]{9cm}
The call did not result in a bit being set from 0 to 1 at any position in
the map, within the timeout window given by the \txt{to} argument.
\end{minipage} \\

\txt{xs\_bad\_context} &

\begin{minipage}[t]{9cm}
The timeout argument \txt{to} was not \txt{x\_no\_wait} in the context
of a interrupt handler or timer handler.
\end{minipage} \\

\txt{xs\_deleted} &

\begin{minipage}[t]{9cm}
Another thread has deleted the bitmap when this thread was attempting
the set operation.
\end{minipage} \\


\hline 
\multicolumn{2}{c}{} \\
\caption{Return Status for \txt{x\_map\_any}}
\label{table:map_any}
\end{longtable}
\normalsize


%     \hline
%     \end{tabular}
%     \caption{Return Status for \txt{x\_map\_any}}
%     \label{table:map_any}
%   \end{center}
% \end{table}

\subsubsection{Setting a Specific Bit in an Event Map}

A specific bit in an event bitmap can be set by means of the following call:

\txt{x\_status x\_map\_set(x\_map map, w\_size index, x\_window to);}

The \txt{index} argument indicates the bit position that needs to be set
from 0 to 1. Note that while the \txt{x\_umap\_set} function call
returned a status value indicating whether the bit was flipped from 0 to 1 or
was set to 1 already, this call does not offer this kind of functionality.

In the event version of a bitmap, when the bit at position \txt{index} is
already set to 1, the call will block until the bit becomes set to 0 again,
for as long as indicated by the \txt{to} argument.

This behavior can be compared with the mutex. Please note that this is a
very poor mutex with respect to performance and behavior, i.e. a normal
\txt{x\_mutex} is better w.r.t. performance and priority inversion
behavior; a \txt{x\_mutex} also records which thread owns the mutex and
is thus better suited for some locking functionality.

A map in \oswald also bears some resemblance with the \txt{x\_signals}
structure and semantics. The \txt{x\_signals} has much more powerfull
semantics however as combinations of flags are supported and clearing is
also user controllable. The event bitmaps in \oswald have less possibilities
but offer a bulk bit store with thread synchronization primitives, while
signals are much more flexible, but can only handle 31 bits at a time.

The different return values that this call can produce are summarized
in table \ref{table:map_set}.  



\footnotesize
\begin{longtable}{||l|p{9cm}||}
\hline
\hfill \textbf{Return Value} \hfill\null & \textbf{Meaning}  \\ 
\hline
\endhead
\hline
\endfoot
\endlastfoot
\hline



% \begin{table}[!ht]
%   \begin{center}
%     \begin{tabular}{||>{\footnotesize}l<{\normalsize}|>{\footnotesize}c<{\normalsize}||} \hline
%     \textbf{Return Value} & \textbf{Meaning} \\ \hline

\txt{xs\_success} &
\begin{minipage}[t]{9cm}
The call succeeded and the bit at the index given by the \txt{index}
argument has been successfully set from 0 to 1.
\end{minipage} \\

\txt{xs\_no\_instance} &

\begin{minipage}[t]{9cm}
Within the timeout given by the \txt{to} argument, the bit could not be
set from 0 to 1, or the value of the \txt{index} argument exceeds the
number of bit positions in the bitmap.
\end{minipage} \\

\txt{xs\_bad\_context} &

\begin{minipage}[t]{9cm}
The timeout argument \txt{to} was not \txt{x\_no\_wait} in the context
of a interrupt handler or timer handler.
\end{minipage} \\

\txt{xs\_deleted} &

\begin{minipage}[t]{9cm}
Another thread has deleted the bitmap when this thread was attempting
the set operation.
\end{minipage} \\


\hline 
\multicolumn{2}{c}{} \\
\caption{Return Status for \txt{x\_map\_set}}
\label{table:map_set}
\end{longtable}
\normalsize


%     \hline
%     \end{tabular}
%     \caption{Return Status for \txt{x\_map\_set}}
%     \label{table:map_set}
%   \end{center}
% \end{table}

\subsubsection{Resetting a Bit in an Event Map}

A bit can be reset from the 1 to the 0 state in an event bitmap, by means of
the following call:

\txt{x\_status x\_map\_reset(x\_map map, w\_size index);}

The bit at position \txt{index} will be set from 1 to 0. When the bit did
not have the value 1, i.e. was not set, this call will return
\txt{xs\_no\_instance}. When \txt{index} was going beyond the range of
entries in the bitmap, the status returned is also
\txt{xs\_no\_instance}. When the bit was set from 1 to 0, the status
returned is \txt{xs\_success}.

The different return values that this call can produce are summarized
in table \ref{table:map_reset}.  



\footnotesize
\begin{longtable}{||l|p{9cm}||}
\hline
\hfill \textbf{Return Value} \hfill\null & \textbf{Meaning}  \\ 
\hline
\endhead
\hline
\endfoot
\endlastfoot
\hline



% \begin{table}[!ht]
%   \begin{center}
%     \begin{tabular}{||>{\footnotesize}l<{\normalsize}|>{\footnotesize}c<{\normalsize}||} \hline
%     \textbf{Return Value} & \textbf{Meaning} \\ \hline

\txt{xs\_success} &
\begin{minipage}[t]{9cm}
The call succeeded and the bit at the index given by the \txt{index}
argument has been successfully reset from 1 to 0.
\end{minipage} \\

\txt{xs\_no\_instance} &

\begin{minipage}[t]{9cm}
The bit at position \txt{index} did not have value 1, or the
\txt{index} argument was beyond the maximum number of entries allowed in
this map.
\end{minipage} \\

\txt{xs\_deleted} &

\begin{minipage}[t]{9cm}
Another thread has deleted the bitmap when this thread was attempting
the reset operation.
\end{minipage} \\


\hline 
\multicolumn{2}{c}{} \\
\caption{Return Status for \txt{x\_map\_reset}}
\label{table:map_reset}
\end{longtable}
\normalsize


%     \hline
%     \end{tabular}
%     \caption{Return Status for \txt{x\_map\_reset}}
%     \label{table:map_reset}
%   \end{center}
% \end{table}

\subsubsection{Probing a Bit in an Event Map} 

The status of a bit in an event bitmap can be probed by means of the
following call:

\txt{x\_status x\_map\_probe(x\_map map, w\_size index, x\_boolean * value);}

Upon successfull completion of this call, the argument pointed to by
\txt{value} will contain the boolean value of \txt{true} or
\txt{false} if the bit at position \txt{index} was set to 1 or set to
0, respectively.

The different return values that this call can produce are summarized
in table \ref{table:map_probe}.  



\footnotesize
\begin{longtable}{||l|p{9cm}||}
\hline
\hfill \textbf{Return Value} \hfill\null & \textbf{Meaning}  \\ 
\hline
\endhead
\hline
\endfoot
\endlastfoot
\hline


% \begin{table}[!ht]
%   \begin{center}
%     \begin{tabular}{||>{\footnotesize}l<{\normalsize}|>{\footnotesize}c<{\normalsize}||} \hline
%     \textbf{Return Value} & \textbf{Meaning} \\ \hline

\txt{xs\_success} &
\begin{minipage}[t]{9cm}
The call succeeded and the value of the bit at the index given by the \txt{index}
argument has been written in the argument pointed to by \txt{value}.
\end{minipage} \\

\txt{xs\_no\_instance} &

\begin{minipage}[t]{9cm}
The \txt{index} argument was beyond the maximum number of entries allowed in
this map.
\end{minipage} \\

\txt{xs\_deleted} &

\begin{minipage}[t]{9cm}
Another thread has deleted the bitmap when this thread was attempting
the probe operation.
\end{minipage} \\


\hline 
\multicolumn{2}{c}{} \\
\caption{Return Status for \txt{x\_map\_probe}}
\label{table:map_probe}
\end{longtable}
\normalsize


%     \hline
%     \end{tabular}
%     \caption{Return Status for \txt{x\_map\_probe}}
%     \label{table:map_probe}
%   \end{center}
% \end{table}



















%
% $Id: memory.tex,v 1.1.1.1 2004/07/12 14:07:44 cvs Exp $
%

\subsection{Memory Allocation}

\subsubsection{Operation}

\oswald provides mechanisms for dynamically allocating and releasing memory,
much like the well known \txt{malloc} and \txt{free} functions from
the standard C library\footnote{The memory allocation in \oswald is based on
the ideas of the very well known and very good memory routines of Doug Lea.
Doug, when we ever meet, I'll buy you a beer...}.
Besides allocating and freeing memory like malloc and free does, the memory
routines of \oswald provide extra functionality like a fast freeing function
with later scaveging of blocks, memory tagging and functions to walk over
the memory blocks and calling a user defined function on each block.

There are however, some issues that are specific for the \oswald dynamic
memory allocator.

\begin{itemize}
\item The largest memory chunk that can be allocated in a single call is
8388596 bytes or 8 Mb - 4 bytes, exactly. Limiting the blocks to this maximum size enables us
to implement memory tags, described in the next bullet. Note that this does
not mean that only 8 Mb is available; if more than 8 Mb of memory is
required to perform a job, several allocations need to be done one after
another. 
\item Memory tags are 9 bits of information that can be assigned to a block
of allocated memory, for which no extra space needs to be allocated. The 9
bit tags are contained in the normal internal memory householding
information. The tag bits can be used for whatever purpose the programmer
has in mind. 
\item \oswald provides a function, with a callback mechanism, that allows a
programmer to walk over all the allocated memory blocks. This feature, in
combination with the tag bits, can provide powerfull mechanisms, without
memory space overhead, to manipulate certain types of memory blocks. It can
reduce the number of referal pointers and thus memory space. It can for
instance we used to replace 'next' and 'previous' pointers in structures
that would require a linked list operation and for which speed is not that
important.
\item \oswald provides a \txt{x\_mem\_discard} function that very quickly marks a
memory block as garbage. It doesn't free the memory yet, at the moment of
the call. The freeing is performed by a call to \txt{x\_mem\_collect}
that coalesces and frees all garbage blocks in a single call. This reduces
locking and unlocking overhead and probably reduces fragmentation when the
collecting is done after a subsequent discarded of a whole slew of memory
blocks.
\end{itemize}

\subsubsection{Allocating Memory}

Memory is allocated by means of the following call:

\txt{void * x\_mem\_alloc(x\_size numbytes, x\_word id);}

which returns \txt{NULL} on failure or a non \txt{NULL} pointer when the allocation was
successful. Allocations larger than 8 Mb - 4 bytes always return \txt{NULL}. The
contents of the returned memory block are not cleared.

For obtaining a cleared memory block, the call:

\txt{void * x\_mem\_calloc(x\_size numbytes, x\_word id);}

can be used. The function internally uses a duff's device for quickly
clearing memory.

Note that both allocation functions require and ID to the passed. This is
the information that will be stored in the header of the block, in the tag
field. ID values 0 to 31 inclusive are reserved for \oswald.

\subsubsection{Reallocating Memory}

\txt{void * x\_mem\_realloc(void * block, x\_size numbytes);}

\subsubsection{Releasing Memory}

\txt{void x\_mem\_free(void * block);}

\subsubsection{Discarding Memory \& Collecting It}

\txt{void x\_mem\_discard(void * block);}

\txt{x\_status x\_mem\_collect(x\_size * bytes, x\_size * blocks);}

\subsubsection{Setting \& Reading Tags}

Memory tags are a 9 bit piece of information that can be used and that carry
no overhead in the memory usage. They are usefull to replace \txt{next}
and \txt{previous} pointer links in lists. When a structure is 'listed'
but doesn't require fast removal and insertion or lookup, the tag can be
used to store the type of the structure and the memory walker function can
be used to check for a certain tag and perform operations on the found
blocks. 

A tag can be set with the call:

\txt{x\_status x\_mem\_tag\_set(void * block, x\_word tag);}

The lower 9 bits of the tag will be stored in the memory chunk, the higher
bits above the first 9 bytes are silently discarded. The return value of
this call is the return status of the \txt{x\_monitor\_enter} or
\txt{x\_monitor\_exit} call that is performed inside, since the tag is
copied in the chunk after locking the memory functions.

The 9 bits can be read out by means of the call:

\txt{x\_word x\_mem\_tag\_get(void * block);}

Which will return the 9 bits of tag information in the lower 9 bits of the
returned word.

Note that \oswald reserves the values 0 to 31 inclusive, of the tag
information. With 9 bits of information, this leaves us with, $512 - 32 =
480$ user defined values that can be stored in the tag bits. Not much, but
they don't carry overhead either...

\subsubsection{Getting the Size of a Chunk}

The following function will return the useable size in bytes of a block of
memory.

\txt{x\_size x\_mem\_size(void * block);}

The value returned could be very well larger than the number of bytes that
has been requested for. Since all memory chunks size are 8 byte aligned, a
call to \txt{x\_mem\_alloc(22)} will yield a return value from this
function of 28. Also when the memory allocator is chopping up a block to
return in a call to e.g. \txt{x\_mem\_alloc} and the remainder block
would become smaller than 16 bytes, the block is not chopped up and the
internally recorded size will be larger than the argument to the allocator
function. In any case, it is safe to use the number of bytes returned by
this call for storing data.

\subsubsection{Iterating over Allocated Memory}

\oswald provides a function that can be used to walk over all the allocated
memory blocks.

\txt{x\_status x\_mem\_walk(x\_sleep to, void (*cb)(void * m, void * a), void * a);}

This function will walk over all the memory blocks and will call the
callback function \txt{cb} for each memory block that is not freed or
garbage. The arguments to the callback function are \txt{m} which is a
pointer to the in use memory block and \txt{a} a copy of the pointer
argument passed to the memory walk function, which can be used for whatever
purpose. 

The \txt{to} parameter is a timeout. Before all the blocks are scanned,
the memory allocator is locked and this timeout value is used to acquire the
memory allocator. Its meaning is the same as the timeout parameter of the
\txt{x\_monitor\_enter} function. After the walk, the lock is released.

The return status of the function is the return status of the monitor lock
that is used inside the walking function. Since the walker function provides
a security mechanism that prevents the same thread of calling it twice
(this could happen by the code in the callback function), the return status
can also be \txt{xs\_bad\_context} to indicate that same thread tried to
call the walker function more than once.

Note that you \textit{SHOULD NOT} call \txt{x\_mem\_free} from within the
callback function as it would change the internal lists and could crash.
There is no provision yet to secure for this situation\footnote{It would not
be difficult to do so, but would increase overhead during normal
operations.}. It is safe to call \txt{x\_mem\_discard} on the block since
this call does not change the interal lists.

\subsubsection{Pointer Checking}

A function that can be used to check if a certain pointer is pointing to a
memory block that has been allocated through any of the allocation routines
is:

\txt{x\_boolean x\_mem\_is\_block(void * pointer);}

It will return \txt{true} when the pointer points to memory that is
allocated and is not freed allready and is not garbage.


% ------------------------------------------------------------------------+
% Copyright (c) 2001 by Punch Telematix. All rights reserved.             |
%                                                                         |
% Redistribution and use in source and binary forms, with or without      |
% modification, are permitted provided that the following conditions      |
% are met:                                                                |
% 1. Redistributions of source code must retain the above copyright       |
%    notice, this list of conditions and the following disclaimer.        |
% 2. Redistributions in binary form must reproduce the above copyright    |
%    notice, this list of conditions and the following disclaimer in the  |
%    documentation and/or other materials provided with the distribution. |
% 3. Neither the name of Punch Telematix nor the names of other           |
%    contributors may be used to endorse or promote products derived      |
%    from this software without specific prior written permission.        |
%                                                                         |
% THIS SOFTWARE IS PROVIDED ``AS IS'' AND ANY EXPRESS OR IMPLIED          |
% WARRANTIES, INCLUDING, BUT NOT LIMITED TO, THE IMPLIED WARRANTIES OF    |
% MERCHANTABILITY AND FITNESS FOR A PARTICULAR PURPOSE ARE DISCLAIMED.    |
% IN NO EVENT SHALL PUNCH TELEMATIX OR OTHER CONTRIBUTORS BE LIABLE       |
% FOR ANY DIRECT, INDIRECT, INCIDENTAL, SPECIAL, EXEMPLARY, OR            |
% CONSEQUENTIAL DAMAGES (INCLUDING, BUT NOT LIMITED TO, PROCUREMENT OF    |
% SUBSTITUTE GOODS OR SERVICES; LOSS OF USE, DATA, OR PROFITS; OR         |
% BUSINESS INTERRUPTION) HOWEVER CAUSED AND ON ANY THEORY OF LIABILITY,   |
% WHETHER IN CONTRACT, STRICT LIABILITY, OR TORT (INCLUDING NEGLIGENCE    |
% OR OTHERWISE) ARISING IN ANY WAY OUT OF THE USE OF THIS SOFTWARE, EVEN  |
% IF ADVISED OF THE POSSIBILITY OF SUCH DAMAGE.                           |
% ------------------------------------------------------------------------+

%
% $Id: exception.tex,v 1.1.1.1 2004/07/12 14:07:44 cvs Exp $
%

\subsection{Exception Handling}

\subsubsection{Operation}

\subsubsection{Exception Structure Definition}

The structure definition of an exception and the ancillary structures, is as follows:

\bcode
\begin{verbatim}
 1: typedef void (*x_exception_cb)(void * arg);
 2: typedef struct x_Xcb * x_xcb;
 3:
 4: typedef struct x_Xcb {
 5:   x_xcb previous;
 6:   x_xcb next;
 7:   x_exception_cb cb;
 8:   void * arg;
 9: } x_Xcb;
10:
11: typedef struct x_Exception * x_exception;
12:
13: typedef struct x_Mutex {
14:   void * pc;
15:   void * sp;
16:   unsigned int registers[NUM_CALLEE_SAVED];
17:   x_boolean fired;
18:   x_xcb callbacks;
19:   x_Xcb Callbacks;
20: } x_Exception;
\end{verbatim}
\ecode

The relevant fields in the exception structure are the following:

\begin{itemize}
\item \txt{x\_exception$\rightarrow$pc} This field is used by the
\textsf{x\_context\_save} function to store the program counter of the
instruction following the call.
\item \txt{x\_exception$\rightarrow$sp} This field is used by the
\textsf{x\_context\_save} function to store the stack pointer at the moment
of the the call.
\item \txt{x\_exception$\rightarrow$registers} Is used to store all the
registers that are normally saved by the callee\footnote{The function that
is being called.}. This array is filled by the CPU specific
\textsf{x\_context\_save} function. Its contents are later restored when an
exception is being thrown, by the \textsf{x\_context\_restore} function. The
size of this array is CPU specific and the macro definition
\textsf{NUM\_CALLEE\_SAVED} is set in the include file of the specific CPU.
\item \txt{x\_exception$\rightarrow$fired} Is a boolean that indicates wether
an exception has been thrown or not.
\item \txt{x\_exception$\rightarrow$callbacks} Is the begin of a circular
linked list of callbacks that will be executed when an exception has been
thrown. This pointer is preset to the address of the embedded
\textsf{Callbacks} structure that is the last element of this structure.
\end{itemize}

\subsubsection{Using Native Exceptions}


% ------------------------------------------------------------------------+
% Copyright (c) 2001 by Punch Telematix. All rights reserved.             |
%                                                                         |
% Redistribution and use in source and binary forms, with or without      |
% modification, are permitted provided that the following conditions      |
% are met:                                                                |
% 1. Redistributions of source code must retain the above copyright       |
%    notice, this list of conditions and the following disclaimer.        |
% 2. Redistributions in binary form must reproduce the above copyright    |
%    notice, this list of conditions and the following disclaimer in the  |
%    documentation and/or other materials provided with the distribution. |
% 3. Neither the name of Punch Telematix nor the names of other           |
%    contributors may be used to endorse or promote products derived      |
%    from this software without specific prior written permission.        |
%                                                                         |
% THIS SOFTWARE IS PROVIDED ``AS IS'' AND ANY EXPRESS OR IMPLIED          |
% WARRANTIES, INCLUDING, BUT NOT LIMITED TO, THE IMPLIED WARRANTIES OF    |
% MERCHANTABILITY AND FITNESS FOR A PARTICULAR PURPOSE ARE DISCLAIMED.    |
% IN NO EVENT SHALL PUNCH TELEMATIX OR OTHER CONTRIBUTORS BE LIABLE       |
% FOR ANY DIRECT, INDIRECT, INCIDENTAL, SPECIAL, EXEMPLARY, OR            |
% CONSEQUENTIAL DAMAGES (INCLUDING, BUT NOT LIMITED TO, PROCUREMENT OF    |
% SUBSTITUTE GOODS OR SERVICES; LOSS OF USE, DATA, OR PROFITS; OR         |
% BUSINESS INTERRUPTION) HOWEVER CAUSED AND ON ANY THEORY OF LIABILITY,   |
% WHETHER IN CONTRACT, STRICT LIABILITY, OR TORT (INCLUDING NEGLIGENCE    |
% OR OTHERWISE) ARISING IN ANY WAY OUT OF THE USE OF THIS SOFTWARE, EVEN  |
% IF ADVISED OF THE POSSIBILITY OF SUCH DAMAGE.                           |
% ------------------------------------------------------------------------+

%
% $Id: thread.tex,v 1.1.1.1 2004/07/12 14:07:44 cvs Exp $
%

\subsection{Threads}

\subsubsection{Thread Structure Definition}

The structure definition of a thread is as follows:

\bcode
\begin{verbatim}
 1: typedef struct x_Thread * x_thread;
 2: typedef void (*x_entry)(void * argument);
 3: typedef void (*x_action)(x_thread thread);
 4:
 5: typedef struct x_Thread {
 6:   x_cpu cpu;
 7:   w_ubyte * sp;
 8:   w_ubyte * b_stack;
 9:   w_ubyte * e_stack;
10:   x_entry entry;
11:   void * argument;
12:   w_ushort id;
13:   w_ubyte a_prio;
14:   volatile w_ubyte c_prio;
15:   w_ubyte c_quantums;
16:   w_ubyte a_quantums;
17:   volatile w_ubyte state;
18:   void * xref;
19:   x_flags flags;
20:   volatile x_thread next;
21:   volatile x_thread previous;
22:   x_sleep sticks;
23:   x_sleep wakeup;
24:   volatile x_thread snext;
25:   x_action action;
26:   volatile x_thread l_waiting;
27:   volatile x_monitor waiting_for;
28:   volatile x_thread l_competing;
29:   volatile x_event competing_for;
30:   volatile x_event l_owned;
31:   w_size m_count;
32: } x_Thread;
\end{verbatim}
\ecode

The relevant fields in the thread structure are the following:

\begin{itemize}
\item \txt{x\_thread$\rightarrow$cpu} This is the CPU specific
placeholder for things like saved stack pointer, program counter and other
arguments. Please refer to the cpu section for more information on specific
fields in this structure. TODO
\item \txt{x\_thread$\rightarrow$s\_sp} For some ports, this is the field
in which the stack pointer is saved at the moment of a thread switch.
\item \txt{x\_thread$\rightarrow$b\_stack} This is the lowest address of
the user supplied stack space. For the user, it is the first available byte of
the memory that was allocated and passed as stack space at thread creation
time. Note that on most processors and hosts, the stack grows downwards and
this is thus, from this viewpoint, the end of the stack.
\item \txt{x\_thread$\rightarrow$e\_stack} This is the opposite of the
previous field, the end of the stack, or from the point of view of
processors that have a downwards growing stack, the place where the pushes
of stack items begins. Note that this value is word aligned, so it could be
that it is not the end of the memory region that has been passed at thread
creation time.
\item \txt{x\_thread$\rightarrow$entry} This is the entry function for
the thread. At thread start, this is the function that will be called. The
thread will end when it returns from this function.
\item \txt{x\_thread$\rightarrow$argument} This is the argument, given at
thread creation time, that is passed to the function described in the
previous field. See also the type definition at line 2.
\item \txt{x\_thread$\rightarrow$id} The identification number of a
thread. This is a unique, 16 bits number in \oswald.
\item \txt{x\_thread$\rightarrow$a\_prio} The assigned priority of a
thread.
\item \txt{x\_thread$\rightarrow$c\_prio} The current priority of a
thread. Note that during some operations, namely mutex and monitor
operations, the threads priority can be boosted to fight priority inversion.
This is the threads priority at any moment in \oswald. When the threads
priority has been boosted, after the operation, this field will be reset to
the \txt{a\_prio} field value.
\item \txt{x\_thread$\rightarrow$c\_quantums} Threads execute for a
number of time slices, before they hand over the processor to a thread at
the same priority, that is ready to run. This field holds the number of
slices or quantums that the thread still has left.
\item \txt{x\_thread$\rightarrow$a\_quantums} This is the number of
quantums that the \txt{c\_quantums} field will be 'reloaded' with when
the thread has exhausted its
number of quantums. It is fixed now, but could become an argument that can
be modulated later.
\item \txt{x\_thread$\rightarrow$state} The state that the thread is in;
more information on thread states can be found below.
\item \txt{x\_thread$\rightarrow$xref} A cross reference pointer that can
be used to point back to user supplied structures. It is not used by any
Owald function.
\item \txt{x\_thread$\rightarrow$flags} The flags that this thread has.
The flags can carry \oswald information and user information. Flags are
described more in detail below.
\item \txt{x\_thread$\rightarrow$next} Linked list field. See also next
field.
\item \txt{x\_thread$\rightarrow$previous} Threads are kept in a doubly
linked circular list; each priority control block has two lists, one that
keeps the ready to run threads and one that keeps the pending threads; 
this and the previous field form this list.
\item \txt{x\_thread$\rightarrow$sticks} When a thread is pending,
waiting for an event to happen or just sleeping, this field holds the number
of ticks that it still has to go before being woken up.
\item \txt{x\_thread$\rightarrow$wakeup} At wakeup time, \oswald saves the
current time in this field; between the time of wakeup and the time
that the thread starts executing again, time could have elapsed allready. We
need to compensate this time in the event routines that take a timing
argument.
\item \txt{x\_thread$\rightarrow$snext} The pending threads are kept in a
singly linked list, that is woven through this field.
\item \txt{x\_thread$\rightarrow$action} When a thread is pending and
becomes alive again because the \txt{sticks} field became 0, this is the
function pointer that is called to execute whatever householding stuff that
needs done, determined by the pending state the thread was in.
\item \txt{x\_thread$\rightarrow$l\_waiting} When threads are waiting on
monitors, this field is used to form the singly linked list of threads that
are waiting on the same monitor. Note that a thread can only be waiting on a
single monitor at a time.
\item \txt{x\_thread$\rightarrow$waiting\_for} When the thread is in a
waiting list of a monitor, the monitor it is waiting for is given in this
field.
\item \txt{x\_thread$\rightarrow$l\_competing} A thread can be competing
for an event (mutex, monitor, queue, i.e. any of the events); the list of
threads that are competing for the event is competing formed by this field. Again note that a thread
can only be competing for a single event.
\item \txt{x\_thread$\rightarrow$competing\_for} The event that a thread
is competing for.
\item \txt{x\_thread$\rightarrow$l\_owned} Events like mutexes and
monitors are 'owned' by a thread, when they are 'locked'. Threads can own
multiple events at the same time. The linked list of events that are owned
by a thread, starts at this field, and is further formed by the
\txt{event$\rightarrow$l\_owned} field.
\item \txt{x\_thread$\rightarrow$m\_count} When a thread waits on a
monitor, the number of times it has locked or entered the monitor, is saved
in this field, so that it can be restored later, when the thread becomes
owner again of the monitor.
\end{itemize}

\subsubsection{Thread States}

The different states that a thread can be in are summarised in table
\ref{table:thread_states}.

A thread is only ready to be scheduled and to run, when its state is not 0.
Any other state in table \ref{table:thread_states} indicates that the thread
is waiting for a certain event to happen and is not ready to run.

Note that the thread states that are not 0 (ready) and that indicate that
the thread is waiting for a certain event to happen, correspond with the
numerical value of the event type. I.e. the event type indicator of a mutex
event has value 1, for a semaphore event the value is 4 and so forth.


\footnotesize
\begin{longtable}{||l|c|p{9cm}||}
\hline
\hfill \textbf{State Name} \hfill\null & \textbf{Value} & \textbf{Meaning}  \\ 
\hline
\endhead
\hline
\endfoot
\endlastfoot
\hline


% \begin{table}[!ht]
%   \begin{center}
%    \begin{tabular}{||>{\footnotesize}l<{\normalsize}|>{\footnotesize}c<{\normalsize}|>{\footnotesize}c<{\normalsize}||} \hline
%     \textbf{State Name} & \textbf{Value} & \textbf{Meaning} \\ \hline

\txt{xt\_ready} & 0 &
\begin{minipage}[t]{8.5cm}
This is the state when a thread is ready to run, or is
running. In \oswald, there is no special state for the currently running
thread. 
\end{minipage} \\

\txt{xt\_mutex} & 1 &
\begin{minipage}[t]{8.5cm}
The state of a thread that is waiting for a mutex to become
available for locking.
\end{minipage} \\

\txt{xt\_queue} & 2 &
\begin{minipage}[t]{8.5cm}
The state of a thread waiting for elements to be posted on an empty queue or
waiting for space to become available in a full queue.
\end{minipage} \\

\txt{xt\_mailbox} & 3 &
\begin{minipage}[t]{8.5cm}
The state of a thread waiting for data to become available in a mailbox or
for the mailbox to become empty so that a message can be posted.
\end{minipage} \\

\txt{xt\_semaphore} & 4 &
\begin{minipage}[t]{8.5cm}
The state of a thread waiting for a semaphores count to become greater than
0.
\end{minipage} \\

\txt{xt\_signals} & 5 &
\begin{minipage}[t]{8.5cm}
The state of a thread waiting on a set of signals to satisfy the get
request.
\end{minipage} \\

\txt{xt\_monitor} & 6 &
\begin{minipage}[t]{8.5cm}
The state of a thread that is trying to enter a monitor.
\end{minipage} \\

\txt{xt\_block} & 7 &
\begin{minipage}[t]{8.5cm}
The state of a thread that is waiting for a free block to become available
in a block pool.
\end{minipage} \\

\txt{xt\_map} & 8 &
\begin{minipage}[t]{8.5cm}
The state of a thread that is waiting for a bit to become 0 (reset) in an event
bitmap.
\end{minipage} \\

\txt{xt\_waiting} & 9 &
\begin{minipage}[t]{8.5cm}
The state of a thread that is waiting on a monitor.
\end{minipage} \\

\txt{xt\_suspended} & 10 &
\begin{minipage}[t]{8.5cm}
The state of a thread that is suspended.
\end{minipage} \\

\txt{xt\_sleeping} & 11 &
\begin{minipage}[t]{8.5cm}
The state of a thread that is sleeping after calling the
\txt{x\_thread\_sleep} function.
\end{minipage} \\

\txt{xt\_ended} & 13 &
\begin{minipage}[t]{8.5cm}
The state of a thread that returned from the start function.
\end{minipage} \\



\hline 
\multicolumn{3}{c}{} \\
\caption{Thread states}
\label{table:thread_states}
\end{longtable}
\normalsize


%     \hline
%     \end{tabular}
%     \caption{Thread states}
%     \label{table:thread_states}
%   \end{center}
% \end{table}

\subsubsection{Creating a Thread}

A thread is created by means of the call:

\txt{x\_status x\_thread\_create(x\_thread thread, x\_entry entry, void
*argument, x\_ubyte *stack, x\_size size, x\_size prio, x\_flags flags);}

The arguments to this call are:
\begin{enumerate}
\item \txt{thread}, the thread structure pointer. This structure will contain the internal state of a thread.

\item \txt{entry}, the pointer to a function that the thread will start with. The type definition \txt{x\_entry}
is \txt{typedef void (*x\_entry)(void * argument)}, so \txt{x\_entry}
is a function pointer of a function that returns void and takes a void
pointer as its single argument. The thread stops and is put an an
\txt{xt\_ended} state when this function returns.

\item \txt{argument}, the argument that will be passed to the entry
function. It's use is defined by the programmer. \oswald will not use this
argument for any reason and will just pass it along to the entry function.

\item \txt{stack}, points to a suitably sized block of memory that will
be used as stack space by the thread.

\item \txt{size}, indicates the number of bytes that the \txt{stack}
argument has. Note that the size allowed must be larger or equal than
\txt{MIN\_STACK\_SIZE}.

\item \txt{prio}, is the priority of the thread. Real time priorities start from 1
and go up to 63. Soft priorities (round robin threads) start from 64 and go
up to 128. The lower the value of this \txt{prio} argument, the higher
the priority of the thread.

\item \txt{flags}, this argument can have only 2 different values:
\begin{itemize}
\item \txt{TF\_SUSPENDED}, indicates that the thread should not start
immediately but is put in a suspended state. The thread must explicitely be
started with an \txt{x\_thread\_resume} call.

\item \txt{TF\_START} to indicate that the thread should start immediately,
i.e. not in a suspended state.
\end{itemize}

\end{enumerate}

The possible return values for this call are sumarized in table
\ref{table:x_thread_create}.

\footnotesize
\begin{longtable}{||l|p{9cm}||}
\hline
\hfill \textbf{Return Value} \hfill\null & \textbf{Meaning}  \\ 
\hline
\endhead
\hline
\endfoot
\endlastfoot
\hline


% \begin{table}[!ht]
%   \begin{center}
%     \begin{tabular}{||>{\footnotesize}l<{\normalsize}|>{\footnotesize}c<{\normalsize}||} \hline
%     \textbf{Return Value} & \textbf{Meaning} \\ \hline

\txt{xs\_success} &

\begin{minipage}[t]{9cm}
The thread was succesfully created.
\end{minipage} \\

\txt{xs\_bad\_argument} &

\begin{minipage}[t]{9cm}
Some argument to the thread create was bad, e.g. the stack size was less
than \txt{MIN\_STACK\_SIZE}, or the entry function was NULL or the flags argument
was not one of the allowed values.
\end{minipage} \\


\hline 
\multicolumn{2}{c}{} \\
\caption{Return Status for \txt{x\_thread\_create}}
\label{table:x_thread_create}
\end{longtable}
\normalsize


%     \hline
%     \end{tabular}
%     \caption{Return Status for \txt{x\_thread\_create}}
%     \label{table:x_thread_create}
%   \end{center}
% \end{table}

\subsubsection{Suspending a Thread}

There are two different calls available to suspend a thread:

\txt{x\_status x\_thread\_suspend(x\_thread thread);} \\
\txt{x\_status x\_thread\_suspend\_cb(x\_thread thread, x\_ecb cb, void *arg);}

Both functions will suspend a thread but the second call implements a
callback facility which makes it potentially much safer to suspend a thread.

The thread that is to be suspended is passed as \txt{thread} argument. A
thread can call this function to suspend itself.

A thread that is in a suspended state can be deleted with the
\txt{x\_thread\_delete} call or resumed with the
\txt{x\_thread\_resume} function.

It is potentially unsafe to suspend a thread when the caller doesn't know
exactly in what state a thread is. Suppose that thread A is just performing
a memory allocation with the \txt{x\_mem\_alloc} call. The memory package
that implements this call will lock a mutex while manipulating internal
memory structures. The owner of this mutex is the thread that is performing
the call, in this example thread A. If thread A was to be suspended when
owning this lock, no other thread in the entire system could perform a
memory allocation call or release call, potentially bringing the whole
system to a virtual standstill.

To give the programmer the possibility to remedy these kind of situations,
the callback system is available. The \txt{x\_ecb} declares the following
type of function.

\txt{typedef x\_boolean (*x\_ecb)(x\_event event, void *argument);}

It is a function pointer, which function takes as arguments an event pointer
and an argument pointer and returns a boolean.

When a thread is suspended with the callback variant, the callback function
is called for each event, a mutex or a monitor, that the thread being
suspended, is owner of. The \txt{argument} field of the
\txt{x\_thread\_suspend\_cb} function, is passed on as the second argument
to the callback function and is programmer defined.

In the callback function, any actions can be undertaken under control of the
programmer. She could decide to check wether the owned event is a mutex or
monitor and call the relevant release function for it. When the return value
of the callback, for any invocation, was not \txt{true}, the
thread will \textbf{not} be suspended and the return status of the
\txt{x\_thread\_suspend\_cb} call will be \txt{xs\_owner} to indicate
that the thread being suspended owned at least one event.

If all the invocations of the callback returned the \txt{false} value.
The thread will be suspended, regardless wether it still owned events or
not. If the thread still owned events, the return status will also be
\txt{xs\_owner}. When the thread didn't own any events anymore, it
returns \txt{xs\_success}.

For a full overview of the return values of either suspend call, refer to
table \ref{table:x_thread_suspend}.

\footnotesize
\begin{longtable}{||l|p{9cm}||}
\hline
\hfill \textbf{Return Value} \hfill\null & \textbf{Meaning}  \\ 
\hline
\endhead
\hline
\endfoot
\endlastfoot
\hline


% \begin{table}[!ht]
%   \begin{center}
%     \begin{tabular}{||>{\footnotesize}l<{\normalsize}|>{\footnotesize}c<{\normalsize}||} \hline
%     \textbf{Return Value} & \textbf{Meaning} \\ \hline

\txt{xs\_success} &

\begin{minipage}[t]{9cm}
The thread was succesfully suspended and doesn't own any events anymore.
\end{minipage} \\

\txt{xs\_no\_instance} &

\begin{minipage}[t]{9cm}
The thread being suspended was either in a \txt{xt\_ended} state or a
\txt{xt\_suspended} state allready.
\end{minipage} \\

\txt{xs\_owner} &

\begin{minipage}[t]{9cm}
This indicates that the thread still owns events. Wether the thread was
really suspended depends on the context. When the suspension was tried with
the non callback variant, the thread is suspended.

When the suspension was performed with the callback variant, it depends on
the context. If the callback returned at least a single \txt{false}
value. The thread will not be suspended. If the callback always returned
\txt{true} as value, the thread will be suspended, but still ows events.
I.e. the callback did not succeed in releasing the owned events.
\end{minipage} \\


\hline 
\multicolumn{2}{c}{} \\
\caption{Return Status for \txt{x\_thread\_suspend}}
\label{table:x_thread_suspend}
\end{longtable}
\normalsize

%     \hline
%     \end{tabular}
%     \caption{Return Status for \txt{x\_thread\_suspend}}
%     \label{table:x_thread_suspend}
%   \end{center}
% \end{table}

\subsubsection{Resuming a Thread}

A thread can be resumed with the call:

\txt{x\_status x\_thread\_resume(x\_thread thread);}

The \txt{thread} argument to the call is the thread that needs resuming.
The return values of this call are simple and summarised in table
\ref{table:x_thread_resume}.

\footnotesize
\begin{longtable}{||l|p{9cm}||}
\hline
\hfill \textbf{Return Value} \hfill\null & \textbf{Meaning}  \\ 
\hline
\endhead
\hline
\endfoot
\endlastfoot
\hline


% \begin{table}[h]
%   \begin{center}
%     \begin{tabular}{||>{\footnotesize}l<{\normalsize}|>{\footnotesize}c<{\normalsize}||} \hline
%     \textbf{Return Value} & \textbf{Meaning} \\ \hline

\txt{xs\_success} &

\begin{minipage}[t]{9cm}
The thread was succesfully resumed.
\end{minipage} \\

\txt{xs\_no\_instance} &

\begin{minipage}[t]{9cm}
The thread being resumed is not in a suspended state, i.e. there was no
preceeding \txt{x\_thread\_suspend} call for this thread.
\end{minipage} \\


\hline 
\multicolumn{2}{c}{} \\
\caption{Return Status for \txt{x\_thread\_resume}}
\label{table:x_thread_resume}
\end{longtable}
\normalsize

%     \hline
%     \end{tabular}
%     \caption{Return Status for \txt{x\_thread\_resume}}
%     \label{table:x_thread_resume}
%   \end{center}
% \end{table}

\subsubsection{Making a Thread Sleep \& Wake up}

A thread can go to sleep for a number a certain number of ticks with the
following call:

\txt{void x\_thread\_sleep(x\_sleep ticks);}

Note that a \txt{ticks} argument of 0 has no effect. A thread can be made
to sleep for an eternal amount of time by given the \txt{ticks} argument
the value of \txt{x\_eternal}. 

A thread that has been put to sleep, for an eternal time or a limited time,
can be woken up with the following call:

\txt{x\_status x\_thread\_wakeup(x\_thread thread);}

When the thread was not sleeping, the status returned is
\txt{xs\_no\_instance}, otherwise, \txt{xs\_success} is returned.

\subsubsection{Deleting a Thread}

A thread can only be deleted when its state is either \txt{xt\_ended}
because if returned from the entry function, or when it's state is
\txt{xt\_suspended}, because of one of the \txt{x\_thread\_suspend}
calls.

\subsubsection{Changing Thread Priority}

A threads priority can be changed with the call:

\txt{x\_status x\_thread\_priority\_set(x\_thread thread, x\_size newprio);}

This function will return \txt{xs\_success} when the thread did change
priority or \txt{xs\_bad\_argument} if the \txt{newprio} argument was
out of bounds.

\subsubsection{Identifying the Current Thread}

The \txt{x\_thread} pointer of the currently running thread can be
found with the following call:

\txt{x\_thread x\_thread\_current(void);}

The 16 bits unique ID of a thread is found in the
\txt{x\_thread$\rightarrow$id} field of the \txt{x\_thread} structure.

Note that this ID is unique and is recycled. This means that the thread ID
does not convey any information about the start order of threads; e.g. a
thread with ID 128 could have been created after the thread with ID 130.







% ------------------------------------------------------------------------+
% Copyright (c) 2001 by Punch Telematix. All rights reserved.             |
%                                                                         |
% Redistribution and use in source and binary forms, with or without      |
% modification, are permitted provided that the following conditions      |
% are met:                                                                |
% 1. Redistributions of source code must retain the above copyright       |
%    notice, this list of conditions and the following disclaimer.        |
% 2. Redistributions in binary form must reproduce the above copyright    |
%    notice, this list of conditions and the following disclaimer in the  |
%    documentation and/or other materials provided with the distribution. |
% 3. Neither the name of Punch Telematix nor the names of other           |
%    contributors may be used to endorse or promote products derived      |
%    from this software without specific prior written permission.        |
%                                                                         |
% THIS SOFTWARE IS PROVIDED ``AS IS'' AND ANY EXPRESS OR IMPLIED          |
% WARRANTIES, INCLUDING, BUT NOT LIMITED TO, THE IMPLIED WARRANTIES OF    |
% MERCHANTABILITY AND FITNESS FOR A PARTICULAR PURPOSE ARE DISCLAIMED.    |
% IN NO EVENT SHALL PUNCH TELEMATIX OR OTHER CONTRIBUTORS BE LIABLE       |
% FOR ANY DIRECT, INDIRECT, INCIDENTAL, SPECIAL, EXEMPLARY, OR            |
% CONSEQUENTIAL DAMAGES (INCLUDING, BUT NOT LIMITED TO, PROCUREMENT OF    |
% SUBSTITUTE GOODS OR SERVICES; LOSS OF USE, DATA, OR PROFITS; OR         |
% BUSINESS INTERRUPTION) HOWEVER CAUSED AND ON ANY THEORY OF LIABILITY,   |
% WHETHER IN CONTRACT, STRICT LIABILITY, OR TORT (INCLUDING NEGLIGENCE    |
% OR OTHERWISE) ARISING IN ANY WAY OUT OF THE USE OF THIS SOFTWARE, EVEN  |
% IF ADVISED OF THE POSSIBILITY OF SUCH DAMAGE.                           |
% ------------------------------------------------------------------------+

%
% $Id: timer.tex,v 1.1.1.1 2004/07/12 14:07:44 cvs Exp $
%

\subsection{Timers}

\subsubsection{Operation}

\subsubsection{Timer Structure Definition}

The structure definition of a timer is as follows:

\bcode
\begin{verbatim}
 1: typedef struct x_Timer * x_timer;
 2: typedef void (*x_timerfire)(x_timer timer);
 3:
 4: typedef struct x_Timer {
 5:   volatile x_timer previous;
 6:   volatile x_timer next;
 7:   volatile x_timer list;
 8:   w_size id;
 9:   x_timerfire timerfire;
10:   void * argument;
11:   x_sleep initial;
12:   x_sleep repeat;
13:   x_sleep delta;
14:   w_size fired;
15:   x_flags flags;
16: } x_Timer;
\end{verbatim}
\ecode

The relevant fields in the timer structure are the following:

\begin{itemize}
\item \txt{x\_timer$\rightarrow$previous}
\item \txt{x\_timer$\rightarrow$next}
\item \txt{x\_timer$\rightarrow$list}
\item \txt{x\_timer$\rightarrow$id}
\item \txt{x\_timer$\rightarrow$timerfire}
\item \txt{x\_timer$\rightarrow$argument}
\item \txt{x\_timer$\rightarrow$initial}
\item \txt{x\_timer$\rightarrow$repeat}
\item \txt{x\_timer$\rightarrow$delta}
\item \txt{x\_timer$\rightarrow$fired}
\item \txt{x\_timer$\rightarrow$flags}
\end{itemize}

\subsubsection{Creating a Timer}

\subsubsection{Stopping a Timer}

\subsubsection{Resuming a Timer}

\subsubsection{Deleting a Timer}

\subsubsection{Changing a Timer}




%
% $Id: atomic.tex,v 1.1.1.1 2004/07/12 14:07:44 cvs Exp $
%

\subsection{Atomic Operations}

\subsubsection{Principle}

Atomic operations are called 'atomic' because in performing an operation,
they can not be interrupted. This is the short explanation, now up to
something with more meat on it...

Suppose we have the following C code fragment
in a multi threaded environment\footnote{We use a label and a goto to show
the 2 offending lines right after another. Please don't use this example to
judge the C code quality of \oswald...}:

\bcode
\begin{verbatim}
 1: int fencepost = 0;      // fencepost is a thread global variable

 2: ...

 3:  retry:
 4:  if (fencepost == 0) {  // Check if 'fencepost' is free
 5:                         <---- Place of thread switch between A and B
 6:    fencepost = 1;       // Indicate that we are the owner of the fencepost
 7:  }
 8:  else {
 9:    x_thread_sleep(10);
10:    goto retry;
11:  }

12:  ... do some 'critical' stuff ...

13:  fencepost = 0;         // Leave the critical region

14: ...

\end{verbatim}
\ecode


This is a piece of code that would give, at first sight, a safe feeling to a
programmer that at any time, only a single thread can execute the 'critical' stuff
section shown in the example; it looks safe all-right; check that no other
thread is in the critical section by reading the \textbf{fencepost} variable and if
it is 0, we set it to 1 so that other threads have to wait and take a nap
until the \textbf{fencepost} reads 0.

This assumption is wrong! Suppose thread A is performing line 4 and has read
the value of the \textbf{fencepost} variable into a register. The value of
the \textbf{fencepost} variable is actually 0 and thus the register value is
also 0. Now at that moment, the scheduler stops this thread, saving all
registers and makes another thread B ready to run the same code fragment. This
thread B reaches line 4, again reads in \textbf{fencepost} and sees it is 0,
then proceeds to execute line 6 and writes a value of 1 in the
\textbf{fencepost} global variable, assuming it has access to the critical
section. This thread B is again stopped by the
scheduler and thread A is again made to run, thread A was stopped at line 5
and thus it writes, in its turn, a 1 into the \textbf{fencepost} variable,
and proceeds into the critical section. In
other words, both thread A and B now assume they are the only thread that is
executing the 'critical' section.

In other words, the above construct is wrong, because the operation of
\textbf{reading} a variable for checking and \textbf{writing} a new value back,
are not indivisible operations. For this sample of code to work, the lines,
4 and 6 should be executed without any thread scheduling or other
interruption occurring. This nature of operations being indivisible is called
\textbf{atomicity} in computer science. Just like an atom can not be
split\footnote{At least programmers can't, physicist and nuclear powerplants do it all the time...},
an atomic operation can not be split or interrupted.

This kind of atomic operations are supported by \oswald with help of the
underlying CPU that offers special atomic instructions for implementing atomic
constructs.

The constructs that \oswald offers for atomic operations are:

\begin{itemize}
\item \txt{void x\_atomic\_swap(void * address1, void * address2);}

A function that atomically swaps the contents of \txt{address1} and
\txt{address2}.

\item \txt{x\_boolean x\_enter\_atomic(void * address, void * contents);}

A function that tries to enter an atomic region or mutually exclusive
region and reports back whether it succeeded or the caller should try again.
When successful, the contents at \txt{address} will be the same as \txt{contents}.

\item \txt{x\_boolean x\_exit\_atomic(void * address, void * contents);}

A function that tries to exit an atomic region and reports back whether it
was successful or not. When successful, the contents at \txt{address}
will be 0 or NULL.
\end{itemize}

By means of these simple functions, other more powerful atomic operations
can be build, e.g. compare and swap, mutexes, semaphores, monitors... Note
that the \oswald mutexes, semaphores and monitors don't use these atomic
constructs.

\subsubsection{Atomic Swap}

\oswald offers a function: 

\txt{void x\_atomic\_swap(void * address1, void * address2);}

This function will swap the contents pointed to by \txt{address1} and
\txt{address2} in an atomic way. It is equivalent to executing the
following C code fragment.

\bcode
\begin{verbatim}
1: void x_atomic_swap(void * address1, void * address2) {

2:  unsigned int tmp;

3:  tmp = * (unsigned int *)address1;
4:  * (unsigned int *)address1 = * (unsigned int *)address2;
5:  * (unsigned int *)address2 = tmp;

6: }
\end{verbatim}
\ecode

But with a guarantee that the reading and writing operations that occur in
lines 3 to 5 are never interrupted. I.e. they happen atomically.

Note that the atomicity that is offered by this call, is only 32 bits wide
(the size of an int). When 64 bit or other constructs need atomic reading or
writing, a special function should be written.

\subsubsection{Atomic Regions}

Atomic regions are implemented by means of the enter and exit functions.
Both calls go hand in hand and are best illustrated by means of an example.

\bcode
\begin{verbatim}
 1:
 2: x_thread owner; // Owner of the critical region, a thread global variable.
 3:
 4: void atomic_region(void) {
 5:
 6:   while (! x_enter_atomic(&owner, x_thread_current())) {
 7:     x_thread_sleep(1);
 8:   }
 9:
10:   ... The critical region, the owner thread is recorded in 'owner' ...
11:
12:   while (! x_exit_atomic(&owner)) {
13:     x_thread_sleep(1);
14:   }
15:
16: }
\end{verbatim}
\ecode

The \txt{x\_enter\_atomic} call will try to place the current thread
pointer into the variable \txt{owner}. It can fail for two reasons:

\begin{enumerate}
\item The contents of the \txt{owner} variable are not NULL since another
thread is in the critical region. The \txt{x\_enter\_atomic} call will
return false.
\item The contents of the \txt{owner} variable contain a special value,
that indicate that another thread has atomically swapped\footnote{In fact,
the \txt{x\_atomic\_swap} functionality is used for this.} in the contents of
the \txt{owner} variable to examine its contents. The thread that is
examining the \txt{owner} variable has put a 'special' value in the
\txt{owner} variable to indicate that he is looking into the contents of
the \txt{owner} variable. This special variable has the value of 1. The
\txt{x\_enter\_atomic} will also return false in this case.
\end{enumerate}

Whatever the reason for failure, the calling thread has to wait or sleep a little
bit so that the other thread that is examining the variable or has locked the
critical region, is done with the job.

When the \txt{x\_enter\_atomic} succeeds with a return value of 'true',
it means that the \textbf{previous} contents of \txt{owner} were NULL and that it
\textbf{now} contains the current thread
pointer and as such the current thread can safely enter the critical region.

When the thread wants to leave the critical section, it calls the
\txt{x\_exit\_atomic} function so that the contents of \txt{owner} are
again set to NULL, so that other threads can try to get into the critical
region.

It is important to note that although at line 12, the current thread owns
the critical region, the \txt{x\_exit\_atomic} call can still fail since
another thread could have put the special 'looking value' of 1 into the
\txt{owner} variable, in which case the current owning thread has to wait
a little until the other thread has stopped examining the \txt{owner}
variable and replaced the special looking value again with the current
thread pointer.

Another equally important remark is that although the second argument to
\txt{x\_enter\_atomic} is a void pointer, it doesn't need to be a
pointer. It can contain whatever information is usefull for the programmer.
The call itself does not use this argument. So it could as well contain a
counter or a combination of a thread identifier and the number of times a
thread has entered the critical region; this way, powerfull ad lightweight constructs
can be made like e.g. monitors and such. There are only 2 values that should
\textbf{NOT} be used as values for the second argument; the special 0 or
NULL value and the special 'looking value' 1.

Atomic operations are called leigthweight since they don't set or reset
interrupt flags. In \oswald they are especially leightweight since they are
implemented as inline assembly calls and don't have function calling
overhead\footnote{In the \oswald implementation for x86, the
\txt{x\_atomic\_swap} implementation is only 3 opcodes large, while the
\txt{x\_enter\_atomic} operation is 62 bytes large. The
\txt{x\_exit\_atomic} is even smaller than the enter function. They are
all implemented in the \fl{cpu.h} header file.}.
















%
% $Id: modules.tex,v 1.1.1.1 2004/07/12 14:07:44 cvs Exp $
%

\subsection{Modules}

\subsubsection{Operation}

Modules are object code, generated by a C
compiler. By means of the support for modules in \oswald, they can be loaded when 
the kernel is already running. The module system thus enables the programmer 
to add functionality to the kernel at runtime.

This offers the kernel programmer a powerful tool:

\begin{itemize}
\item It allows to add new functionality, unknown at the moment the kernel
was started up, to be inserted without stopping it.
\item It allows to customize the behavior of the kernel at a later stage.
\item It allows to upgrade certain behavior or capabilities, without
having to statically link the code for this behavior in at the time the
kernel was compiled and linked.
\item It allows to remove certain functionality from the kernel when it is
no longer needed, to save on space and potentially CPU usage.
\end{itemize}

Since modules are in fact object files, generated by a compiler and
optionally 'prelinked' by a linker, the module code of \oswald must perform
some extra steps to make 'it run', steps which are normally performed
by the linker.

These steps can be enumerated as:

\begin{enumerate}
\item Loading the module; this means translating the format of the compiler
product object code (\oswald only supports the ELF format) into the internal
format that \oswald can work with further. This step also performs some
checks on the offered ELF data to see whether it complies with the ELF
standard.

This step of loading ELF data into the internal format is performed by calling the \txt{x\_module\_load}
function.

\item Resolving the module; this means that the names of the symbols that are either exported from
the module or imported are bound to the correct address.
This step of resolving symbols is performed by calling the \txt{x\_module\_resolve}
function.

\item The relocation step; the compiler generates code without knowing at
what memory address the code will end up in the running kernel. It therefore
generates code assuming the starting address is \txt{0x00000000} and will
produce information that will indicate at which places in the code, the real
address must be 'patched'. This information consists of 2 parts:
\begin{enumerate}
\item A symbol that describes the address that needs to be patched into this
location, i.e. the \textbf{what} goes into the patch.
\item A relocation record that describes \textbf{how} and \textbf{where} this symbol address needs
to be patched into the code.
\end{enumerate}
This process of patching is called relocation and is performed by calling the \txt{x\_module\_relocate}
function.

\item Initialization; each module must export a function that is called by
the kernel, to initialize the internal module data structures. 
Only when this step has been run successfully, the module is ready to be
used by other parts of the kernel or by other modules.
\end{enumerate}

Is is possible to change the value of a symbol after step 3, so
that module startup parametrization can be performed. To achieve this, one can
use the \txt{x\_module\_search} function to find a symbol and change its
value as needed. For initial parametrization, this must be done before step
4 of initialization.

\textbf{Sample Code}

The following is a code sample that illustrates how a module can be loaded,
resolved, relocated, initialized and how an exported function can be
searched for and called.

Note that the specifics of how the ELF data is read in from a file or from
memory is beyond the scope of the sample.

\bcode
\begin{verbatim}
  x_Module Module;
  x_module module = &Module;
  x_status status;
  int (*function)(int, int); 
  x_ubyte * elf_data;

  /*
  ** The variable 'function' is a function pointer; this 
  ** function takes two int's as arguments and returns an
  ** int as result. We assume that the variable 'elf_data' 
  ** points to an in memory image of an ELF object file.
  */

  status = x_module_load(module, elf_data, malloc, free);
  if (status != xs_success) {
    printf("Loading failed %d\n", status);
    // handle failure
  }

  status = x_module_resolve(module);
  if (status != xs_success) {
    printf("Resolving failed %d\n", status);
    // handle failure
  }

  status = x_module_relocate(module);
  if (status != xs_success) {
    printf("Relocation failed %d\n", status);
    // handle failure
  }
    
  status = x_module_init(module);
  if (status != xs_success) {
    printf("Initialization failed %d\n", status);
    // handle failure
  }

  /*
  ** Suppose that the module is known to export a function 
  ** with the name 'do_add' that adds two integers and 
  ** returns the result.
  */

  status = x_module_search(module, "do_add", &function);
  if (status == xs_success) {  
    printf("Address of 'do_add' is %p\n", function);
    printf("do_add(10, 20) = %d\n", (*function)(10, 20));
  }
  else {
    printf("Not found %d\n", status);
  }

  ...
\end{verbatim}
\ecode

\subsubsection{Brief Explanation on ELF, Resolution and Relocation}

This section will explain some elements on the ELF format that can not be
found back in the documentation or will clarify some structures. An overview
of a very simple ELF object file and how it is transformed into a memory
image is shown in figure \ref{fig:elf}.

The best way to study the ELF format is reading the documentation a few
times\footnote{I know it is hard to read, but it is the real thing...} and
using the 'objdump' command of the GNU toolset. Learn to use this command,
it's your friend.

\begin{figure}[!ht]
  \begin{center}
    \includegraphics[width=\textwidth]{elf}
    \caption{ELF Structures and the In Memory Image.\label{fig:elf}}
 \end{center}
\end{figure}

In figure \ref{fig:elf} only a few sections relevant for the following
explanation are shown, more section types exists. For more information, we
refer to the available ELF documentation. 

The following explanation is looking very repetitive but this is for the
sake of clarity. For somebody that is experienced in the art of ELF, it
will look childish\footnote{Eric, this means you shouldn't read further...}.

The very simple ELF object code of figure \ref{fig:elf} has 2 sections that
need to be translated to memory, i.e. the \txt{.text} section that
contains executable code and the \txt{.data} section that contains data
that is referred to from the executable code in the \txt{.text} section. The
assembly code that is represented in figure \ref{fig:elf} is the following small
but very useful function\footnote{Yes, I know that there is no newline in
the format string.}.

\bcode
\begin{verbatim}
void myfun(void) {
  printf("this is a string");
}
\end{verbatim}
\ecode

In the assembly code of the \txt{.text} segment, we see that the address
of \txt{printf} is not yet filled in, since the address was not known
at compile time. Also the address of the string \txt{"this is a string"}
is not yet known at compile time. The \txt{.data} section contains the
characters that make up the string.

It is important to note that the placeholder for the string address
(relocation 1) contains the 32 bit word \txt{0x00000000}, while the
placeholder for the \txt{print} call (relocation 2), contains the word
\txt{0xfffffffc} in little endian format. Written as a signed integer,
this second placeholder contains the decimal value $-4$. 

The contents of these placeholders are values that will be used in the
relocation process. In ELF speak, they are called an \textbf{addend}. In the ELF
documentation, the addend is indicated by the character \textbf{'A'}. The
addend is a value that is added during the relocation algorithm to the end
result. Note that the placeholder value $-4$ for the call is to compensate
for the fact that the program counter on a x86 architecture is pointing four
bytes further than the current instruction, so if a program counter relative value needs to be
patched into the \txt{.text} section, we need to compensate for this
program counter offset; therefore the $-4$ addend value.

The section address in the ELF image of an object file always starts at the
value \txt{0x00000000}
since the compiler does not know where in memory the image will go. In the
module code, the contents of the \txt{.text} and \txt{.data} will be
copied over into an allocated memory area. In this specific example, the
\txt{.text} section starts at \txt{0x4740} and the \txt{.data}
section immediately follows at address \txt{0x4758}. Note that some bytes
in the \txt{.text} memory area are wasted because of alignment
considerations. These alignment requirements are also given in the ELF
structure. Please refer to the TIS document for more information.

The next step, after allocating memory and copying over the relevant section
contents to that memory, is resolving the addresses of the symbols used in
the ELF data. In this specific example, there is 1 imported symbol, namely
\txt{printf}. The symbol
number 2, that refers to \txt{printf} has an undefined section and an
undefined type. The ELF standard does not care what type an undefined
element is, it could be an object or a function pointer. The module code
needs to resolve all these undefined symbols and attach them to the correct
address, i.e. it has to fill in the proper value for the undefined symbols.
These undefined symbols are searched for in the available symbol tables. If
such a symbol can not be resolved, the module can not be used. In this
specific example, the symbol value for \txt{printf} has been resolved
into the address \txt{0x00004f00}.

In the resolution step, the values for the remaining symbols also needs to
be filled in. In our example, this means adding the value of the symbol to
the in memory address of the section the symbol refers to. For symbol 1,
this means adding the value to the in memory address of the \txt{.data}
section and for symbol 3, adding the value to the \txt{.text} in memory address.
The address value of a symbol, is indicated by the character \textbf{'S'} in
the ELF standard.

From the binding information, we can also deduce that the symbol for
\txt{myfun} has binding 'global' and is thus exported from the module. It
is therefore a very good idea to make all possible definitions in the source
code static, such that only the required functions are exported (and
valuable memory space will be saved in the symbol tables).

After the resolution step, the relocation takes place. The example shows 2
relocation records. All relocation records from a .rel section, refer to the same
section that needs patching; in this example, there is only one section that
needs relocation and therefore there is only a single \txt{.rel} section.
In more complex ELF files, there can be a multitude of relocation sections
and symbol tables. In this specific example, the section that needs patching
is the \txt{.text} section. The offset of a relocation record added to
the memory address where the section is, yields the address in memory that
needs relocation, in the ELF text this is indicated by the character
\textbf{'P'}\footnote{Probably to indicate that this address needs
patching.}.

The relocation can now be performed. In our simple example, only 2 different
relocation types or algorithms are required:

\begin{itemize}
\item \txt{R\_386\_32} This is an absolute relocation, given by the
following algorithm:

$*P = A + S$

The first relocation is of this type and as one can see from the mnemonics,
it is used to push the address of the format string on the stack. Therefore
it is an absolute address and the addend is 0.

\item \txt{R\_386\_32} This is a program counter relative relocation,
given by the algorithm:

$*P = A + S - P$

This relocation type is used for the second relocation record and its result
is an argument for the \txt{call} opcode of the x86. Since this
opcode expects a relative and signed offset from the current
program counter, we need to compensate for the fact that the program counter
is already pointing 4 bytes further, therefore the $-4$ addend value or
said otherwise, P contains the value that the program counter has at the \textbf{call} instruction
plus 4 bytes.
\end{itemize}

In the example given by figure \ref{fig:elf}, note that the relocations have
NOT been performed yet in the in memory representation, i.e. the original
addends \txt{0x00000000} and \txt{0xfffffffc} are still in the in memory
data.

Note that relocation types are CPU specific, so the relocation type e.g. 4 for a
X86 CPU is not the same type as the relocation type 4 for an ARM processor!
In more complex cases as this example (e.g. shared object files and position
independent code), many more relocation types exist and sections to relocate
exist, but the principles always remain the same.

\subsubsection{Identifiers, Symbol Tables and Symbols}

An important element in the module code is the handling of symbols and
identifiers. Symbols are associations of a \textbf{name} and a \textbf{memory
address}. As indicated above, symbols play in important role in the
relocation process and to export functionality from the module to other
parts of the kernel or other modules. 

Symbols are combined in tables, called symbol tables. The name of a symbol
is called an \textbf{identifier} in \oswald. How identifiers, symbol tables
and symbols relate to each other is explained in figure \ref{fig:symbols}.

\begin{figure}[!ht]
  \begin{center}
    \rotatebox{270}{\includegraphics[height=0.65\textheight]{symbols}}
    \caption{Relation between identifiers, symbol tables and symbols.\label{fig:symbols}}
 \end{center}
\end{figure}

Figure \ref{fig:symbols} shows how the identifier for the \txt{printf} symbol is
linked into the hashtable and points to all the symbols that have the same
identifier. If the \txt{x\_ident$\rightarrow$symbols} field is traced
through the dashed and dotted line, the first symbol encountered defines the
address of the \txt{printf} function and links further through the
\txt{x\_symbol$\rightarrow$next} field to two other symbols that refer to
this symbol to define the address value of \txt{printf}. This is indicated by
the fact that these two symbols refer back to the first, by means of the
dotted line and their \txt{x\_symbol$\rightarrow$value.defines} fields.
Also note that all the symbols refer to the identifier for \txt{printf}
through their \txt{x\_symbol$\rightarrow$ident} field.

The first symbol table, the one that contains the symbol definition of the
\txt{printf} function, has its \txt{x\_symtab$\rightarrow$module}
field pointing to \txt{NULL} indicating that this symbol table belongs to
the kernel itself and contains kernel exported function addresses.

In the following paragraphs, the structures for identifiers, symbol tables
and symbols will be explained in more detail.

%\newpage
\textbf{Identifier Structure Definition}

The structure definition of an identifier is as follows:

\bcode
\begin{verbatim}
 1: typedef struct x_Ident * x_ident;
 2:
 3: typedef struct x_Ident {
 4:   x_ident next;
 5:   x_symbol symbols;
 6:   x_ubyte string[0];
18: } x_Ident;
\end{verbatim}
\ecode

The relevant fields in the identifier structure are the following:

\begin{itemize}
\item \txt{x\_ident$\rightarrow$next} All identifiers in the module code
are unique. To look up identifiers, they are kept in a hashtable and entries
with the same hash value are 'chained'. This field chains this identifier to
the next identifier with the same hash value.
\item \txt{x\_ident$\rightarrow$symbols} A symbol is defined later on.
The list of symbols that have the same identifier starts with this field.
\item \txt{x\_ident$\rightarrow$string} The identifier structure is a
variable sized structure and the end of the structure is used to store the
ASCII characters that make up the string representation of the name.
\end{itemize}

\textbf{Symbol Table Structure Definition}

The structure definition of a symbol table is as follows:

\bcode
\begin{verbatim}
 1: typedef struct x_Symtab * x_symtab;
 2:
 3: typedef struct x_Symtab {
 4:   x_symtab next;
 5:   x_module module;
 6:   x_int num_symbols;
 7:   x_int capacity;
 8:   x_uword madgic;
 9:   x_Symbol symbols[0];
18: } x_Symtab;
\end{verbatim}
\ecode

The relevant fields in the symtab structure are the following:

\begin{itemize}
\item \txt{x\_symtab$\rightarrow$next} Symbol tables are kept in a linked
list and this field is used to chain further to the next symbol table in the
list.

\item \txt{x\_symtab$\rightarrow$module} Symbol tables are associated
with a module. The module either exports or imports the symbols that are in
its symbol table. There is only one case where this field contains NULL, and
this is when a symbol table is associated with symbols exported by the
kernel. Note that the kernel symbol table only exports symbols.

\item \txt{x\_symtab$\rightarrow$num\_symbols} This field contains the
number of symbols that are available in the symbol table.

\item \txt{x\_symtab$\rightarrow$capacity} Some symbol tables are
allocated with a certain number of free slots, to fill in later. This field
total number of slots that are available in the symbol table. The number of
available slots is found from \txt{capacity} - \txt{num\_symbols}.

\item \txt{x\_symtab$\rightarrow$madgic} There is a function
\txt{x\_symbol2symtab} that can be used to find back the symbol table
from a symbol reference. This field is used as a fencepost to help this
function; it indicates the start of the array of \txt{x\_Symbol} elements
in the \txt{x\_Symtab} structure.

\item \txt{x\_symtab$\rightarrow$symbols} This is the array of
\txt{x\_Symbol} structures that are part of the symbol table.
\end{itemize}

\textbf{Symbol Structure Definition}

The structure definition of a symbol is as follows:

\bcode
\begin{verbatim}
 1: typedef struct x_Symbol * x_symbol;
 2:
 3: typedef struct x_Symbol {
 4:   x_symbol next;
 5:   x_ident ident;
 6:   x_flags flags;
 7:   union {
 8:     x_address address;
 9:     x_symbol defines;
10:   } value;
18: } x_Symbol;
\end{verbatim}
\ecode

The relevant fields in the symbol structure are the following:

\begin{itemize}
\item \txt{x\_symbol$\rightarrow$next} This is the field that chains to
the next symbol with the same identifier.

\item \txt{x\_symbol$\rightarrow$ident} This is the reference back to the
identifier that is the name of the symbol.

\item \txt{x\_symbol$\rightarrow$flags} Symbols can carry different flags
to indicate e.g. which field of the \txt{value} union is valid. Also the
number of referrals to this symbol is stored in the lower bits of the
\txt{flags} field. The number of referrals only applies to symbols that
define an address and indicates how many other symbols rely on this symbol
to define the address.

The most important flags are:
\begin{enumerate}
\item \txt{SYM\_EXPORTED} This symbol defines a memory address, i.e. the
\txt{value.address} field is valid.
\item \txt{SYM\_IMPORTED} The symbol refers to another symbol that defines
the address value, i.e. the \txt{value.defines} field is valid and points
to the defining symbol.
\item \txt{SYM\_RESOLVED} The symbol has been set up properly and is bound
to either and address or to the symbol that defines the address.
\item \txt{SYM\_FUNCTION} The symbol refers to a function pointer.
\item \txt{SYM\_VARIABLE} The symbol refers to a data object.
\item \txt{SYM\_SPECIAL} The symbol refers to a special object or function
pointer like e.g. the initializer function, or other internally used
elements. 
\end{enumerate}

More flags are possible for which we refer to the source code of the module
functionality.

\item \txt{x\_symbol$\rightarrow$value.address} This is the address
location that is associated with the symbol.

\item \txt{x\_symbol$\rightarrow$value.defines} When a symbol refers to
another defining symbol, this field is the reference to the symbol that
defines the address.
\end{itemize}

\subsubsection{Loading a Module}

Loading is the initial step in transforming ELF compliant data (e.g. an ELF object
file) into a module that can be used by \oswald.

A module is initially loaded by means of the following call:

\txt{x\_status x\_module\_load(x\_module module, x\_ubyte * elf\_data, x\_malloc a, x\_free f);}

The different return values that this call can produce are summarized
in table \ref{table:module_load}.

\oswald assumes that the passed pointer \txt{elf\_data} is a single chunk
of memory that contains all the objects file data and considers it read only
data. This enables the programmer that wants to package the kernel to put
this data in read only memory.
The ELF structures that are part of the data and that need to be modified,
will be copied over into runtime modifiable structures.

\footnotesize
\begin{longtable}{||l|p{9cm}||}
\hline
\hfill \textbf{Return Value} \hfill\null & \textbf{Meaning}  \\ 
\hline
\endhead
\hline
\endfoot
\endlastfoot
\hline


% \begin{table}[!ht]
%   \begin{center}
%     \begin{tabular}{||>{\footnotesize}l<{\normalsize}|>{\footnotesize}c<{\normalsize}||} \hline
%     \textbf{Return Value} & \textbf{Meaning} \\ \hline

\txt{xs\_success} &
\begin{minipage}[t]{9cm}
The loading was successful and the subsequent step of resolution can be taken for preparing
the module further. The \txt{MOD\_LOADED} flag is set after successful
return.
\end{minipage} \\

\txt{xs\_no\_mem} &
\begin{minipage}[t]{9cm}
In allocating memory for its internal structures, the function encountered a
NULL reply from the allocation function. Nothing of the potentially already allocated
memory is freed.
\end{minipage} \\

\txt{xs\_not\_elf} &
\begin{minipage}[t]{9cm}
In checking the internal consistency of the ELF file, or in checking the
version of the ELF ABI that this code supports, a discrepancy was found. No
further processing can be done on this module.
\end{minipage} \\


\hline 
\multicolumn{2}{c}{} \\
\caption{Return Status for \txt{x\_module\_load}}
\label{table:module_load}
\end{longtable}
\normalsize


%     \hline
%     \end{tabular}
%     \caption{Return Status for \txt{x\_module\_load}}
%     \label{table:module_load}
%   \end{center}
% \end{table}

\subsubsection{Resolving a Module}

A module can be resolved with the following call:

\txt{x\_status x\_module\_resolve(x\_module module);}

The different return values that this call can produce are summarized
in table \ref{table:module_resolve}.  

\footnotesize
\begin{longtable}{||l|p{9cm}||}
\hline
\hfill \textbf{Return Value} \hfill\null & \textbf{Meaning}  \\ 
\hline
\endhead
\hline
\endfoot
\endlastfoot
\hline

% \begin{table}[!ht]
%   \begin{center}
%     \begin{tabular}{||>{\footnotesize}l<{\normalsize}|>{\footnotesize}c<{\normalsize}||} \hline
%     \textbf{Return Value} & \textbf{Meaning} \\ \hline

\txt{xs\_success} &
\begin{minipage}[t]{9cm}
The module symbols have been successfully resolved and are linked up with
the proper addresses. Calling this function for a module that has been
resolved already also yields this return value. As a result the
\txt{MOD\_RESOLVED} flag is set.
\end{minipage} \\

\txt{xs\_seq\_error} &
\begin{minipage}[t]{9cm}
An error occurred because the module was not yet loaded, i.e. the
\txt{MOD\_LOADED} flag was not set for this module.
\end{minipage} \\

\txt{xs\_no\_symbol} &
\begin{minipage}[t]{9cm}
A symbol that is imported by the module could not be found back in the
existing symbol tables. Maybe additional modules need to be loaded to define
the missing symbol.
\end{minipage} \\


\hline 
\multicolumn{2}{c}{} \\
\caption{Return Status for \txt{x\_module\_resolve}}
\label{table:module_resolve}
\end{longtable}
\normalsize


%     \hline
%     \end{tabular}
%     \caption{Return Status for \txt{x\_module\_resolve}}
%     \label{table:module_resolve}
%   \end{center}
% \end{table}

\subsubsection{Relocating a Module}

A module can be relocated with the following call:

\txt{x\_status x\_module\_relocate(x\_module module);}

The different return values that this call can produce are summarized
in table \ref{table:module_relocate}.  

\footnotesize
\begin{longtable}{||l|p{9cm}||}
\hline
\hfill \textbf{Return Value} \hfill\null & \textbf{Meaning}  \\ 
\hline
\endhead
\hline
\endfoot
\endlastfoot
\hline


% \begin{table}[!ht]
%   \begin{center}
%     \begin{tabular}{||>{\footnotesize}l<{\normalsize}|>{\footnotesize}c<{\normalsize}||} \hline
%     \textbf{Return Value} & \textbf{Meaning} \\ \hline

\txt{xs\_success} &
\begin{minipage}[t]{9cm}
Information on the return value.
\end{minipage} \\


\hline 
\multicolumn{2}{c}{} \\
\caption{Return Status for \txt{x\_module\_relocate}}
\label{table:module_relocate}
\end{longtable}
\normalsize


%     \hline
%     \end{tabular}
%     \caption{Return Status for \txt{x\_module\_relocate}}
%     \label{table:module_relocate}
%   \end{center}
% \end{table}

\subsubsection{Initializing a Module}

A module can be initialized with the following call:

\txt{x\_status x\_module\_initialize(x\_module module);}

The different return values that this call can produce are summarized
in table \ref{table:module_initialize}.  


\footnotesize
\begin{longtable}{||l|p{9cm}||}
\hline
\hfill \textbf{Return Value} \hfill\null & \textbf{Meaning}  \\ 
\hline
\endhead
\hline
\endfoot
\endlastfoot
\hline


% \begin{table}[!ht]
%   \begin{center}
%     \begin{tabular}{||>{\footnotesize}l<{\normalsize}|>{\footnotesize}c<{\normalsize}||} \hline
%     \textbf{Return Value} & \textbf{Meaning} \\ \hline

\txt{xs\_success} &
\begin{minipage}[t]{9cm}
Information on the return value.
\end{minipage} \\



\hline 
\multicolumn{2}{c}{} \\
\caption{Return Status for \txt{x\_module\_initialize}}
\label{table:module_initialize}
\end{longtable}
\normalsize

%     \hline
%     \end{tabular}
%     \caption{Return Status for \txt{x\_module\_initialize}}
%     \label{table:module_initialize}
%   \end{center}
% \end{table}

\subsubsection{Searching for an Exported Symbol}

A certain function address can be searched for in a module with the following call:

\txt{x\_status x\_module\_search(x\_module module, const char * name, void ** address);}

The different return values that this call can produce are summarized
in table \ref{table:module_search}.  

\footnotesize
\begin{longtable}{||l|p{9cm}||}
\hline
\hfill \textbf{Return Value} \hfill\null & \textbf{Meaning}  \\ 
\hline
\endhead
\hline
\endfoot
\endlastfoot
\hline


% \begin{table}[!ht]
%   \begin{center}
%     \begin{tabular}{||>{\footnotesize}l<{\normalsize}|>{\footnotesize}c<{\normalsize}||} \hline
%     \textbf{Return Value} & \textbf{Meaning} \\ \hline

\txt{xs\_success} &
\begin{minipage}[t]{9cm}
Information on the return value.
\end{minipage} \\


\hline 
\multicolumn{2}{c}{} \\
\caption{Return Status for \txt{x\_module\_search}}
\label{table:module_search}
\end{longtable}
\normalsize

%     \hline
%     \end{tabular}
%     \caption{Return Status for \txt{x\_module\_search}}
%     \label{table:module_search}
%   \end{center}
% \end{table}

\subsubsection{Parsing a Function Identifier for JNI}

The module code of \oswald offers a utility function that can parse the
function identifier into the appropriate components for interfacing with a
JNI system.

The JNI utility 'javah' will read Java class files that contain declarations
of native functions and will generate the appropriate header file and
signatures to be used in C code. Since Java has the capabilities of
overloading methods and C has not, there is a translation step taking from
the Java naming methods into an C style naming.

For instance, the overloaded Java class methods:

\txt{My\_Peer.destroy(char b[], int i, String s)}

\txt{My\_Peer.destroy(long j, char b[], int i, String s)} 

is translated into function names that are acceptable in C:

\txt{Java\_My\_1Peer\_destroy\_\_\_3BILjava\_lang\_String\_2}

\txt{Java\_My\_1Peer\_destroy\_\_J\_3BILjava\_lang\_String\_2}

I.e. the Java method names are mangled.
The specifics of this mangling are outside of the scope of this
document. Any good book on JNI will go into the details of this name
mangling.

\oswald offers a utility function \txt{x\_symbol\_java} that helps in
translating the C style function names back into the appropriate Java
conventions that can be used inside a virtual machine. This function is:

\txt{x\_int x\_symbol\_java(x\_symbol symbol, unsigned char * buffer, x\_size num);}

The arguments are the \txt{symbol} that needs parsing, a \txt{buffer} and the size of
this buffer in \txt{num}. The function will return the number of
characters that are used in the buffer. This function will check for
overflow of the buffer. The buffer is filled with 3 components of the
function name:

\begin{enumerate}
\item The name of the class, in this case 'My\_Peer', as a nul terminated
character string, that begins at \txt{buffer + buffer[0]}. The length of
the class name without the trailing nul character included is stored in
\txt{buffer + buffer[1]}.
\item The name of the method, in this case 'destroy', as a nul terminated
character string, that begins at \txt{buffer + buffer[2]}. The length of
the method name without the trailing nul character included is stored in
\txt{buffer + buffer[3]}.
\item The signature of the arguments, as a nul terminated character string.
For the first example this is '([BILjava/lang/String;)',
and for the second this is '(J[BILjava/lang/String;)'. The string begins at \txt{buffer + buffer[4]}. The length of
the argument signature without the trailing nul character included is stored in
\txt{buffer + buffer[5]}.
\end{enumerate}

So the first 6 positions of the \txt{buffer} array of characters are used
as indexes into this array and the lengths of the strings. This limits the
lengths and indexes to be less than 255.

Also note that the argument signature component of the buffer is written in the
descriptor format of the Java Virtual Machine specification, but without the
return type\footnote{This is no limitation since Java overloaded methods
must all return the same type.}.

When this function is called on a symbol that does not represent a Java
native method function name, i.e. a name that doesn't begin with
'Java\_', the returned result is 0.

%\bibliographystyle{natbib}
%\bibliography{modules}




