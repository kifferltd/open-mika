 
% ENTER PRODUCT NAME, DOCUMENT TYPE, & FILE NAME FOR THE DOCUMENT %
\newcommand{\ProductName}{\oswald $^{\normalsize{\texttrademark}}$} % type in product

\newcommand{\DocumentType}{\Huge{Reference Specification}} % type in document type (e.g. White Paper)

\newcommand{\FileName}{Oswald_REF-SPEC} % without the extension.  be sure there
                                    % are no spaces in between the { } curly
                                    % brackets.... 


% ENTER Document Title & DOCUMENT DESCRIPTION:                    %
\newcommand{\DocumentTitle}{The Real Time Operating Layer for Wonka}

\newcommand{\DocumentDescription}{\textsf{OSwald}, \emph{The Real Time Operating Layer for
Wonka}, was designed with specific goals.  The main goal being to complement
the Wonka Java Virtual Kernel.  The aim of this reference
specification is to examine the specific design goals and their implementation
so that developers may utilize and modify \oswald to their
needs. }


% ENTER THE PRODUCT LOGO and ITS RELATIVE LOCATION FROM THIS DIRECTORY:
% (i.e.               ../../../doc/template/AlogoColor'
%       or simply     ProductLogo ' if it resides in current directory)

\newcommand{\ProductLogo}{../../../doc/template/AlogoColor.jpg}


% DEFINE TEMPLATE DIR & THE VARIOUS GRAPHICS DIRECTORIES         %
% AND THE GRAPHICS EXTENSION...                                  %

\newcommand{\templatepath}{\graphicspath{
                        {./}
                        {../../../doc/template/} 
                                                   }}

\newcommand{\GraphicsExtension}{.pdf}


% ENTER THE AUTHOR OR GROUP RESPONSIBLE FOR MAINTAINING DOCUMENT:
\newcommand{\TheAuthor}{Steven Buytaert}

% ENTER DATE CREATED:  
\newcommand{\datewhen}{April 2001}


% ENTER THE DESIRED HEADER AND FOOTER TEXT:
\newcommand{\HeaderText}{Proprietary}
% (typically    Internal - Confidential' otherwise suggest \acunia\ )

\newcommand{\FooterText}{\DocumentTitle}
% (typically  `Article Title')


