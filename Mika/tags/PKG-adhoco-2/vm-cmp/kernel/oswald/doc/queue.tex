%
% $Id: queue.tex,v 1.1.1.1 2004/07/12 14:07:44 cvs Exp $
%

\subsection{Queues}

\subsubsection{Operation}

\subsubsection{Queue Structure Definition}

The structure definition of a queue is as follows:

\bcode
\begin{verbatim}
 1: typedef struct x_Queue * x_queue;
 2:
 3: typedef struct x_Queue {
 4:   x_Event Event;
 5:   w_word * messages;
 6:   volatile w_word * write;
 7:   volatile w_word * read;
 8:   volatile w_size available;
 9:   w_size capacity;
10:   w_word * limit;
11: } x_Queue;
\end{verbatim}
\ecode

The relevant fields in the queue structure are the following:

\begin{itemize}
\item \txt{x\_queue$\rightarrow$Event} This is the universal event structure that is a field
in all threadable components or elements. It controls the synchronized access
to the queue component and the signalling between threads to indicate changes
in the queue structure.
\item \txt{x\_queue$\rightarrow$messages} The memory area passed at
creation time that will contain the 32 bit messages of the queue. This field
is not changed during the lifetime of the queue.
\item \txt{x\_queue$\rightarrow$write} The current pointer in the message
area, where the next message to send can be written.
\item \txt{x\_queue$\rightarrow$read} The pointer to the message that is
to be read at the next call to \txt{x\_queue\_receive}.
\item \txt{x\_queue$\rightarrow$available} The number of messages that
are waiting in the queue to be read.
\item \txt{x\_queue$\rightarrow$limit} A convenience pointer that points
1 word beyond the memory area given by the \txt{messages} field. It is
used in the wrap around checks for \txt{read} and \txt{write}.
\end{itemize}

\subsubsection{Creating a Queue}

\txt{x\_status x\_queue\_create(x\_queue queue, void * space, w\_size capacity);}

A queue is created and initialized; \txt{queue} is a reference to a queue structure,
\txt{space} is a pointer to suitable sized buffer, i.e. \txt{space} must point to a storage space of
\txt{capacity} 32 bit words.

\subsubsection{Deleting a Queue}

The following call will delete a queue:

\txt{x\_status x\_queue\_delete(x\_queue queue);}

This call will try to delete a queue that is referred to by the \txt{queue} argument. 
Any possible remaining messages are discarded and any waiting threads are notified from
this fact by means of the \txt{xs\_deleted} status.

The memory of the queue structure itself is not released in any way. The memory that was passed
as the \txt{space} argument at queue creation time is not released. It is up to the application to release
these resources.

\subsubsection{Sending Data over a Queue}

Data is send to a queue by means of the following call:

\txt{x\_status x\_queue\_send(x\_queue queue, void * data, x\_sleep timeout);}

This call will try to send \txt{data} over a queue that is referred to by the
\txt{queue} argument. If there was space available in the queue, during the
\txt{timeout} window, the value of \txt{data}, \textbf{not the data
pointed to it,} will be put into the queue. If no space was available within the
\txt{timeout} window, the status \txt{xs\_no\_instance} is returned. 

Data is copied into the queue in a 'first in - first out' fashion.

If within the \txt{timeout} window, no space became available in the queue, the returned status will
be \txt{xs\_no\_instance}.

If a \txt{timeout} value is given from within an interrupt handler or timer handler that was not
\txt{x\_no\_wait}, the status \txt{xs\_bad\_context} is returned.

If this call resulted in the thread waiting for
space in the queue to come available, and the queue was deleted during the
\txt{timeout} value, the returned status will be \txt{xs\_deleted}.

The different return values that this call can produce are summarized in
table \ref{table:rs_queue_send}.


\begin{longtable}{||l|p{7cm}||}
\hline
\hfill \textbf{Return Value} \hfill\null & \textbf{Meaning} \\ 
\hline
\endhead
\hline
\endfoot
\endlastfoot
\hline




% \begin{table}[!ht]
%   \begin{center}
%     \begin{tabular}{||>{\footnotesize}l<{\normalsize}|>{\footnotesize}c<{\normalsize}||} \hline
%     \textbf{Return Value} & \textbf{Meaning} \\ \hline

\txt{xs\_success} & The call succeeded and the message has been delivered into the queue. \\

\txt{xs\_no\_instance} & There became no message slot available in the queue during the timeout window. \\

\txt{xs\_bad\_context} & A \txt{timeout} argument other than \txt{x\_no\_wait} was given from within a timer or interrupt handler context. \\

\txt{xs\_deleted} & The queue structure has been deleted by another thread during the call. \\

\txt{xs\_bad\_element} & The passed reference \txt{queue} doesn't refer to a valid queue structure. \\

%     \hline
%     \end{tabular}
%     \caption{Return Status for \txt{x\_queue\_send}}
%     \label{table:rs_queue_send}
%   \end{center}
% \end{table}

\hline 
\multicolumn{2}{c}{} \\
\caption{Return Status for \txt{x\_queue\_send}}
\label{table:rs_queue_send}
\end{longtable}
\normalsize

\subsubsection{Receiving Data from a Queue}

Data is received from a queue by means of the following call:

\txt{x\_status x\_queue\_receive(x\_queue queue, void * * data, x\_sleep timeout);}

This call will try to receive data from a queue that is referred to by the
\txt{queue} argument. If there was any data available in the queue, during the
\txt{timeout} window, this data will
be copied into the variable pointed to by \txt{data}. If no data was available within the
\txt{timeout} window, the status \txt{xs\_no\_instance} is returned. 

Data is returned from the queue in a 'first in - first out' fashion.

If a \txt{timeout} value is given from within an interrupt handler or timer handler that was not
\txt{x\_no\_wait}, the status \txt{xs\_bad\_context} is returned.

If this call resulted in the thread waiting for
data in the queue to come available, and the queue was deleted during the
\txt{timeout} value, the returned status will be \txt{xs\_deleted}.

The different return values that this call can produce are summarized in
table \ref{table:rs_queue_receive}.

\begin{longtable}{||l|p{7cm}||}
\hline
\hfill \textbf{Return Value} \hfill\null & \textbf{Meaning} \\ 
\hline 
\endhead
\hline
\endfoot
\endlastfoot
\hline


% \begin{table}[!ht]
%   \begin{center}
%     \begin{tabular}{||>{\footnotesize}l<{\normalsize}|>{\footnotesize}c<{\normalsize}||} \hline
%     \textbf{Return Value} & \textbf{Meaning} \\ \hline

\txt{xs\_success} & The call succeeded and data contains the message retrieved from the queue. \\

\txt{xs\_no\_instance} & There became no message available in the queue during the timeout window. \\

\txt{xs\_bad\_context} & A \txt{timeout} argument other than \txt{x\_no\_wait} was given from within a timer or interrupt handler context. \\

\txt{xs\_deleted} & The queue structure has been deleted by another thread during the call.  \\

\txt{xs\_bad\_element} & The passed reference \txt{queue} doesn't refer to a queue structure. \\


\hline 
\multicolumn{2}{c}{} \\
\caption{Return Status for \txt{x\_queue\_receive}}
\label{table:rs_queue_receive}
\end{longtable}
\normalsize




%     \hline
%     \end{tabular}
%     \caption{Return Status for \txt{x\_queue\_receive}}
%     \label{table:rs_queue_receive}
%   \end{center}
% \end{table}








