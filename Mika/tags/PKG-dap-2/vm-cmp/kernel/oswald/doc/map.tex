%
% $Id: map.tex,v 1.1.1.1 2004/07/12 14:07:44 cvs Exp $
%

\subsection{Bitmaps}

\subsubsection{Operation}

Bitmaps can be used to store a binary state. A bitmap can contain, at a
certain position, called an 'index' a state symbolized by a 0 (reset state)
or a 1 (set state). Bitmaps can be used for whatever functionality where a
programmer wants to store a 0 or 1 state in a very space conserving way.

In \oswald, bitmaps are always a multiple of a 32 bit rounded number, since
the bits are stored on a word basis. This means that the smallest bitmap that
can be created is 32 bits wide or 1 word wide.

The positions in a bit map start out at 0 and go up to the number of
positions as given in the respective create calls, minus 1. To say it
simple, when a map is created with 64 positions, the indexes start at 0 and
go up to and including position 63.

In \oswald, there are 2 flavors of bitmaps, that also have slightly
different semantics (different meaning for the application interface
functions).

\begin{itemize}
\item Non event bitmaps; these bitmaps structures don't allow synchronized
access. Two threads can be changing, setting or resetting, the same bit at
the same time, and the outcome of operations that are happening virtually at
the same time is undefined. These bitmaps can be used to assign unique ids
or can be used by a single thread at the time.
\item Event or synchronized bitmaps; these bitmaps do allow several threads
to operate on them at the same time and \oswald will make sure that the
access is controlled.
\end{itemize}

We refer to the different functions outlined below to make the difference in
meaning clear. The non synchronized bitmaps are structures of the type
\txt{x\_Umap}, for 'unsynchronized map', while the event like bitmaps are
structures of type \txt{x\_Map}. References to these structure are
indicated by the \txt{x\_umap} and \txt{x\_map} types respectively.
The structure of both types of maps is given below in the structure
definition.

The \txt{x\_map} event bitmaps are implemented by means of the
\txt{x\_umap} calls, so that no code is duplicated in \oswald. The
\txt{x\_Map} type is also implemented by including the \txt{x\_Umap} as
shown below in the structure definition.

\subsubsection{Umap and Map Structure Definition}

The structure definition of an umap is as follows:

\bcode
\begin{verbatim}
 1: typedef struct x_Umap * x_umap;
 2:
 3: typedef struct x_Umap {
 5:   w_size entries;
 6:   w_word mask;
 7:   volatile w_size cache_0;
 8:   w_word * table;
 9: } x_Umap;
\end{verbatim}
\ecode

The relevant fields in the umap structure are the following:

\begin{itemize}
\item \txt{x\_umap$\rightarrow$entries} The number of bits that this map
contains. Note that this field, which is passed as an argument at the map
creation time, is always rounded up to a multiple of 32.
\item \txt{x\_umap$\rightarrow$mask} This is the mask we should use to
clear out the space required for an index in a 32 bit word. This is usefull
in the case where a 32 bit word is used for both flags and an id. The lower
bits of the flags word can be used to store the index than. The mask can be
used to clear this area in the flags word.
\item \txt{x\_umap$\rightarrow$cache\_0} When looking for the first bit in
the map that is 0, this is the index in the table array of words where the
first 0 bit can be found.
\item \txt{x\_umap$\rightarrow$table} The memory area passed at creation
time where the bits of the map will be stored. It must be able to accommodate
the number of words indicated by \txt{x\_umap$\rightarrow$entries} /
\txt{sizeof(w\_word)};
\end{itemize}

\subsubsection{Map Structure Definition}

The structure definition of a map is as follows:

\bcode
\begin{verbatim}
 1: typedef struct x_Map * x_map;
 2:
 3: typedef struct x_Map {
 4:   x_Event Event;
 5:   x_Umap umap;
 6: } x_Map;
\end{verbatim}
\ecode

The relevant fields in the map structure are the following:

\begin{itemize}
\item \txt{x\_map$\rightarrow$Event} This is the universal event structure that is a field
in all threadable components or elements. It controls the synchronized access
to the map component and the signalling between threads to indicate changes
in the map structure.
\item \txt{x\_map$\rightarrow$Umap} The Umap structure that will be used
to manipulate the bit positions, as defined above.
\end{itemize}

For more information on the two types of bitmaps and the difference between
the two types, please refer to the next section on operation of a bitmap.

\subsubsection{Non Event Maps}

The following operations are defined for non synchronized bitmaps.

\begin{itemize}
\item Creating a map, note that there is no delete function for this kind of
maps.
\item Setting the first available 0 bit to 1. I.e. requesting for a bit to
be set from 0 to 1 at \textbf{any} index that has a 0 bit available.
\item Setting a bit from 0 to 1 at a specific index in the map.
\item Resetting a given 1 bit to 0 again.
\item Probing the status of a bit at a certain position.
\end{itemize}

As already indicated, when several threads have simultaneous access to a
map, the outcome of any of these calls is undefined.

\subsubsection{Creating a Non Event Map}

\txt{x\_size x\_umap\_create(x\_umap umap, w\_size entries, w\_word * table);}

This creates the map, with \txt{entries} number of indexes. The
\txt{entries} number is first rounded up to a size that is a multiple of
32, since maps are handled on a word basis. Therefore, the space pointed to
by the \txt{table} argument, must be able to accommodate the correct,
rounded up number of words.

This function returns the rounded up number of entries that are available in
the bitmap. This value returned is the number of entries that will be used
for internal calculations. E.g., when the call is performed with
\txt{entries} equal to 60, the table will accommodate 64 entries and the
\txt{table} argument must point to a suitably sized space. This
information can also be found in the \txt{x\_umap$\rightarrow$entries}
field of the \txt{x\_Umap} structure.

The possible index numbers start at 0 and go up to and including
\txt{entries} - 1.

\subsubsection{Setting A Non Specific Bit in a Non Event Map}

For requesting that the first available 0 bit will be set, the following
call can be used:

\txt{x\_status x\_umap\_any(x\_umap umap, w\_size * index);}

The different return values that this call can produce are summarized
in table \ref{table:umap_any}.  



\footnotesize
\begin{longtable}{||l|p{9cm}||}
\hline
\hfill \textbf{Return Value} \hfill\null & \textbf{Meaning}  \\ 
\hline
\endhead
\hline
\endfoot
\endlastfoot
\hline

% \begin{table}[!ht]
%   \begin{center}
%     \begin{tabular}{||>{\footnotesize}l<{\normalsize}|>{\footnotesize}c<{\normalsize}||} \hline
%     \textbf{Return Value} & \textbf{Meaning} \\ \hline

\txt{xs\_success} &
\begin{minipage}[t]{9cm}
The call succeeded and the pointer indicated by \txt{index} is set to the
bit index that has been set to 1.
\end{minipage} \\

\txt{xs\_no\_instance} &

\begin{minipage}[t]{9cm}
There was no single 0 bit available in the map. The value pointed to by
\txt{index} has not been changed.
\end{minipage} \\


\hline 
\multicolumn{2}{c}{} \\
\caption{Return Status for \txt{x\_umap\_any}}
\label{table:umap_any}
\end{longtable}
\normalsize



%     \hline
%     \end{tabular}
%     \caption{Return Status for \txt{x\_umap\_any}}
%     \label{table:umap_any}
%   \end{center}
% \end{table}

\subsubsection{Setting a Specific Bit in a Non Event Map}

The function call to set a bit to 1 at a specific index in the table is
given by:

\txt{x\_boolean x\_umap\_set(x\_umap umap, w\_size index);}

This function returns a boolean \txt{true} or \txt{false} value; it
returns \txt{true} when the bit at \txt{index} was changed from a 0 to
a 1 position. It returns \txt{false}, when the bit position at
\txt{index} already was set to 1, i.e. did not change.

Note that when the index requested to be set, exceeds the number of entries
in the map, the return value will always be \txt{false} and the bitmap
remains unchanged.

\subsubsection{Resetting a Bit in a Non Event Map}

A specific bit can be reset to 0, by means of the following function call:

\txt{x\_status x\_umap\_reset(x\_umap umap, w\_size index);}

The different return values that this call can produce are summarized
in table \ref{table:umap_reset}.  



\footnotesize
\begin{longtable}{||l|p{9cm}||}
\hline
\hfill \textbf{Return Value} \hfill\null & \textbf{Meaning}  \\ 
\hline
\endhead
\hline
\endfoot
\endlastfoot
\hline



% \begin{table}[!ht]
%   \begin{center}
%     \begin{tabular}{||>{\footnotesize}l<{\normalsize}|>{\footnotesize}c<{\normalsize}||} \hline
%     \textbf{Return Value} & \textbf{Meaning} \\ \hline

\txt{xs\_success} &
\begin{minipage}[t]{9cm}
The call succeeded and the bit at \txt{index} has been successfully reset
from 1 to 0.
\end{minipage} \\

\txt{xs\_no\_instance} &

\begin{minipage}[t]{9cm}
The \txt{index} was beyond the number of entries of the bitmap or the bit
at position \txt{index} was already 0, so no change happened.
\end{minipage} \\


\hline 
\multicolumn{2}{c}{} \\
\caption{Return Status for \txt{x\_umap\_reset}}
\label{table:umap_reset}
\end{longtable}
\normalsize



%     \hline
%     \end{tabular}
%     \caption{Return Status for \txt{x\_umap\_reset}}
%     \label{table:umap_reset}
%   \end{center}
% \end{table}

\subsubsection{Probing a Bit in a Non Event Map}

The value of a bit in a non event map can be determined by means of the
following function call:

\txt{x\_boolean x\_umap\_probe(x\_umap umap, w\_size index);}

The function returns the boolean value \txt{true} when the bit at
position \txt{index} is set to 1. It will return the value \txt{value}
when the bit at position \txt{index} is reset to 0 or when the value of
\txt{index} is beyond the number available positions in the bit map.

\subsubsection{Event Bitmaps}

Event bitmaps can be used to communicate between more than 1 thread. The
event mechanisms of \oswald will prevent simultaneous modification of the map
in an uncontrolled or undefined way.

Threads can wait until a certain bit is reset from 1 to 0, or until a 0
bit becomes available. 

Note that the structure type of an event type map is \txt{x\_Map} and a
pointer to this structure is of the type \txt{x\_map}.

\subsubsection{Creating an Event Map} 

An event bitmap can be created by means of the following call.

\txt{x\_status x\_map\_create(x\_map map, w\_size entries, w\_word * table);}

The arguments passed have the same semantic meaning as with the
\txt{x\_umap\_create} call described above, except the return argument is
different. 

The value of the \txt{entries} is rounded up to the nearest value that is
divisible by 32. The result of this rounding up is not returned as a result,
but can be found back in the field \txt{map$\rightarrow$Umap.entries},
i.e. in the \txt{entries} field of the non synchronized bitmap, that is
one of the fields of the \txt{x\_Map} structure.

\subsubsection{Deleting an Event Map} 

\subsubsection{Setting A Non Specific Bit in an Event Map}

Requesting that the first available 0 bit be set to 1 in an event map, is
established by means of the following function call:

\txt{x\_status x\_map\_any(x\_map map, w\_size * index, x\_window to);}

The \txt{index} points to a an index variable and will be suitably
updated upon successful return of the call. The timeout argument
\txt{to} indicates the window during which the thread will wait for a 0
bit to become available. It can take the normal range of timeout, starting
from \txt{x\_no\_wait} up to the value \txt{x\_eternal} to wait
forever for a bit to become available. 

The different return values that this call can produce are summarized
in table \ref{table:map_any}.  




\footnotesize
\begin{longtable}{||l|p{9cm}||}
\hline
\hfill \textbf{Return Value} \hfill\null & \textbf{Meaning}  \\ 
\hline
\endhead
\hline
\endfoot
\endlastfoot
\hline



% \begin{table}[!ht]
%   \begin{center}
%     \begin{tabular}{||>{\footnotesize}l<{\normalsize}|>{\footnotesize}c<{\normalsize}||} \hline
%     \textbf{Return Value} & \textbf{Meaning} \\ \hline

\txt{xs\_success} &
\begin{minipage}[t]{9cm}
The call succeeded and the bit at the index pointed to by the \txt{index}
argument has been successfully set from 0 to 1.
\end{minipage} \\

\txt{xs\_no\_instance} &

\begin{minipage}[t]{9cm}
The call did not result in a bit being set from 0 to 1 at any position in
the map, within the timeout window given by the \txt{to} argument.
\end{minipage} \\

\txt{xs\_bad\_context} &

\begin{minipage}[t]{9cm}
The timeout argument \txt{to} was not \txt{x\_no\_wait} in the context
of a interrupt handler or timer handler.
\end{minipage} \\

\txt{xs\_deleted} &

\begin{minipage}[t]{9cm}
Another thread has deleted the bitmap when this thread was attempting
the set operation.
\end{minipage} \\


\hline 
\multicolumn{2}{c}{} \\
\caption{Return Status for \txt{x\_map\_any}}
\label{table:map_any}
\end{longtable}
\normalsize


%     \hline
%     \end{tabular}
%     \caption{Return Status for \txt{x\_map\_any}}
%     \label{table:map_any}
%   \end{center}
% \end{table}

\subsubsection{Setting a Specific Bit in an Event Map}

A specific bit in an event bitmap can be set by means of the following call:

\txt{x\_status x\_map\_set(x\_map map, w\_size index, x\_window to);}

The \txt{index} argument indicates the bit position that needs to be set
from 0 to 1. Note that while the \txt{x\_umap\_set} function call
returned a status value indicating whether the bit was flipped from 0 to 1 or
was set to 1 already, this call does not offer this kind of functionality.

In the event version of a bitmap, when the bit at position \txt{index} is
already set to 1, the call will block until the bit becomes set to 0 again,
for as long as indicated by the \txt{to} argument.

This behavior can be compared with the mutex. Please note that this is a
very poor mutex with respect to performance and behavior, i.e. a normal
\txt{x\_mutex} is better w.r.t. performance and priority inversion
behavior; a \txt{x\_mutex} also records which thread owns the mutex and
is thus better suited for some locking functionality.

A map in \oswald also bears some resemblance with the \txt{x\_signals}
structure and semantics. The \txt{x\_signals} has much more powerfull
semantics however as combinations of flags are supported and clearing is
also user controllable. The event bitmaps in \oswald have less possibilities
but offer a bulk bit store with thread synchronization primitives, while
signals are much more flexible, but can only handle 31 bits at a time.

The different return values that this call can produce are summarized
in table \ref{table:map_set}.  



\footnotesize
\begin{longtable}{||l|p{9cm}||}
\hline
\hfill \textbf{Return Value} \hfill\null & \textbf{Meaning}  \\ 
\hline
\endhead
\hline
\endfoot
\endlastfoot
\hline



% \begin{table}[!ht]
%   \begin{center}
%     \begin{tabular}{||>{\footnotesize}l<{\normalsize}|>{\footnotesize}c<{\normalsize}||} \hline
%     \textbf{Return Value} & \textbf{Meaning} \\ \hline

\txt{xs\_success} &
\begin{minipage}[t]{9cm}
The call succeeded and the bit at the index given by the \txt{index}
argument has been successfully set from 0 to 1.
\end{minipage} \\

\txt{xs\_no\_instance} &

\begin{minipage}[t]{9cm}
Within the timeout given by the \txt{to} argument, the bit could not be
set from 0 to 1, or the value of the \txt{index} argument exceeds the
number of bit positions in the bitmap.
\end{minipage} \\

\txt{xs\_bad\_context} &

\begin{minipage}[t]{9cm}
The timeout argument \txt{to} was not \txt{x\_no\_wait} in the context
of a interrupt handler or timer handler.
\end{minipage} \\

\txt{xs\_deleted} &

\begin{minipage}[t]{9cm}
Another thread has deleted the bitmap when this thread was attempting
the set operation.
\end{minipage} \\


\hline 
\multicolumn{2}{c}{} \\
\caption{Return Status for \txt{x\_map\_set}}
\label{table:map_set}
\end{longtable}
\normalsize


%     \hline
%     \end{tabular}
%     \caption{Return Status for \txt{x\_map\_set}}
%     \label{table:map_set}
%   \end{center}
% \end{table}

\subsubsection{Resetting a Bit in an Event Map}

A bit can be reset from the 1 to the 0 state in an event bitmap, by means of
the following call:

\txt{x\_status x\_map\_reset(x\_map map, w\_size index);}

The bit at position \txt{index} will be set from 1 to 0. When the bit did
not have the value 1, i.e. was not set, this call will return
\txt{xs\_no\_instance}. When \txt{index} was going beyond the range of
entries in the bitmap, the status returned is also
\txt{xs\_no\_instance}. When the bit was set from 1 to 0, the status
returned is \txt{xs\_success}.

The different return values that this call can produce are summarized
in table \ref{table:map_reset}.  



\footnotesize
\begin{longtable}{||l|p{9cm}||}
\hline
\hfill \textbf{Return Value} \hfill\null & \textbf{Meaning}  \\ 
\hline
\endhead
\hline
\endfoot
\endlastfoot
\hline



% \begin{table}[!ht]
%   \begin{center}
%     \begin{tabular}{||>{\footnotesize}l<{\normalsize}|>{\footnotesize}c<{\normalsize}||} \hline
%     \textbf{Return Value} & \textbf{Meaning} \\ \hline

\txt{xs\_success} &
\begin{minipage}[t]{9cm}
The call succeeded and the bit at the index given by the \txt{index}
argument has been successfully reset from 1 to 0.
\end{minipage} \\

\txt{xs\_no\_instance} &

\begin{minipage}[t]{9cm}
The bit at position \txt{index} did not have value 1, or the
\txt{index} argument was beyond the maximum number of entries allowed in
this map.
\end{minipage} \\

\txt{xs\_deleted} &

\begin{minipage}[t]{9cm}
Another thread has deleted the bitmap when this thread was attempting
the reset operation.
\end{minipage} \\


\hline 
\multicolumn{2}{c}{} \\
\caption{Return Status for \txt{x\_map\_reset}}
\label{table:map_reset}
\end{longtable}
\normalsize


%     \hline
%     \end{tabular}
%     \caption{Return Status for \txt{x\_map\_reset}}
%     \label{table:map_reset}
%   \end{center}
% \end{table}

\subsubsection{Probing a Bit in an Event Map} 

The status of a bit in an event bitmap can be probed by means of the
following call:

\txt{x\_status x\_map\_probe(x\_map map, w\_size index, x\_boolean * value);}

Upon successfull completion of this call, the argument pointed to by
\txt{value} will contain the boolean value of \txt{true} or
\txt{false} if the bit at position \txt{index} was set to 1 or set to
0, respectively.

The different return values that this call can produce are summarized
in table \ref{table:map_probe}.  



\footnotesize
\begin{longtable}{||l|p{9cm}||}
\hline
\hfill \textbf{Return Value} \hfill\null & \textbf{Meaning}  \\ 
\hline
\endhead
\hline
\endfoot
\endlastfoot
\hline


% \begin{table}[!ht]
%   \begin{center}
%     \begin{tabular}{||>{\footnotesize}l<{\normalsize}|>{\footnotesize}c<{\normalsize}||} \hline
%     \textbf{Return Value} & \textbf{Meaning} \\ \hline

\txt{xs\_success} &
\begin{minipage}[t]{9cm}
The call succeeded and the value of the bit at the index given by the \txt{index}
argument has been written in the argument pointed to by \txt{value}.
\end{minipage} \\

\txt{xs\_no\_instance} &

\begin{minipage}[t]{9cm}
The \txt{index} argument was beyond the maximum number of entries allowed in
this map.
\end{minipage} \\

\txt{xs\_deleted} &

\begin{minipage}[t]{9cm}
Another thread has deleted the bitmap when this thread was attempting
the probe operation.
\end{minipage} \\


\hline 
\multicolumn{2}{c}{} \\
\caption{Return Status for \txt{x\_map\_probe}}
\label{table:map_probe}
\end{longtable}
\normalsize


%     \hline
%     \end{tabular}
%     \caption{Return Status for \txt{x\_map\_probe}}
%     \label{table:map_probe}
%   \end{center}
% \end{table}

















