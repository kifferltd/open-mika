% ------------------------------------------------------------------------+
% Copyright (c) 2001 by Punch Telematix. All rights reserved.             |
%                                                                         |
% Redistribution and use in source and binary forms, with or without      |
% modification, are permitted provided that the following conditions      |
% are met:                                                                |
% 1. Redistributions of source code must retain the above copyright       |
%    notice, this list of conditions and the following disclaimer.        |
% 2. Redistributions in binary form must reproduce the above copyright    |
%    notice, this list of conditions and the following disclaimer in the  |
%    documentation and/or other materials provided with the distribution. |
% 3. Neither the name of Punch Telematix nor the names of other           |
%    contributors may be used to endorse or promote products derived      |
%    from this software without specific prior written permission.        |
%                                                                         |
% THIS SOFTWARE IS PROVIDED ``AS IS'' AND ANY EXPRESS OR IMPLIED          |
% WARRANTIES, INCLUDING, BUT NOT LIMITED TO, THE IMPLIED WARRANTIES OF    |
% MERCHANTABILITY AND FITNESS FOR A PARTICULAR PURPOSE ARE DISCLAIMED.    |
% IN NO EVENT SHALL PUNCH TELEMATIX OR OTHER CONTRIBUTORS BE LIABLE       |
% FOR ANY DIRECT, INDIRECT, INCIDENTAL, SPECIAL, EXEMPLARY, OR            |
% CONSEQUENTIAL DAMAGES (INCLUDING, BUT NOT LIMITED TO, PROCUREMENT OF    |
% SUBSTITUTE GOODS OR SERVICES; LOSS OF USE, DATA, OR PROFITS; OR         |
% BUSINESS INTERRUPTION) HOWEVER CAUSED AND ON ANY THEORY OF LIABILITY,   |
% WHETHER IN CONTRACT, STRICT LIABILITY, OR TORT (INCLUDING NEGLIGENCE    |
% OR OTHERWISE) ARISING IN ANY WAY OUT OF THE USE OF THIS SOFTWARE, EVEN  |
% IF ADVISED OF THE POSSIBILITY OF SUCH DAMAGE.                           |
% ------------------------------------------------------------------------+

\chapter{Introduction}

% \begin{minipage}[t]{13cm}
% \begin{quote}
% {\em Uncle Oswald is, if you remember, the greatest rogue, bounder, connoisseur,
% bon vivant and fornicator of all time. Here, many famous names are mentioned
% and there is obviously a grave risk that families and friends are going to
% take offense...}

% \hspace{4cm}{\bf Roald Dahl, "My Uncle Oswald"}
% \end{quote}
% \end{minipage}

\textbf{Please note that this documentation is still very much a work in
progress. If you are looking for a project, dive in and write...}

\section{Why Another Operating System}

There are several good reasons for making our own operating system:

\begin{itemize}
\item Porting the OS to another processor is relatively easy since we know
all the insides and out-sides. Only 3 assembly functions are required:
\begin{enumerate}
\item \txt{x\_init\_stack}: an assembly function that sets up the stack of
a thread that is being created.
\item \txt{x\_thread\_switch}: an assembly function that will switch
between different threads; the current thread stack and status is saved and the thread
that should run next is reinstated.
\item \txt{x\_start\_init}: an assembly function that will start off the
initial thread of the kernel. All other threads are started by means of a
call to \txt{x\_thread\_switch}.
\end{enumerate}

\item Porting the OS to another processor is done by our own company; this
means that the porting effort, which is an element of the critical path of
development is in our own hands.
\item There are no external royalties to be payed to the OS vendor.
\item Implementing special features like the combination of real time
threads with round robin threads is relatively easy.
\item Integration with our hardware is almost perfect. Extra features that
need hardware support and vice versa can be implemented without the need to
rely on an outside party.
\item Code bloat can be kept to a minimum and is under our own control.
\item And probably the most important element is that Oswald is tailor fit for
Wonka, the Java Virtual Kernel. Any feature that is required by Wonka, can
be implemented in Oswald; awe full, isn't it...
\end{itemize}

There are also some things to say against writing yet another operating
system:

\begin{itemize}
\item Writing a solid real time kernel is a daunting task.
\item Porting to another processor requires extra internal resources.
\item And so forth
\end{itemize}
