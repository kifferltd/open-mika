% ------------------------------------------------------------------------+
% Copyright (c) 2001 by Punch Telematix. All rights reserved.             |
%                                                                         |
% Redistribution and use in source and binary forms, with or without      |
% modification, are permitted provided that the following conditions      |
% are met:                                                                |
% 1. Redistributions of source code must retain the above copyright       |
%    notice, this list of conditions and the following disclaimer.        |
% 2. Redistributions in binary form must reproduce the above copyright    |
%    notice, this list of conditions and the following disclaimer in the  |
%    documentation and/or other materials provided with the distribution. |
% 3. Neither the name of Punch Telematix nor the names of other           |
%    contributors may be used to endorse or promote products derived      |
%    from this software without specific prior written permission.        |
%                                                                         |
% THIS SOFTWARE IS PROVIDED ``AS IS'' AND ANY EXPRESS OR IMPLIED          |
% WARRANTIES, INCLUDING, BUT NOT LIMITED TO, THE IMPLIED WARRANTIES OF    |
% MERCHANTABILITY AND FITNESS FOR A PARTICULAR PURPOSE ARE DISCLAIMED.    |
% IN NO EVENT SHALL PUNCH TELEMATIX OR OTHER CONTRIBUTORS BE LIABLE       |
% FOR ANY DIRECT, INDIRECT, INCIDENTAL, SPECIAL, EXEMPLARY, OR            |
% CONSEQUENTIAL DAMAGES (INCLUDING, BUT NOT LIMITED TO, PROCUREMENT OF    |
% SUBSTITUTE GOODS OR SERVICES; LOSS OF USE, DATA, OR PROFITS; OR         |
% BUSINESS INTERRUPTION) HOWEVER CAUSED AND ON ANY THEORY OF LIABILITY,   |
% WHETHER IN CONTRACT, STRICT LIABILITY, OR TORT (INCLUDING NEGLIGENCE    |
% OR OTHERWISE) ARISING IN ANY WAY OUT OF THE USE OF THIS SOFTWARE, EVEN  |
% IF ADVISED OF THE POSSIBILITY OF SUCH DAMAGE.                           |
% ------------------------------------------------------------------------+

%
% $Id: atomic.tex,v 1.1.1.1 2004/07/12 14:07:44 cvs Exp $
%

\subsection{Atomic Operations}

\subsubsection{Principle}

Atomic operations are called 'atomic' because in performing an operation,
they can not be interrupted. This is the short explanation, now up to
something with more meat on it...

Suppose we have the following C code fragment
in a multi threaded environment\footnote{We use a label and a goto to show
the 2 offending lines right after another. Please don't use this example to
judge the C code quality of \oswald...}:

\bcode
\begin{verbatim}
 1: int fencepost = 0;      // fencepost is a thread global variable

 2: ...

 3:  retry:
 4:  if (fencepost == 0) {  // Check if 'fencepost' is free
 5:                         <---- Place of thread switch between A and B
 6:    fencepost = 1;       // Indicate that we are the owner of the fencepost
 7:  }
 8:  else {
 9:    x_thread_sleep(10);
10:    goto retry;
11:  }

12:  ... do some 'critical' stuff ...

13:  fencepost = 0;         // Leave the critical region

14: ...

\end{verbatim}
\ecode


This is a piece of code that would give, at first sight, a safe feeling to a
programmer that at any time, only a single thread can execute the 'critical' stuff
section shown in the example; it looks safe all-right; check that no other
thread is in the critical section by reading the \textbf{fencepost} variable and if
it is 0, we set it to 1 so that other threads have to wait and take a nap
until the \textbf{fencepost} reads 0.

This assumption is wrong! Suppose thread A is performing line 4 and has read
the value of the \textbf{fencepost} variable into a register. The value of
the \textbf{fencepost} variable is actually 0 and thus the register value is
also 0. Now at that moment, the scheduler stops this thread, saving all
registers and makes another thread B ready to run the same code fragment. This
thread B reaches line 4, again reads in \textbf{fencepost} and sees it is 0,
then proceeds to execute line 6 and writes a value of 1 in the
\textbf{fencepost} global variable, assuming it has access to the critical
section. This thread B is again stopped by the
scheduler and thread A is again made to run, thread A was stopped at line 5
and thus it writes, in its turn, a 1 into the \textbf{fencepost} variable,
and proceeds into the critical section. In
other words, both thread A and B now assume they are the only thread that is
executing the 'critical' section.

In other words, the above construct is wrong, because the operation of
\textbf{reading} a variable for checking and \textbf{writing} a new value back,
are not indivisible operations. For this sample of code to work, the lines,
4 and 6 should be executed without any thread scheduling or other
interruption occurring. This nature of operations being indivisible is called
\textbf{atomicity} in computer science. Just like an atom can not be
split\footnote{At least programmers can't, physicist and nuclear powerplants do it all the time...},
an atomic operation can not be split or interrupted.

This kind of atomic operations are supported by \oswald with help of the
underlying CPU that offers special atomic instructions for implementing atomic
constructs.

The constructs that \oswald offers for atomic operations are:

\begin{itemize}
\item \txt{void x\_atomic\_swap(void * address1, void * address2);}

A function that atomically swaps the contents of \txt{address1} and
\txt{address2}.

\item \txt{x\_boolean x\_enter\_atomic(void * address, void * contents);}

A function that tries to enter an atomic region or mutually exclusive
region and reports back whether it succeeded or the caller should try again.
When successful, the contents at \txt{address} will be the same as \txt{contents}.

\item \txt{x\_boolean x\_exit\_atomic(void * address, void * contents);}

A function that tries to exit an atomic region and reports back whether it
was successful or not. When successful, the contents at \txt{address}
will be 0 or NULL.
\end{itemize}

By means of these simple functions, other more powerful atomic operations
can be build, e.g. compare and swap, mutexes, semaphores, monitors... Note
that the \oswald mutexes, semaphores and monitors don't use these atomic
constructs.

\subsubsection{Atomic Swap}

\oswald offers a function: 

\txt{void x\_atomic\_swap(void * address1, void * address2);}

This function will swap the contents pointed to by \txt{address1} and
\txt{address2} in an atomic way. It is equivalent to executing the
following C code fragment.

\bcode
\begin{verbatim}
1: void x_atomic_swap(void * address1, void * address2) {

2:  unsigned int tmp;

3:  tmp = * (unsigned int *)address1;
4:  * (unsigned int *)address1 = * (unsigned int *)address2;
5:  * (unsigned int *)address2 = tmp;

6: }
\end{verbatim}
\ecode

But with a guarantee that the reading and writing operations that occur in
lines 3 to 5 are never interrupted. I.e. they happen atomically.

Note that the atomicity that is offered by this call, is only 32 bits wide
(the size of an int). When 64 bit or other constructs need atomic reading or
writing, a special function should be written.

\subsubsection{Atomic Regions}

Atomic regions are implemented by means of the enter and exit functions.
Both calls go hand in hand and are best illustrated by means of an example.

\bcode
\begin{verbatim}
 1:
 2: x_thread owner; // Owner of the critical region, a thread global variable.
 3:
 4: void atomic_region(void) {
 5:
 6:   while (! x_enter_atomic(&owner, x_thread_current())) {
 7:     x_thread_sleep(1);
 8:   }
 9:
10:   ... The critical region, the owner thread is recorded in 'owner' ...
11:
12:   while (! x_exit_atomic(&owner)) {
13:     x_thread_sleep(1);
14:   }
15:
16: }
\end{verbatim}
\ecode

The \txt{x\_enter\_atomic} call will try to place the current thread
pointer into the variable \txt{owner}. It can fail for two reasons:

\begin{enumerate}
\item The contents of the \txt{owner} variable are not NULL since another
thread is in the critical region. The \txt{x\_enter\_atomic} call will
return false.
\item The contents of the \txt{owner} variable contain a special value,
that indicate that another thread has atomically swapped\footnote{In fact,
the \txt{x\_atomic\_swap} functionality is used for this.} in the contents of
the \txt{owner} variable to examine its contents. The thread that is
examining the \txt{owner} variable has put a 'special' value in the
\txt{owner} variable to indicate that he is looking into the contents of
the \txt{owner} variable. This special variable has the value of 1. The
\txt{x\_enter\_atomic} will also return false in this case.
\end{enumerate}

Whatever the reason for failure, the calling thread has to wait or sleep a little
bit so that the other thread that is examining the variable or has locked the
critical region, is done with the job.

When the \txt{x\_enter\_atomic} succeeds with a return value of 'true',
it means that the \textbf{previous} contents of \txt{owner} were NULL and that it
\textbf{now} contains the current thread
pointer and as such the current thread can safely enter the critical region.

When the thread wants to leave the critical section, it calls the
\txt{x\_exit\_atomic} function so that the contents of \txt{owner} are
again set to NULL, so that other threads can try to get into the critical
region.

It is important to note that although at line 12, the current thread owns
the critical region, the \txt{x\_exit\_atomic} call can still fail since
another thread could have put the special 'looking value' of 1 into the
\txt{owner} variable, in which case the current owning thread has to wait
a little until the other thread has stopped examining the \txt{owner}
variable and replaced the special looking value again with the current
thread pointer.

Another equally important remark is that although the second argument to
\txt{x\_enter\_atomic} is a void pointer, it doesn't need to be a
pointer. It can contain whatever information is usefull for the programmer.
The call itself does not use this argument. So it could as well contain a
counter or a combination of a thread identifier and the number of times a
thread has entered the critical region; this way, powerfull ad lightweight constructs
can be made like e.g. monitors and such. There are only 2 values that should
\textbf{NOT} be used as values for the second argument; the special 0 or
NULL value and the special 'looking value' 1.

Atomic operations are called leigthweight since they don't set or reset
interrupt flags. In \oswald they are especially leightweight since they are
implemented as inline assembly calls and don't have function calling
overhead\footnote{In the \oswald implementation for x86, the
\txt{x\_atomic\_swap} implementation is only 3 opcodes large, while the
\txt{x\_enter\_atomic} operation is 62 bytes large. The
\txt{x\_exit\_atomic} is even smaller than the enter function. They are
all implemented in the \fl{cpu.h} header file.}.














