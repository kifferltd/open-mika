% ------------------------------------------------------------------------+
% Copyright (c) 2001 by Punch Telematix. All rights reserved.             |
%                                                                         |
% Redistribution and use in source and binary forms, with or without      |
% modification, are permitted provided that the following conditions      |
% are met:                                                                |
% 1. Redistributions of source code must retain the above copyright       |
%    notice, this list of conditions and the following disclaimer.        |
% 2. Redistributions in binary form must reproduce the above copyright    |
%    notice, this list of conditions and the following disclaimer in the  |
%    documentation and/or other materials provided with the distribution. |
% 3. Neither the name of Punch Telematix nor the names of other           |
%    contributors may be used to endorse or promote products derived      |
%    from this software without specific prior written permission.        |
%                                                                         |
% THIS SOFTWARE IS PROVIDED ``AS IS'' AND ANY EXPRESS OR IMPLIED          |
% WARRANTIES, INCLUDING, BUT NOT LIMITED TO, THE IMPLIED WARRANTIES OF    |
% MERCHANTABILITY AND FITNESS FOR A PARTICULAR PURPOSE ARE DISCLAIMED.    |
% IN NO EVENT SHALL PUNCH TELEMATIX OR OTHER CONTRIBUTORS BE LIABLE       |
% FOR ANY DIRECT, INDIRECT, INCIDENTAL, SPECIAL, EXEMPLARY, OR            |
% CONSEQUENTIAL DAMAGES (INCLUDING, BUT NOT LIMITED TO, PROCUREMENT OF    |
% SUBSTITUTE GOODS OR SERVICES; LOSS OF USE, DATA, OR PROFITS; OR         |
% BUSINESS INTERRUPTION) HOWEVER CAUSED AND ON ANY THEORY OF LIABILITY,   |
% WHETHER IN CONTRACT, STRICT LIABILITY, OR TORT (INCLUDING NEGLIGENCE    |
% OR OTHERWISE) ARISING IN ANY WAY OUT OF THE USE OF THIS SOFTWARE, EVEN  |
% IF ADVISED OF THE POSSIBILITY OF SUCH DAMAGE.                           |
% ------------------------------------------------------------------------+

%
% $Id: sem.tex,v 1.1.1.1 2004/07/12 14:07:44 cvs Exp $
%

\subsection{Semaphores}

\subsubsection{Operation}

\subsubsection{Semaphore Structure Definition}

The structure definition of a semaphore is as follows:

\bcode
\begin{verbatim}
 1: typedef struct x_Sem * x_sem;
 2:
 3: typedef struct x_Sem {
 4:   x_Event Event;
 5:   volatile w_size current;
 6: } x_Sem;
\end{verbatim}
\ecode

The relevant fields in the semaphore structure are the following:

\begin{itemize}
\item \txt{x\_sem$\rightarrow$Event} This is the universal event structure that is a field
in all threadable components or elements. It controls the synchronized access
to the semaphore component and the signalling between threads to indicate changes
in the semaphore structure.
\item \txt{x\_sem$\rightarrow$current} The current count of the
semaphore. Threads asking to get the semaphore, and find this field set to
0, will block as long as this value does not become greater than 0.
\end{itemize}


\subsubsection{Creating a Semaphore}

\txt{x\_status x\_sem\_create(x\_sem sem, w\_size initial);}

A semaphore is created and initialized; \txt{sem} is a reference to a semaphore
structure and will be initialized to a count indicated by \txt{initial}.

\subsubsection{Deleting a Semaphore}

The following call will deleted a semaphore:

\txt{x\_status x\_sem\_delete(x\_sem sem);}

This call will try to delete a semaphore that is referred to by the \txt{sem} argument. 
Any waiting threads are notified from this fact by means of the \txt{xs\_deleted} status.

The memory of the semaphore structure itself is not released in any way.
It is up to the application to release this resource.

\subsubsection{Incrementing a Semaphore}

The count of a semaphore is incremented by means of the following call:

\txt{x\_status x\_sem\_put(x\_sem sem);}

This call will increment the count of the semaphore that is referred to by the
\txt{sem} argument. If the call succeeds, \txt{xs\_success} is returned. Note
that a semaphore can start out at 0 and that subsequent push commands will
increase the semaphore. I.e., there is no mechanism to check that a
semaphore is pushed beyond a certain preset number. Control for this is up
to the programmer using the semaphore.

When any threads are waiting on the semaphore count to become higher than
zero, the thread with the highest priority will be woken up to acquire the
semaphore.

The different return values that this call can produce are summarized in
table \ref{table:rs_sem_put}.

\footnotesize
\begin{longtable}{||l|p{9cm}||}
\hline
\hfill \textbf{Return Value} \hfill\null & \textbf{Meaning} \hfill \\ 
\endhead
\hline
\endfoot
\endlastfoot
\hline

% \begin{table}[!ht]
%   \begin{center}
%     \begin{tabular}{||>{\footnotesize}l<{\normalsize}|>{\footnotesize}c<{\normalsize}||} \hline
%     \textbf{Return Value} & \textbf{Meaning} \\ \hline

\txt{xs\_success} & The call succeeded and the semaphore count has been incremented. \\

\txt{xs\_deleted} & The semaphore structure has been deleted by another thread during the call. \\

\txt{xs\_bad\_element} & The passed reference \txt{sem} does not refer to a semaphore structure. \\



\hline 
\multicolumn{2}{c}{} \\
\caption{Return Status for \txt{x\_sem\_put}}
\label{table:rs_sem_put}
\end{longtable}
\normalsize

%     \hline
%     \end{tabular}
%     \caption{Return Status for \txt{x\_sem\_put}}
%     \label{table:rs_sem_put}
%   \end{center}
% \end{table}

\subsubsection{Decrementing a Semaphore}

The count of a semaphore is decremented, or its resource is acquired, by means of the following call:

\txt{x\_status x\_sem\_get(x\_sem sem, x\_sleep timeout);}

This call will try to decrement the count of the semaphore that is referred to by the
\txt{sem} argument. If the current count is zero, the thread will wait for the
specified amount of \txt{timeout} ticks. If the count became greater than zero, during the
\txt{timeout} window, this call will return \txt{xs\_success}. If the count didn't become
greater than zero within the \txt{timeout} window, the status \txt{xs\_no\_instance} is returned. 

If a \txt{timeout} value is given from within an interrupt handler or timer handler that was not
\txt{x\_no\_wait}, the status \txt{xs\_bad\_context} is returned.

If this call resulted in the thread waiting for the count to become more
than zero, and the semaphore was deleted during the \txt{timeout} value, the
returned status will be \txt{xs\_deleted}.

The different return values that this call can produce are summarized in
table \ref{table:rs_sem_get}.

\begin{table}[!htbp]
  \begin{center}
    \begin{tabular}{||>{\footnotesize}l<{\normalsize}|>{\footnotesize}c<{\normalsize}||} \hline
    \textbf{Return Value} & \textbf{Meaning} \\ \hline

\txt{xs\_success} &

\begin{minipage}[t]{9cm}
The call succeeded and the semaphore count has been decremented.
\end{minipage} \\

\txt{xs\_no\_instance} &

\begin{minipage}[t]{9cm}
The call did not succeed within the given timeout window.
\end{minipage} \\

\txt{xs\_bad\_context} &

\begin{minipage}[t]{9cm}
A timeout option other than \txt{x\_no\_wait} has been given from within a timer
or interrupt handler context.
\end{minipage} \\

\txt{xs\_deleted} &

\begin{minipage}[t]{9cm}
The semaphore structure has been deleted by another thread during the call.
\end{minipage} \\

\txt{xs\_bad\_element} &

\begin{minipage}[t]{9cm}
The passed reference \txt{sem} doesn't refer to a semaphore structure.
\end{minipage} \\

    \hline
    \end{tabular}
    \caption{Return Status for \txt{x\_sem\_get}}
    \label{table:rs_sem_get}
  \end{center}
\end{table}
\normalsize
