% ------------------------------------------------------------------------+
% Copyright (c) 2001 by Punch Telematix. All rights reserved.             |
%                                                                         |
% Redistribution and use in source and binary forms, with or without      |
% modification, are permitted provided that the following conditions      |
% are met:                                                                |
% 1. Redistributions of source code must retain the above copyright       |
%    notice, this list of conditions and the following disclaimer.        |
% 2. Redistributions in binary form must reproduce the above copyright    |
%    notice, this list of conditions and the following disclaimer in the  |
%    documentation and/or other materials provided with the distribution. |
% 3. Neither the name of Punch Telematix nor the names of other           |
%    contributors may be used to endorse or promote products derived      |
%    from this software without specific prior written permission.        |
%                                                                         |
% THIS SOFTWARE IS PROVIDED ``AS IS'' AND ANY EXPRESS OR IMPLIED          |
% WARRANTIES, INCLUDING, BUT NOT LIMITED TO, THE IMPLIED WARRANTIES OF    |
% MERCHANTABILITY AND FITNESS FOR A PARTICULAR PURPOSE ARE DISCLAIMED.    |
% IN NO EVENT SHALL PUNCH TELEMATIX OR OTHER CONTRIBUTORS BE LIABLE       |
% FOR ANY DIRECT, INDIRECT, INCIDENTAL, SPECIAL, EXEMPLARY, OR            |
% CONSEQUENTIAL DAMAGES (INCLUDING, BUT NOT LIMITED TO, PROCUREMENT OF    |
% SUBSTITUTE GOODS OR SERVICES; LOSS OF USE, DATA, OR PROFITS; OR         |
% BUSINESS INTERRUPTION) HOWEVER CAUSED AND ON ANY THEORY OF LIABILITY,   |
% WHETHER IN CONTRACT, STRICT LIABILITY, OR TORT (INCLUDING NEGLIGENCE    |
% OR OTHERWISE) ARISING IN ANY WAY OUT OF THE USE OF THIS SOFTWARE, EVEN  |
% IF ADVISED OF THE POSSIBILITY OF SUCH DAMAGE.                           |
% ------------------------------------------------------------------------+

%
% $Id: booting.tex,v 1.1.1.1 2004/07/12 14:07:44 cvs Exp $
%

\section{The Booting Procedure}

The booting procedure of \oswald is described by means of
table \ref{table:boot_procedure}. 

In this table, the different function calls and the sequence in which they
are called is described. The dot indicates the place or module where these
calls are defined. The \textsf{CPU} and \textsf{Host} both reside in the
\textsf{hal} or {\em hardware abstraction layer} directory of the \oswald source
tree. The \textsf{\oswald} label indicates that the function resides in the
generic source code part of \oswald.


\footnotesize
\begin{longtable}{||l|c|c|c|p{5cm}||}
\hline
\hfill \textbf{Function} \hfill\null & \textbf{CPU} & \textbf{Host} & \textbf{\oswald} & \hfill \textbf{Notes} \hfill\null \\ 
\endhead
\hline
\endfoot
\endlastfoot
\hline

\textsf{x\_setup\_kernel(mem)}  &           &           & $\bullet$ &
The start. The passed memory argument \txt{mem} is static memory.
This argument is passed down, possibly after allocating some memory.
See the code sample in this section on the use of
allocating static memory. \\

\txt{x\_setup\_cpu(mem)}     & $\bullet$ &           &           &
This function can be used to set up any CPU specific householding stuff. \\ 

\txt{x\_setup\_irq(mem)}     &           & $\bullet$ &           &
Set up the host specific IRQ handlers and default handlers. \\

\txt{x\_setup\_host(mem)}    &           & $\bullet$ &           &
Set up the host specific householding, e.g. on a Linux host, this function
sets up the virtual timer that will provide the timer tick to \oswald.
\\

\txt{x\_setup\_pcbs(mem)}    &           &           & $\bullet$ &
Set up the priority control blocks. \\

\txt{x\_setup\_init(mem)}    &           &           & $\bullet$ &
Set up the initial thread. \\

\txt{x\_start\_init(void)}   & $\bullet$ &           &           &
This function will start the init thread set up in the previous phase. This
is the initial thread switch that takes place. The init thread then calls
all the following functions. \\ \hline

\txt{x\_init\_entry(mem)}    &           &           & $\bullet$ &
Set up the initial thread.\\

\txt{x\_init\_drivers(mem)}  &           & $\bullet$ &           &
At this point, the different hardware drivers that are preset in the system,
can be initialized as well as other drivers. \\

\txt{x\_os\_main(mem)}       &           &           &           &
This is a function to be provided by the application programmer. It creates
the elements (threads, events, ...) that are launched after
\txt{x\_os\_main} returns. \\ 
\hline 
\multicolumn{5}{c}{} \\
\caption{The Boot Procedure of \oswald}
\label{table:boot_procedure}
\end{longtable}
\normalsize




% \footnotesize
% % \begin{center}
% \begin{longtable}{|l|c|c|c|p{5cm}|}
% \hline
% %  \begin{center}
% %   \begin{tabular}{||p{3,5cm}|>{\hfill}p{1cm}<{\hfill\null}|>{\hfill}p{1cm}<{\hfill\null}|>{\hfill}p{1,5cm}<{\hfill\null}|p{5cm}||}    \hline 

%     \hfill \textbf{Function} \hfill\null & \textbf{CPU} & \textbf{Host} & \textbf{\oswald} & \hfill \textbf{Notes} \hfill\null \\ \hline

% \textsf{x\_setup\_kernel(mem)}  &           &           & $\bullet$ &
% The start. The passed memory argument \textsf{mem} is static memory.
% This argument is passed down, possibly after allocating some memory.
% See the code sample in this section on the use of
% allocating static memory. \\

% \textsf{x\_setup\_cpu(mem)}     & $\bullet$ &           &           &
% This function can be used to set up any CPU specific householding stuff. \\ 

% \textsf{x\_setup\_irq(mem)}     &           & $\bullet$ &           &
% Set up the host specific IRQ handlers and default handlers. \\

% \textsf{x\_setup\_host(mem)}    &           & $\bullet$ &           &
% Set up the host specific householding, e.g. on a Linux host, this function
% sets up the virtual timer that will provide the timer tick to \oswald.
% \\

% \textsf{x\_setup\_pcbs(mem)}    &           &           & $\bullet$ &
% Set up the priority control blocks. \\

% \textsf{x\_setup\_init(mem)}    &           &           & $\bullet$ &
% Set up the initial thread. \\

% \textsf{x\_start\_init(void)}   & $\bullet$ &           &           &
% This function will start the init thread set up in the previous phase. This
% is the initial thread switch that takes place. The init thread then calls
% all the following functions. \\ \hline

% \textsf{x\_init\_entry(mem)}    &           &           & $\bullet$ &
% Set up the initial thread.\\

% \textsf{x\_init\_drivers(mem)}  &           & $\bullet$ &           &
% At this point, the different hardware drivers that are preset in the system,
% can be initialized as well as other drivers. \\

% \textsf{x\_os\_main(mem)}       &           &           &           &
% This is a function to be provided by the application programmer. It creates
% the elements (threads, events, ...) that are launched after
% \textsf{x\_os\_main} returns. \\ \hline \\
% %    \hline \\
% %    \end{tabular} \\
% %    \vspace{2cm} \\
%     \caption{The Boot Procedure of \oswald}
%     \label{table:boot_procedure}
% % \end{center}
% %\caption{The Boot Procedure of \oswald} \\
% \end{longtable}
% % \end{center}
% \normalsize










% \footnotesize
% \begin{longtable}{c}
% %  \begin{center}
%    \begin{tabular}{||>{\footnotesize}l|>{\footnotesize}c|>{\footnotesize}c|>{\footnotesize}c|>{\footnotesize}c||} \hline
%     \textbf{Function} & \textbf{CPU} & \textbf{Host} & \textbf{\oswald} & \textbf{Notes} \\ \hline

% \textsf{x\_setup\_kernel(mem)}  &           &           & $\bullet$ &
% \begin{minipage}[t]{5cm}
% The start. The passed memory argument \textsf{mem} is static memory.
% This argument is passed down, possibly after allocating some memory.
% See the code sample in this section on the use of
% allocating static memory.
% \end{minipage} \\

% \textsf{x\_setup\_cpu(mem)}     & $\bullet$ &           &           &
% \begin{minipage}[t]{5cm}
% This function can be used to set up any CPU specific householding stuff.
% \end{minipage} \\

% \textsf{x\_setup\_irq(mem)}     &           & $\bullet$ &           &
% \begin{minipage}[t]{5cm}
% Set up the host specific IRQ handlers and default handlers.
% \end{minipage} \\

% \textsf{x\_setup\_host(mem)}    &           & $\bullet$ &           &
% \begin{minipage}[t]{5cm}
% Set up the host specific householding, e.g. on a Linux host, this function
% sets up the virtual timer that will provide the timer tick to \oswald.
% \end{minipage} \\

% \textsf{x\_setup\_pcbs(mem)}    &           &           & $\bullet$ &
% \begin{minipage}[t]{5cm}
% Set up the priority control blocks.
% \end{minipage} \\

% \textsf{x\_setup\_init(mem)}    &           &           & $\bullet$ &
% \begin{minipage}[t]{5cm}
% Set up the initial thread.
% \end{minipage} \\

% \textsf{x\_start\_init(void)}   & $\bullet$ &           &           &
% \begin{minipage}[t]{5cm}
% This function will start the init thread set up in the previous phase. This
% is the initial thread switch that takes place. The init thread then calls
% all the following functions.
% \end{minipage} \\ \hline

% \textsf{x\_init\_entry(mem)}    &           &           & $\bullet$ &
% \begin{minipage}[t]{5cm}
% Set up the initial thread.
% \end{minipage} \\

% \textsf{x\_init\_drivers(mem)}  &           & $\bullet$ &           &
% \begin{minipage}[t]{5cm}
% At this point, the different hardware drivers that are preset in the system,
% can be initialized as well as other drivers.
% \end{minipage} \\

% \textsf{x\_os\_main(mem)}       &           &           &           &
% \begin{minipage}[t]{5cm}
% This is a function to be provided by the application programmer. It creates
% the elements (threads, events, ...) that are launched after
% \textsf{x\_os\_main} returns.
% \end{minipage} \\
%     \hline
%     \end{tabular} \\
%     \caption{The Boot Procedure of \oswald}
%     \label{table:boot_procedure}
% %  \end{center}
% %\caption{The Boot Procedure of \oswald} \\
% \end{longtable}
% \normalsize



Note that the \txt{x\_os\_main} function should be defined by the
application that is going to be run by \oswald. I.e. the programmer that
wants to use \oswald as its driving software, needs to define this function.

All functions take a \txt{w\_ubyte *} argument, that is the current
pointer in the static memory; it usually is passed as one of the functions
arguments. Functions that need to allocate static memory,
i.e. memory that is not going to be released later and needs to be available
for the whole lifetime of the application, they can do so with the
\textbf{macro} call:

\txt{x\_alloc\_static\_mem(memory, number\_of\_bytes);}

This is a macro call that will take the address of the \txt{memory}
argument, it will return a pointer to a chunk of static memory with the size
of the argument \txt{number\_of\_bytes}. The \txt{memory} pointer will
be incremented with the number of bytes that has been allocated.

A small example can illustrate this:

\bcode
\begin{verbatim}
 1: 
 2: w_ubyte * do_some_initialization(w_ubyte * memory) {
 3:
 4:   char * buffer;
 5:   const int buffer_size = 128;
 6: 
 7:   buffer = x_alloc_static_mem(memory, buffer_size);
 8:   
 9:   ... do some other stuff maybe ...
10:  
11:   return memory;
12: 
13: }
\end{verbatim}
\ecode

Note that this macro can be used on other pieces of memory that are
statically allocated within the application.

When the call \txt{x\_os\_main} returns the \txt{mem} argument it was
passed, possibly allocating static memory from it; this value is then used by the
heap functions to define the start of the heap. So after the
\txt{x\_os\_main} function returns, this memory pointer should not be
fumbled with anymore, since the programmer could corrupt the heap.
