% ------------------------------------------------------------------------+
% Copyright (c) 2001 by Punch Telematix. All rights reserved.             |
%                                                                         |
% Redistribution and use in source and binary forms, with or without      |
% modification, are permitted provided that the following conditions      |
% are met:                                                                |
% 1. Redistributions of source code must retain the above copyright       |
%    notice, this list of conditions and the following disclaimer.        |
% 2. Redistributions in binary form must reproduce the above copyright    |
%    notice, this list of conditions and the following disclaimer in the  |
%    documentation and/or other materials provided with the distribution. |
% 3. Neither the name of Punch Telematix nor the names of other           |
%    contributors may be used to endorse or promote products derived      |
%    from this software without specific prior written permission.        |
%                                                                         |
% THIS SOFTWARE IS PROVIDED ``AS IS'' AND ANY EXPRESS OR IMPLIED          |
% WARRANTIES, INCLUDING, BUT NOT LIMITED TO, THE IMPLIED WARRANTIES OF    |
% MERCHANTABILITY AND FITNESS FOR A PARTICULAR PURPOSE ARE DISCLAIMED.    |
% IN NO EVENT SHALL PUNCH TELEMATIX OR OTHER CONTRIBUTORS BE LIABLE       |
% FOR ANY DIRECT, INDIRECT, INCIDENTAL, SPECIAL, EXEMPLARY, OR            |
% CONSEQUENTIAL DAMAGES (INCLUDING, BUT NOT LIMITED TO, PROCUREMENT OF    |
% SUBSTITUTE GOODS OR SERVICES; LOSS OF USE, DATA, OR PROFITS; OR         |
% BUSINESS INTERRUPTION) HOWEVER CAUSED AND ON ANY THEORY OF LIABILITY,   |
% WHETHER IN CONTRACT, STRICT LIABILITY, OR TORT (INCLUDING NEGLIGENCE    |
% OR OTHERWISE) ARISING IN ANY WAY OUT OF THE USE OF THIS SOFTWARE, EVEN  |
% IF ADVISED OF THE POSSIBILITY OF SUCH DAMAGE.                           |
% ------------------------------------------------------------------------+

%
% $Id: memory.tex,v 1.1.1.1 2004/07/12 14:07:44 cvs Exp $
%

\subsection{Memory Allocation}

\subsubsection{Operation}

\oswald provides mechanisms for dynamically allocating and releasing memory,
much like the well known \txt{malloc} and \txt{free} functions from
the standard C library\footnote{The memory allocation in \oswald is based on
the ideas of the very well known and very good memory routines of Doug Lea.
Doug, when we ever meet, I'll buy you a beer...}.
Besides allocating and freeing memory like malloc and free does, the memory
routines of \oswald provide extra functionality like a fast freeing function
with later scaveging of blocks, memory tagging and functions to walk over
the memory blocks and calling a user defined function on each block.

There are however, some issues that are specific for the \oswald dynamic
memory allocator.

\begin{itemize}
\item The largest memory chunk that can be allocated in a single call is
8388596 bytes or 8 Mb - 4 bytes, exactly. Limiting the blocks to this maximum size enables us
to implement memory tags, described in the next bullet. Note that this does
not mean that only 8 Mb is available; if more than 8 Mb of memory is
required to perform a job, several allocations need to be done one after
another. 
\item Memory tags are 9 bits of information that can be assigned to a block
of allocated memory, for which no extra space needs to be allocated. The 9
bit tags are contained in the normal internal memory householding
information. The tag bits can be used for whatever purpose the programmer
has in mind. 
\item \oswald provides a function, with a callback mechanism, that allows a
programmer to walk over all the allocated memory blocks. This feature, in
combination with the tag bits, can provide powerfull mechanisms, without
memory space overhead, to manipulate certain types of memory blocks. It can
reduce the number of referal pointers and thus memory space. It can for
instance we used to replace 'next' and 'previous' pointers in structures
that would require a linked list operation and for which speed is not that
important.
\item \oswald provides a \txt{x\_mem\_discard} function that very quickly marks a
memory block as garbage. It doesn't free the memory yet, at the moment of
the call. The freeing is performed by a call to \txt{x\_mem\_collect}
that coalesces and frees all garbage blocks in a single call. This reduces
locking and unlocking overhead and probably reduces fragmentation when the
collecting is done after a subsequent discarded of a whole slew of memory
blocks.
\end{itemize}

\subsubsection{Allocating Memory}

Memory is allocated by means of the following call:

\txt{void * x\_mem\_alloc(x\_size numbytes, x\_word id);}

which returns \txt{NULL} on failure or a non \txt{NULL} pointer when the allocation was
successful. Allocations larger than 8 Mb - 4 bytes always return \txt{NULL}. The
contents of the returned memory block are not cleared.

For obtaining a cleared memory block, the call:

\txt{void * x\_mem\_calloc(x\_size numbytes, x\_word id);}

can be used. The function internally uses a duff's device for quickly
clearing memory.

Note that both allocation functions require and ID to the passed. This is
the information that will be stored in the header of the block, in the tag
field. ID values 0 to 31 inclusive are reserved for \oswald.

\subsubsection{Reallocating Memory}

\txt{void * x\_mem\_realloc(void * block, x\_size numbytes);}

\subsubsection{Releasing Memory}

\txt{void x\_mem\_free(void * block);}

\subsubsection{Discarding Memory \& Collecting It}

\txt{void x\_mem\_discard(void * block);}

\txt{x\_status x\_mem\_collect(x\_size * bytes, x\_size * blocks);}

\subsubsection{Setting \& Reading Tags}

Memory tags are a 9 bit piece of information that can be used and that carry
no overhead in the memory usage. They are usefull to replace \txt{next}
and \txt{previous} pointer links in lists. When a structure is 'listed'
but doesn't require fast removal and insertion or lookup, the tag can be
used to store the type of the structure and the memory walker function can
be used to check for a certain tag and perform operations on the found
blocks. 

A tag can be set with the call:

\txt{x\_status x\_mem\_tag\_set(void * block, x\_word tag);}

The lower 9 bits of the tag will be stored in the memory chunk, the higher
bits above the first 9 bytes are silently discarded. The return value of
this call is the return status of the \txt{x\_monitor\_enter} or
\txt{x\_monitor\_exit} call that is performed inside, since the tag is
copied in the chunk after locking the memory functions.

The 9 bits can be read out by means of the call:

\txt{x\_word x\_mem\_tag\_get(void * block);}

Which will return the 9 bits of tag information in the lower 9 bits of the
returned word.

Note that \oswald reserves the values 0 to 31 inclusive, of the tag
information. With 9 bits of information, this leaves us with, $512 - 32 =
480$ user defined values that can be stored in the tag bits. Not much, but
they don't carry overhead either...

\subsubsection{Getting the Size of a Chunk}

The following function will return the useable size in bytes of a block of
memory.

\txt{x\_size x\_mem\_size(void * block);}

The value returned could be very well larger than the number of bytes that
has been requested for. Since all memory chunks size are 8 byte aligned, a
call to \txt{x\_mem\_alloc(22)} will yield a return value from this
function of 28. Also when the memory allocator is chopping up a block to
return in a call to e.g. \txt{x\_mem\_alloc} and the remainder block
would become smaller than 16 bytes, the block is not chopped up and the
internally recorded size will be larger than the argument to the allocator
function. In any case, it is safe to use the number of bytes returned by
this call for storing data.

\subsubsection{Iterating over Allocated Memory}

\oswald provides a function that can be used to walk over all the allocated
memory blocks.

\txt{x\_status x\_mem\_walk(x\_sleep to, void (*cb)(void * m, void * a), void * a);}

This function will walk over all the memory blocks and will call the
callback function \txt{cb} for each memory block that is not freed or
garbage. The arguments to the callback function are \txt{m} which is a
pointer to the in use memory block and \txt{a} a copy of the pointer
argument passed to the memory walk function, which can be used for whatever
purpose. 

The \txt{to} parameter is a timeout. Before all the blocks are scanned,
the memory allocator is locked and this timeout value is used to acquire the
memory allocator. Its meaning is the same as the timeout parameter of the
\txt{x\_monitor\_enter} function. After the walk, the lock is released.

The return status of the function is the return status of the monitor lock
that is used inside the walking function. Since the walker function provides
a security mechanism that prevents the same thread of calling it twice
(this could happen by the code in the callback function), the return status
can also be \txt{xs\_bad\_context} to indicate that same thread tried to
call the walker function more than once.

Note that you \textit{SHOULD NOT} call \txt{x\_mem\_free} from within the
callback function as it would change the internal lists and could crash.
There is no provision yet to secure for this situation\footnote{It would not
be difficult to do so, but would increase overhead during normal
operations.}. It is safe to call \txt{x\_mem\_discard} on the block since
this call does not change the interal lists.

\subsubsection{Pointer Checking}

A function that can be used to check if a certain pointer is pointing to a
memory block that has been allocated through any of the allocation routines
is:

\txt{x\_boolean x\_mem\_is\_block(void * pointer);}

It will return \txt{true} when the pointer points to memory that is
allocated and is not freed allready and is not garbage.
