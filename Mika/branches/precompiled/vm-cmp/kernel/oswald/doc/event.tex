% ------------------------------------------------------------------------+
% Copyright (c) 2001 by Punch Telematix. All rights reserved.             |
%                                                                         |
% Redistribution and use in source and binary forms, with or without      |
% modification, are permitted provided that the following conditions      |
% are met:                                                                |
% 1. Redistributions of source code must retain the above copyright       |
%    notice, this list of conditions and the following disclaimer.        |
% 2. Redistributions in binary form must reproduce the above copyright    |
%    notice, this list of conditions and the following disclaimer in the  |
%    documentation and/or other materials provided with the distribution. |
% 3. Neither the name of Punch Telematix nor the names of other           |
%    contributors may be used to endorse or promote products derived      |
%    from this software without specific prior written permission.        |
%                                                                         |
% THIS SOFTWARE IS PROVIDED ``AS IS'' AND ANY EXPRESS OR IMPLIED          |
% WARRANTIES, INCLUDING, BUT NOT LIMITED TO, THE IMPLIED WARRANTIES OF    |
% MERCHANTABILITY AND FITNESS FOR A PARTICULAR PURPOSE ARE DISCLAIMED.    |
% IN NO EVENT SHALL PUNCH TELEMATIX OR OTHER CONTRIBUTORS BE LIABLE       |
% FOR ANY DIRECT, INDIRECT, INCIDENTAL, SPECIAL, EXEMPLARY, OR            |
% CONSEQUENTIAL DAMAGES (INCLUDING, BUT NOT LIMITED TO, PROCUREMENT OF    |
% SUBSTITUTE GOODS OR SERVICES; LOSS OF USE, DATA, OR PROFITS; OR         |
% BUSINESS INTERRUPTION) HOWEVER CAUSED AND ON ANY THEORY OF LIABILITY,   |
% WHETHER IN CONTRACT, STRICT LIABILITY, OR TORT (INCLUDING NEGLIGENCE    |
% OR OTHERWISE) ARISING IN ANY WAY OUT OF THE USE OF THIS SOFTWARE, EVEN  |
% IF ADVISED OF THE POSSIBILITY OF SUCH DAMAGE.                           |
% ------------------------------------------------------------------------+

%
% $Id: event.tex,v 1.1.1.1 2004/07/12 14:07:44 cvs Exp $
%

\subsection{The Event Structure}

The event structure is the fundamental component in any element of \oswald
that is used to pass messages between threads (mutexes, queues, signals,
...). It is always the first field in any event component of \oswald and is
included in such a structure, not referred to.

Its definition is the following:

\bcode
\begin{verbatim}
 1: typedef struct x_Event * x_event;
 2:
 3: typedef struct x_Event {
 4:   volatile w_ushort flags_type;
 5:   volatile w_ushort n_competing;
 6:   volatile x_thread l_competing;
 7:   volatile x_event l_owned;
 8: } x_Event;
\end{verbatim}
\ecode

The relevant fields in the event structure are the following:

\begin{itemize}
\item \txt{x\_event$\rightarrow$type\_flags} The different flags to record the
state of the event, merged with the type of the event to save space. The type of an event is an
element of the \txt{x\_type} enumeration. This enumeration can be found
back in the \fl{event.h} header file.
\item \txt{x\_event$\rightarrow$n\_competing} The number of threads that
are competing on a change in this event.
\item \txt{x\_event$\rightarrow$l\_competing}
The list of threads that are competing for a change in this event. This list
is continued through the \txt{x\_thread$\rightarrow$l\_competing} field.
The event that a thread in this list is competing for, is recorded in the
\txt{x\_thread$\rightarrow$competing\_for} field.
\item \txt{x\_event$\rightarrow$l\_owned} Certain events can be owned by
a thread (mutexes or monitors). A thread that owns events is linked through
this field.
\end{itemize}

There is a special function that can be used to 'wait until an event has
been deleted'.

\txt{x\_status x\_event\_join(void * event, x\_window timeout);}

This function will make a thread wait untill the event (mutex, queue,
mailbox, ...) has been deleted. When the event has been deleted and no
threads are attempting any operation on it, this call returns
\txt{xs\_success}, if some threads were still trying to abort any
operation on the event, the call will return \txt{xs\_incomplete} to
indicate that some threads did not complete the call for the operation yet.

The calling thread will attempt to join the event for as long a number of
ticks as indicated by the \txt{timeout} argument.

This functionality could be used in a thread that allocates memory for the
element and creates it as a service for other threads. After allocation and
creation, the service thread can execute a join call on the event to wait until
all threads have stopped using the element.
